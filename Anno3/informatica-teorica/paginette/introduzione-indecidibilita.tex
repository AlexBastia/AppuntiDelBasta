% \begin{document}
\chapter{Introduzione a problemi e indecidibilita}

Studieremo due arie:
\begin{itemize}
  \item Calcolabilita: ci chiediamo se per quel problema esistera mai un algoritmo in grado di risolverlo (decidibilita' del problema). Non li incontriamo spesso irl dato che siamo piu' spinti a fare robe che sappiamo gestire. Bisogna studiare la struttura del problema per decidere se esiste o meno un algoritmo che lo risolve.
  \item Complessita: vogliamo determinare se un problema decidibile e' "facile" o "difficile", ovvero qual'e' la sua complessita' (diverso dalla complessita' degli algoritmi)
\end{itemize}

Quando diciamo "facile" intendiamo risolvibile in tempo polinomiale.

\dfn{Problema}{
  Relazione fra stringhe
}

\ex{Somma}{
  Dati $ x $ e $ y $ calcola $ x + y \to z $. E' una relazione binaria fra stringhe:
  \begin{itemize}
    \item $ <(2,3), 5> $
    \item $ <(4,3), 7> $
    \vdots
  \end{itemize}
  Dove la prima stringa e' l'input e la seconda e' l'output. Dato che elencare tutte le tuple e' impossibile, le descriviamo a parole ma stando attenti ad essere precisi. Ci servono tre elementi:
  \begin{itemize}
  \item What is input?
  \item What is output?
  \item Che relazione c'e' fra out e in?
  \end{itemize}
}

\ex{Halting}{
  Dati un programma e un input al programma, dare in output 'yes' se termina.
  Problema indecidibile come sappiamo bene. Si usa dimostrazione per diagonalizzazione.
}

% \end{document}
