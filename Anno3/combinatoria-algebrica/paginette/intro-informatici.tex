% \begin{document}
\chapter{Fondamenti di Algebra Astratta per Informatici}
Vediamo le strutture algebriche principali e i morfismi utilizzati nel corso. Notiamo come sta roba non ce la sta a spiegare nessuno se non il sommo Gem, dato che appunto siamo informatici.
\section{Gruppi}
\dfn{Gruppo}
{
  E' una coppia $ (G, \cdot) $ dove:
  \begin{itemize}
  \item $ G $ e' un insieme non vuoto
  \item $ \cdot $ e' un'operazione $ G \times G \to G $ (chiusa su $ G $)
  \end{itemize}

  Che soddsfa gli assiomi:
  \begin{itemize}
  \item Associativita'
  \item Esistenza dell'elemento \textit{neutro} $e$
  \item Esistenza dell'\textit{inverso} $a^{-1}$ per ogni $ a \in G $
  \end{itemize}
}

Alcune delle proprieta' piu' importanti dei gruppi sono:
\begin{itemize}
\item \textbf{Unicita'} dell'elemento inverso e neutro
\item \textbf{Inverso del prodotto}: $ (a \cdot b)^{-1} = b^{-1} \cdot a^{-1} $
\item \textbf{Legge della cancellazione}: $ a \cdot b = a \cdot c $ moltiplicando a sx per $ a^{-1} $ si ottiene
  $ b = c $
\end{itemize}

\nt{
    Se vale anche la proprieta' \textit{commutativa}, allora il gruppo si dice \textit{abeliano}.
}

\ex{Gruppo di addizione sugli interi}{
  Un esempio classico e intuitivo \`e l'insieme dei numeri interi $\mathbb{Z}$ associato all'operazione di addizione $+$:
  \[
    (\mathbb{Z}, +)
  \]
  E' facile dimostrare che l'addizione sugli interi rispetta tutti gli assimi per essere un gruppo:
  \begin{enumerate}
    \item \textbf{Chiusura:} La somma di due numeri interi restituisce sempre un numero intero.
    \[ 
    \forall a, b \in \mathbb{Z}, \quad (a + b) \in \mathbb{Z} 
    \]
    
    \item \textbf{Associativit\`a:} L'ordine di raggruppamento delle addizioni non influisce sul risultato.
    \[ 
    \forall a, b, c \in \mathbb{Z}, \quad (a + b) + c = a + (b + c) 
    \]
    
    \item \textbf{Esistenza dell'Elemento Neutro:} Esiste un elemento, lo $0$, che addizionato a qualsiasi altro intero lo lascia inalterato.
    \[ 
    \exists 0 \in \mathbb{Z} \text{ tale che } \forall a \in \mathbb{Z}, \quad a + 0 = 0 + a = a 
    \]
    
    \item \textbf{Esistenza dell'Elemento Inverso:} Per ogni intero $a$, esiste il suo inverso additivo (l'opposto) $-a$, tale che la loro somma dia l'elemento neutro.
    \[ 
    \forall a \in \mathbb{Z}, \exists (-a) \in \mathbb{Z} \text{ tale che } a + (-a) = (-a) + a = 0 
    \]
\end{enumerate}

Inoltre, l'addizione gode anche della proprieta' commutativa, quindi possiamo dire che $ (\mathbb{Z},+) $ e' anche un gruppo \textit{abeliano}.
}

Vediamo ora cosa sono i sottogruppi:
\dfn{Sottogruppo}{
  Dato un gruppo $ (G, \cdot) $, si dice che $ B $ e' un suo sottogruppo se:
  \begin{itemize}
    \item $ B \subset G $
    \item $ (B, \cdot) $ e' un gruppo 
  \end{itemize}
}

\ex{Sottogruppo somma con interi pari}{
  Un esempio classico, partendo dal gruppo degli interi $(\mathbb{Z}, +)$, \`e l'insieme dei numeri interi pari, indicato con $2\mathbb{Z}$:
\[
2\mathbb{Z} = \{ 2k \mid k \in \mathbb{Z} \} = \{ \dots, -4, -2, 0, 2, 4, \dots \}
\]
Invece di riverificare tutti gli assiomi, per dimostrare che $(2\mathbb{Z}, +)$ \`e un sottogruppo di $(\mathbb{Z}, +)$, \`e sufficiente applicare il Criterio del Sottogruppo e verificare le seguenti tre condizioni:

\begin{enumerate}
    \item \textbf{Non vuotezza (Elemento neutro):} L'elemento neutro del gruppo principale, lo $0$, deve appartenere al sottoinsieme. 
    \[ 
    0 = 2 \cdot 0 \implies 0 \in 2\mathbb{Z} 
    \]
    Questo ci assicura anche che l'insieme non sia vuoto.
    
    \item \textbf{Chiusura rispetto all'operazione:} La somma di due numeri pari \`e ancora un numero pari. Siano $a, b \in 2\mathbb{Z}$; allora esistono due interi $m, n \in \mathbb{Z}$ tali che $a = 2m$ e $b = 2n$.
    \[ 
    a + b = 2m + 2n = 2(m + n) 
    \]
    Poich\'e $(m + n) \in \mathbb{Z}$, deduciamo che la somma $(a + b)$ appartiene ancora a $2\mathbb{Z}$.
    
    \item \textbf{Chiusura rispetto all'inverso:} L'opposto di un numero pari \`e a sua volta pari. Sia $a = 2m \in 2\mathbb{Z}$. Il suo inverso additivo \`e:
    \[ 
    -a = -(2m) = 2(-m) 
    \]
    Poich\'e $-m \in \mathbb{Z}$, allora anche l'inverso additivo $-a$ appartiene a $2\mathbb{Z}$.
\end{enumerate}

\noindent
\textbf{Conclusione:} Poich\'e le condizioni sono soddisfatte, $(2\mathbb{Z}, +)$ \`e a tutti gli effetti un sottogruppo di $(\mathbb{Z}, +)$. In notazione algebrica, questo si indica spesso come $2\mathbb{Z} \leq \mathbb{Z}$.
}

Un importante teorema per i sottogruppi: TODO finisci
\thm{Lagrange}{
  Se $ G $ e' un gruppo finito e $ H $ un suo sottogruppo, allora la cardinalita' degli elementi di $ G $ divide esattamente 
}

\subsection{Centro di un Gruppo}
\dfn{Centro di un Gruppo}{
Sia $G$ un gruppo. Il \textbf{centro} di $G$, tipicamente denotato con $Z(G)$, è l'insieme di tutti gli elementi di $G$ che commutano con ogni elemento del gruppo stesso. In simboli:
$$ Z(G) = \{ z \in G \mid z \cdot g = g \cdot z, \forall g \in G \} $$
}

\nt{
\textbf{Proprietà fondamentali ed Esempi:}
\begin{itemize}
    \item \textbf{È un sottogruppo:} L'elemento neutro $e$ commuta con tutto, quindi $e \in Z(G)$. Essendo chiuso rispetto al prodotto e all'inverso, costituisce un sottogruppo a tutti gli effetti ($Z(G) \le G$).
    \item \textbf{È un sottogruppo normale:} Poiché ogni elemento $z \in Z(G)$ commuta con tutti i $g \in G$, la coniugazione lo lascia invariato: $g z g^{-1} = z g g^{-1} = z \in Z(G)$. Di conseguenza, $Z(G) \trianglelefteq G$.
    \item \textbf{Casi limite:} Se $G$ è abeliano, il centro coincide con tutto il gruppo ($Z(G) = G$). Se invece $Z(G) = \{e\}$, si dice che il gruppo ha centro banale (es. il gruppo simmetrico $S_n$ per $n \ge 3$).
    \item \textbf{Applicazione in Combinatoria Algebrica:} Il centro del gruppo generale lineare $GL(V)$ è costituito esattamente dalle matrici scalari non nulle ($Z = \{\lambda I \mid \lambda \in K^\times\}$). Questo fatto è il motore logico della dimostrazione del \textbf{Lemma di Schur}.
\end{itemize}
}

\ex{Centro del gruppo dei quaternioni}{
  Consideriamo il gruppo dei quaternioni $Q_8$, un noto gruppo non abeliano di ordine 8, i cui elementi sono:
\[
Q_8 = \{ 1, -1, i, -i, j, -j, k, -k \}
\]
Le operazioni in questo gruppo seguono le regole di moltiplicazione $i^2 = j^2 = k^2 = -1$ e $ij = k$, $ji = -k$ (eccetera).

Vogliamo determinare il centro del gruppo, indicato con $Z(Q_8)$. Analizziamo gli elementi:

\begin{itemize}
    \item Gli elementi $1$ e $-1$ commutano con qualsiasi altro elemento all'interno di $Q_8$. Ad esempio, $1 \cdot i = i \cdot 1$ e $(-1) \cdot j = j \cdot (-1)$.
    \item Gli elementi $\pm i, \pm j, \pm k$ \textbf{non} commutano con tutti gli elementi del gruppo. Come si evince dalle regole di moltiplicazione, l'ordine dei fattori conta: $ij = k$, ma $ji = -k$, perci\`o $ij \neq ji$.
\end{itemize}

Di conseguenza, gli unici elementi che commutano con tutto $Q_8$ sono $1$ e $-1$. Il centro del gruppo dei quaternioni \`e quindi il sottogruppo:
\[
Z(Q_8) = \{ 1, -1 \}
\]
}

\subsection{Sottogruppi Normali}
Dobbiamo prima definire un'operazione che ci servira'
\dfn{Operazione di Coniugio}{
Sia $G$ un gruppo e siano $x, g \in G$. Si definisce \textbf{coniugio} di $x$ tramite $g$ l'operazione che associa ad $x$ l'elemento:
$$ x^g = gxg^{-1} $$
Due elementi $x, y \in G$ si dicono \textbf{coniugati} se esiste un elemento $g \in G$ tale che $y = gxg^{-1}$. Questa è una relazione di equivalenza che partiziona il gruppo in \textbf{classi di coniugio}.
}

\nt{
\textbf{Osservazioni e Proprietà:}
\begin{itemize}
    \item \textbf{Automorfismo Interno:} Per ogni $g \in G$, la mappa $\gamma_g: G \to G$ definita da $\gamma_g(x) = gxg^{-1}$ è un automorfismo del gruppo (chiamato automorfismo interno). Questo significa che il coniugio preserva tutte le proprietà algebriche dell'elemento (ad esempio, $x$ e $gxg^{-1}$ hanno sempre lo stesso ordine).
    \item \textbf{Nei Gruppi Abeliani:} Se $G$ è commutativo, il coniugio è banale: $gxg^{-1} = xgg^{-1} = x$. In questo caso, ogni elemento forma una classe di coniugio a sé stante.
    \item \textbf{Invarianza dei Caratteri:} Questa è la proprietà più importante per il Modulo 2. I caratteri di una rappresentazione sono \textbf{funzioni di classe}, ovvero assumono lo stesso valore su tutti gli elementi di una stessa classe di coniugio: $\chi(x) = \chi(gxg^{-1})$.
\end{itemize}
}

Possiamo ora definire cosa sono i sottogruppi normali
\dfn{Sottogruppo Normale}{
Sia $G$ un gruppo e $N$ un suo sottogruppo ($N \le G$). Diciamo che $N$ è un \textbf{sottogruppo normale} di $G$, e si denota con il simbolo $N \trianglelefteq G$, se è invariante rispetto all'operazione di coniugio per qualsiasi elemento del gruppo. 
In formule, deve valere:
$$g n g^{-1} \in N \quad \forall n \in N, \forall g \in G$$
}

\nt{
\textbf{Condizioni Equivalenti}
Nella pratica algebrica, dire che $N \trianglelefteq G$ equivale a verificare una di queste due proprietà:
\begin{itemize}
    \item \textbf{Invarianza globale per coniugio:} $g N g^{-1} = N$ per ogni $g \in G$.
    \item \textbf{Coincidenza dei laterali:} I laterali sinistri coincidono sempre con i laterali destri. Ovvero, $gN = Ng$ per ogni $g \in G$. (Attenzione: questo non significa che gli elementi commutino individualmente, cioè $gn = ng$, ma che gli \textit{insiemi} risultanti siano identici).
\end{itemize}

\textbf{Esempi e Proprietà Fondamentali}
\begin{itemize}
    \item \textbf{Gruppi Abeliani:} Se il gruppo $G$ è commutativo (come i gruppi ciclici o il Gruppo di Klein $V_4$), allora \textit{ogni} suo sottogruppo è banalmente normale, poiché $gng^{-1} = ngg^{-1} = n$.
    \item \textbf{Nucleo di un Omomorfismo:} Il nucleo di un qualsiasi omomorfismo $\phi: G \to H$ è sempre un sottogruppo normale di $G$ ($\ker(\phi) \trianglelefteq G$).
    \item \textbf{Il Centro del Gruppo:} Il centro $Z(G)$, contenendo gli elementi che commutano con tutto, è sempre un sottogruppo normale di $G$.
\end{itemize}

\textbf{Il Fine Ultimo: Il Gruppo Quoziente}
La normalità è la condizione necessaria e sufficiente per poter definire un'operazione coerente sull'insieme dei laterali $\{gN \mid g \in G\}$. Solo se $N \trianglelefteq G$, il prodotto $(aN) \cdot (bN) = (ab)N$ è ben definito. 
Questo ci permette di creare il \textbf{Gruppo Quoziente} $G/N$, una struttura fondamentale che "semplifica" il gruppo di partenza collassando tutto il sottogruppo $N$ nell'elemento neutro.
}


\subsection{Gruppi Simmetrici}
\dfn{Il Gruppo Simmetrico}{
Sia $X$ un insieme. Il \textbf{Gruppo Simmetrico} di $X$, denotato con $Sym(X)$ o $\text{Perm}(X)$, è l'insieme di tutte le funzioni biunivoche (permutazioni) $f: X \to X$. 
Sotto l'operazione di composizione di funzioni, $Sym(X)$ forma un gruppo.
}

\mprop{Isomorfismo fra gruppi simmetrici}{
    Si puo' dimostrare che gruppi simmetrici di insiemi aventi la stessa cardinalita' $n$ sono isomorfi, quindi si tende a considerare il gruppo simmetrico costituito dalle permutazioni degli interi $1,2,\dots,n$ denotato $S_n$.
}

\thm{Classi di Coniugio del gruppo simmetrico}{
Dato un gruppo simmetrico $S_n$, due permutazioni $\sigma, \tau \in S_n$ appartengono alla stessa \textbf{classe di coniugio} se e solo se hanno la stessa \textbf{struttura ciclica}, ovvero se presentano lo stesso numero di cicli della stessa lunghezza nella loro scomposizione in cicli disgiunti.
}

Un \textbf{ciclo} (o permutazione ciclica) è un tipo speciale, e molto semplice, di permutazione.

Intuitivamente, un ciclo prende un sottoinsieme di elementi e li ``fa ruotare'' di una posizione, lasciando tutti gli altri elementi dell'insieme perfettamente immobili al loro posto.

\dfn{Permutazione ciclica e Punto fisso}{
Sia $X$ un insieme. Un ciclo di lunghezza $k$ (chiamato anche \textit{$k$-ciclo}) è una permutazione $\sigma$ tale per cui esistono $k$ elementi distinti $x_1, x_2, \dots, x_k$ in $X$ per i quali vale:
\begin{itemize}
    \item $\sigma(x_1) = x_2$
    \item $\sigma(x_2) = x_3$
    \item $\dots$
    \item $\sigma(x_{k-1}) = x_k$
    \item $\sigma(x_k) = x_1$
\end{itemize}
Per ogni altro elemento $y \in X$ che non fa parte di questo sottoinsieme, il ciclo non ha alcun effetto, ovvero $\sigma(y) = y$. In questo caso si dice che $y$ è un punto fisso della permutazione.
}

\nt{
Invece di scrivere la funzione o la matrice per esteso, in algebra si usa una notazione molto compatta. Il ciclo appena descritto si scrive semplicemente racchiudendo gli elementi interessati tra parentesi tonde, separati da spazi:
\[
    \sigma = (x_1\ x_2\ \dots\ x_k)
\]
Questa scrittura si legge tipicamente da sinistra a destra: ogni elemento viene mandato in quello alla sua destra, e l'ultimo elemento chiude il cerchio venendo rimandato al primo.
}

\ex{Cicli di $ S_5 $}{
Consideriamo il gruppo simmetrico $S_5$, ovvero l'insieme delle permutazioni sull'insieme $X = \{1, 2, 3, 4, 5\}$. 
Definiamo il ciclo $\pi = (1\ 3\ 5)$.

Ecco cosa fa esattamente questa permutazione quando viene applicata agli elementi di $X$:
\begin{itemize}
    \item Manda $1$ in $3$: $\pi(1) = 3$
    \item Manda $3$ in $5$: $\pi(3) = 5$
    \item Manda $5$ in $1$: $\pi(5) = 1$
\end{itemize}
E gli elementi $2$ e $4$? Poiché non compaiono esplicitamente tra le parentesi del ciclo, la regola stabilisce che essi restino fissi:
\begin{itemize}
    \item $\pi(2) = 2$
    \item $\pi(4) = 4$
\end{itemize}

\textbf{Nota bene:} Il ciclo $(1\ 3\ 5)$ rappresenta esattamente la stessa funzione dei cicli $(3\ 5\ 1)$ e $(5\ 1\ 3)$. Poiché si tratta di un ``girotondo'', non importa da quale elemento si inizia a scrivere, purché si rispetti l'ordine sequenziale delle trasformazioni!
}

\nt{
\textbf{Proprietà e Relazione con le Partizioni:}
\begin{itemize}
    \item \textbf{Corrispondenza biunivoca:} Le classi di coniugio di $S_n$ sono in corrispondenza biunivoca con le \textbf{partizioni} dell'intero $n$. Una partizione $\lambda = (\lambda_1, \lambda_2, \dots, \lambda_k)$ tale che $\sum \lambda_i = n$ definisce univocamente una classe di coniugio.
    \item \textbf{Numero di Rappresentazioni:} In virtù della teoria generale, il numero di rappresentazioni irriducibili distinte di $S_n$ su $\mathbb{C}$ è esattamente uguale al numero di partizioni $p(n)$.
    \item \textbf{Esempio $S_3$ ($n=3$):} Le partizioni di 3 sono:
    \begin{itemize}
        \item $(1,1,1) \to$ Identità (cicli di lunghezza 1).
        \item $(2,1) \to$ Trasposizioni $\{(12), (13), (23)\}$.
        \item $(3) \to$ 3-cicli $\{(123), (132)\}$.
    \end{itemize}
    \item \textbf{Rappresentazioni Notevoli:} Ogni $S_n$ possiede sempre almeno due rappresentazioni di grado 1: la \textit{banale} ($\chi(g)=1$ per ogni $g$) e la \textit{segnatura} ($\chi(g)=\text{sgn}(g)$).
    \item \textbf{Diagrammi di Young:} Le rappresentazioni irriducibili di $S_n$ vengono classificate e costruite tramite i \textbf{Diagrammi di Young}, che sono la rappresentazione grafica delle partizioni di $n$.
\end{itemize}
}

\subsection{Gruppi Lineari Generali}

\dfn{Il Gruppo Lineare Generale di uno Spazio Vettoriale}{
Sia $V$ uno spazio vettoriale su un campo $K$. Il \textbf{Gruppo Lineare Generale} di $V$, denotato con $GL(V)$ o $\text{Aut}(V)$, è l'insieme di tutti gli \textbf{automorfismi lineari} dello spazio $V$ (ovvero, tutte le applicazioni lineari biunivoche $f: V \to V$).

La struttura $(GL(V), \circ)$ forma un gruppo dove l'operazione interna è la \textbf{composizione di funzioni}:
\begin{itemize}
    \item \textbf{Chiusura:} La composizione di due automorfismi lineari è ancora un automorfismo lineare.
    \item \textbf{Elemento neutro:} L'applicazione identica $\text{id}_V$ (tale che $\text{id}_V(v) = v, \forall v \in V$).
    \item \textbf{Inverso:} L'applicazione lineare inversa $f^{-1}$, che esiste sempre ed è unica poiché $f$ è una biiezione.
\end{itemize}
}

\nt{
\textbf{Distinzione tra $Sym(V)$ e $GL(V)$:}
\begin{itemize}
    \item \textbf{Natura delle trasformazioni:} Mentre $Sym(V)$ contiene \textit{qualsiasi} funzione biettiva (anche quelle che "rimescolano" i vettori in modo selvaggio e non lineare), il gruppo $GL(V)$ è il sottogruppo di $Sym(V)$ costituito solo dalle trasformazioni che sono anche \textbf{lineari}.
    \item \textbf{Inclusione:} $GL(V) \le Sym(V)$. In termini di Teoria delle Rappresentazioni, diciamo che una rappresentazione è un'azione di $G$ su $V$ tale che l'immagine dell'omomorfismo non sia semplicemente in $Sym(V)$, ma sia contenuta interamente in $GL(V)$.
    \item \textbf{Esempio concettuale:} Se $V = \mathbb{R}^2$, una funzione che sposta il vettore $(1,1)$ in $(2,2)$ e il vettore $(2,2)$ in $(5,0)$ può appartenere a $Sym(V)$ (se è biettiva), ma non potrà mai appartenere a $GL(V)$ perché non rispetta la proporzionalità ($\text{linearità}$).
    \item \textbf{Il "filtro" della Rappresentazione:} Quando scriviamo $\rho: G \to GL(V)$, stiamo imponendo che ogni simmetria del gruppo $G$ agisca sullo spazio $V$ rispettando la sua struttura vettoriale (somma e prodotto per scalare), non solo come un semplice rimescolamento di punti.
\end{itemize}
}

\subsubsection{Rappresentazione Matriciale $GL(n, K)$}
Se lo spazio vettoriale $V$ ha dimensione finita $n$, fissata una base di $V$, ogni isomorfismo lineare può essere rappresentato univocamente da una matrice quadrata di ordine $n$. Di conseguenza, $GL(V)$ è isomorfo al gruppo delle matrici invertibili a coefficienti nel campo $K$:
$$ GL(n, K) = \{ A \in M_{n}(K) \mid \det(A) \neq 0 \} $$
In questa veste, l'operazione del gruppo diventa la \textbf{moltiplicazione riga per colonna} tra matrici e l'elemento neutro è la matrice identità $I_n$.

\subsubsection{Proprietà Fondamentali}
\begin{enumerate}
    \item \textbf{Non Abelianità:} Se la dimensione $n \ge 2$, il gruppo $GL(V)$ è tipicamente \textbf{non commutativo} (poiché il prodotto di matrici non commuta in generale).
    \item \textbf{Il Centro del Gruppo:} Il centro $Z(GL(V))$ (ovvero l'insieme degli elementi che commutano con ogni altro elemento del gruppo) è costituito esattamente dalle \textbf{matrici scalari} non nulle: $Z = \{\lambda I_n \mid \lambda \in K^\times\}$. Questo fatto è il motore logico del \textbf{Lemma di Schur}.
    \item \textbf{Sottogruppo Speciale Lineare:} Il nucleo dell'omomorfismo determinante ($\det: GL(n, K) \to K^\times$) forma un importante sottogruppo normale di $GL(n, K)$, chiamato \textit{Gruppo Speciale Lineare} $SL(n, K)$, composto da tutte e sole le matrici con determinante uguale a $1$.
\end{enumerate}

\subsubsection{Il Ruolo Centrale nel Modulo 2}
Questa definizione è il perno del corso di Combinatoria Algebrica. Definire una rappresentazione di un gruppo finito astratto $G$ su uno spazio vettoriale $V$ significa esattamente stabilire un omomorfismo:
$$ \rho: G \to GL(V) $$
Stiamo, di fatto, "traducendo" la struttura moltiplicativa del gruppo astratto $G$ in operazioni tra matrici invertibili, permettendoci così di sfruttare tutta la potenza dell'Algebra Lineare (autovalori, traccia, diagonalizzazione) per studiare le simmetrie del gruppo.

\subsection{Gruppi Ciclici}
\dfn{Gruppo Ciclico}{
Un gruppo $(G, \cdot)$ si dice \textbf{ciclico} se esiste un elemento $g \in G$, detto \textbf{generatore}, tale che ogni elemento di $G$ possa essere espresso come potenza intera di $g$:
$$G = \langle g \rangle = \{g^k \mid k \in \mathbb{Z}\}$$
}

\subsubsection*{Classificazione e Struttura}
I gruppi ciclici sono classificati in base al loro ordine $|G|$:
\begin{itemize}
    \item Se $|G| = \infty$, allora $G \cong (\mathbb{Z}, +)$.
    \item Se $|G| = n < \infty$, allora $G \cong (\mathbb{Z}/n\mathbb{Z}, +)$, ovvero il gruppo delle classi di resto modulo $n$.
\end{itemize}

\subsubsection*{Proprietà Fondamentali}
\begin{enumerate}
    \item \textbf{Abelianità:} Ogni gruppo ciclico è abeliano. Infatti, $g^a \cdot g^b = g^{a+b} = g^{b+a} = g^b \cdot g^a$.
    \item \textbf{Sottogruppi:} Ogni sottogruppo di un gruppo ciclico è a sua volta ciclico.
    \item \textbf{Teorema dei Divisori:} Se $G$ è un gruppo ciclico di ordine $n$, allora per ogni divisore $d$ di $n$ esiste un unico sottogruppo $H \le G$ tale che $|H| = d$.
    \item \textbf{Generatori:} Un elemento $g^k$ di un gruppo ciclico d'ordine $n$ è un generatore di $G$ se e solo se $\gcd(k, n) = 1$. Il numero di tali generatori è dato dalla funzione $\varphi(n)$ di Eulero.
\end{enumerate}

\ex{Gruppo Ciclico $ \mathbb{Z}_6 $}{
  Consideriamo il gruppo degli interi modulo 6 rispetto all'addizione, indicato con $\mathbb{Z}_6$. I suoi elementi sono le classi di resto modulo 6:
\[
\mathbb{Z}_6 = \{ \overline{0}, \overline{1}, \overline{2}, \overline{3}, \overline{4}, \overline{5} \}
\]
\subsubsection*{Verifica del Generatore}

Verifichiamo se l'elemento $\overline{1}$ \`e un generatore di $\mathbb{Z}_6$. Calcoliamo i suoi multipli (poich\'e l'operazione \`e l'addizione, sommiamo l'elemento a se stesso):

\begin{itemize}
    \item $1 \cdot \overline{1} = \overline{1}$
    \item $2 \cdot \overline{1} = \overline{1} + \overline{1} = \overline{2}$
    \item $3 \cdot \overline{1} = \overline{1} + \overline{1} + \overline{1} = \overline{3}$
    \item $4 \cdot \overline{1} = \overline{4}$
    \item $5 \cdot \overline{1} = \overline{5}$
    \item $6 \cdot \overline{1} = \overline{6} \equiv \overline{0} \pmod{6}$
\end{itemize}

Poich\'e partendo da $\overline{1}$ siamo riusciti a ottenere l'intero insieme $\mathbb{Z}_6$, possiamo affermare che $\mathbb{Z}_6$ \`e un gruppo ciclico generato da $\overline{1}$. In notazione algebrica, scriviamo:
\[
\mathbb{Z}_6 = \langle \overline{1} \rangle
\]

\noindent
\textbf{Nota:} In un gruppo ciclico possono esserci pi\`u generatori. Nel caso di $\mathbb{Z}_6$, anche $\overline{5}$ \`e un generatore ($\langle \overline{5} \rangle = \mathbb{Z}_6$), mentre elementi come $\overline{2}$ generano solo sottogruppi (nello specifico, il sottogruppo $\{ \overline{0}, \overline{2}, \overline{4} \}$).
}

\subsection{Azioni}
\dfn{Azione di un Gruppo}{
Sia $G$ un gruppo e $X$ un insieme non vuoto. Un' \textbf{azione} (a sinistra) di $G$ su $X$ è una funzione $\cdot: G \times X \to X$ che associa a ogni coppia $(g, x)$ un elemento $g \cdot x \in X$, tale che siano soddisfatti i seguenti assiomi:
\begin{enumerate}
    \item \textbf{Identità:} $e \cdot x = x$ per ogni $x \in X$ (dove $e$ è l'elemento neutro di $G$).
    \item \textbf{Compatibilità:} $(gh) \cdot x = g \cdot (h \cdot x)$ per ogni $g, h \in G$ e $x \in X$.
\end{enumerate}
}

\nt{
\textbf{Concetti Chiave e Proprietà:}
\begin{itemize}
    \item \textbf{Omomorfismo di Permutazione:} Un'azione di $G$ su $X$ è equivalente a un omomorfismo di gruppi $\phi: G \to \text{Sym}(X)$. In questo senso, ogni elemento del gruppo viene visto come una permutazione degli elementi di $X$.
    \item \textbf{Orbita:} L'orbita di un elemento $x \in X$ è l'insieme $G \cdot x = \{g \cdot x \mid g \in G\}$. Le orbite formano una partizione dell'insieme $X$.
    \item \textbf{Stabilizzatore:} Lo stabilizzatore di $x \in X$ è il sottogruppo $G_x = \{g \in G \mid g \cdot x = x\}$. Contiene tutti gli elementi del gruppo che "lasciano fermo" $x$.
    \item \textbf{Teorema Orbita-Stabilizzatore:} Se $G$ è finito, la cardinalità dell'orbita di $x$ è data dal numero di laterali dello stabilizzatore: $|G \cdot x| = |G| / |G_x|$.
    \item \textbf{Dall'Azione alla Rappresentazione:} Se l'insieme $X$ è uno spazio vettoriale $V$ e l'azione è lineare (cioè $g \cdot (v+w) = g \cdot v + g \cdot w$ e $g \cdot (\lambda v) = \lambda (g \cdot v)$), allora l'azione è esattamente una \textbf{rappresentazione lineare} di $G$.
\end{itemize}
}

\ex{}{
  Un esempio classico ed estremamente intuitivo \`e l'azione del gruppo diedrale $D_4$ (il gruppo delle simmetrie del quadrato) sull'insieme dei suoi vertici.

\subsubsection*{Definizione del Gruppo e dell'Insieme}

\begin{itemize}
    \item \textbf{L'insieme $X$:} Consideriamo i quattro vertici di un quadrato numerati in senso antiorario: 
    \[
    X = \{1, 2, 3, 4\}
    \]
    \item \textbf{Il gruppo $G$:} Consideriamo $D_4$, che contiene 8 elementi: 4 rotazioni (di $0^\circ, 90^\circ, 180^\circ, 270^\circ$) e 4 riflessioni (rispetto agli assi verticale, orizzontale e alle due diagonali).
\end{itemize}

\subsubsection*{Come agisce il gruppo sull'insieme}

L'azione consiste nell'applicare la trasformazione geometrica al quadrato e osservare dove finisce ciascun vertice. 

Prendiamo come esempio l'elemento $r \in D_4$, che rappresenta la \textbf{rotazione di $90^\circ$ in senso antiorario}. La sua azione sui vertici sar\`a:
\begin{itemize}
    \item $r \cdot 1 = 2$ (il vertice 1 si sposta nella posizione del vertice 2)
    \item $r \cdot 2 = 3$
    \item $r \cdot 3 = 4$
    \item $r \cdot 4 = 1$
\end{itemize}

Prendiamo invece l'elemento $s \in D_4$, che rappresenta la \textbf{riflessione rispetto all'asse verticale}. Supponendo che i vertici 1 e 4 siano a destra e 2 e 3 a sinistra, la sua azione sar\`a:
\begin{itemize}
    \item $s \cdot 1 = 2$ e $s \cdot 2 = 1$ (i vertici in alto si scambiano)
    \item $s \cdot 3 = 4$ e $s \cdot 4 = 3$ (i vertici in basso si scambiano)
\end{itemize}

\noindent
\textbf{Verifica degli assiomi:} Questa operazione soddisfa i due assiomi fondamentali delle azioni di gruppo. L'elemento neutro (la rotazione di $0^\circ$) lascia ogni vertice al suo posto ($e \cdot x = x$), e combinare due simmetrie (ad esempio ruotare e poi riflettere) equivale ad applicare direttamente la simmetria risultante dalla loro composizione geometrica: $g \cdot (h \cdot x) = (gh) \cdot x$.
}

\section{Campo}
\dfn{Campo}{
Un \textbf{campo} $ K $ e' una struttura algebrica dotata di due operazioni:
\begin{itemize}
  \item \textit{Somma}: t.c. $ (K, +) $ e' un \textit{gruppo abeliano} (con elem neutro 0)
  \item \textit{Prodotto}: t.c. $ (K \setminus \{0\}, \cdot ) $ e' un \textit{gruppo abeliano} (con elem neutro 1)
  \item Vale la proprieta'  distributiva del prodotto rispetto alla somma
\end{itemize}
}

Come conseguenza diretta, si ha che l'elemento neutro della somma diventa \textbf{elemento assorbente} ($ \forall a \in K. a \cdot 0 = 0 $). Infatti, se un prodotto e' 0 allora e' sicuro che almeno uno degli operandi e' 0.

\ex{Campo $(\mathbb{Z}_5, +, \cdot)$}{
  Un ottimo esempio di campo finito \`e l'insieme delle classi di resto modulo 5, indicato con $\mathbb{Z}_5 = \{\overline{0}, \overline{1}, \overline{2}, \overline{3}, \overline{4}\}$, equipaggiato con le consuete operazioni di addizione e moltiplicazione modulo 5.

Poich\'e 5 \`e un numero primo, la struttura $(\mathbb{Z}_5, +, \cdot)$ \`e un campo. La caratteristica chiave che lo distingue da altre strutture (come l'anello $\mathbb{Z}_6$) \`e che \textbf{ogni elemento diverso da zero possiede un inverso moltiplicativo}.

\subsubsection*{Verifica degli inversi moltiplicativi in $\mathbb{Z}_5$}

Per ogni elemento $a \in \mathbb{Z}_5 \setminus \{\overline{0}\}$, esiste un elemento $b$ tale che $a \cdot b = \overline{1}$:

\begin{itemize}
    \item L'inverso di $\overline{1}$ \`e $\overline{1}$, poich\'e $\overline{1} \cdot \overline{1} = \overline{1}$.
    \item L'inverso di $\overline{2}$ \`e $\overline{3}$, poich\'e $\overline{2} \cdot \overline{3} = \overline{6} \equiv \overline{1} \pmod{5}$.
    \item L'inverso di $\overline{3}$ \`e $\overline{2}$, essendo la moltiplicazione commutativa ($\overline{3} \cdot \overline{2} = \overline{1}$).
    \item L'inverso di $\overline{4}$ \`e $\overline{4}$, poich\'e $\overline{4} \cdot \overline{4} = \overline{16} \equiv \overline{1} \pmod{5}$.
\end{itemize}

\noindent
\textbf{Campi infiniti:} Gli esempi pi\`u classici e utilizzati di campi con infiniti elementi sono il campo dei numeri razionali $(\mathbb{Q}, +, \cdot)$, il campo dei numeri reali $(\mathbb{R}, +, \cdot)$ e il campo dei numeri complessi $(\mathbb{C}, +, \cdot)$.
}

\subsection{Proprieta' Fondamentali}
\textbf{Caratteristica del Campo} $ (char(K)) $: E' il piu' piccol intero positivo $ p $ tale che sommando l'elemento identita' 1 a se' stesso $ p $ volte si ottiene 0 (l'elemento neutro) (cioe' $ 1 + ... + 1 $ = 0). Se non esiste un valore $ p $ (non si ritorna mai all'elemento neutro) allora $ char(K) = 0 $.

\textbf{Chiusura Algebrica}: Un campo $ K $ si dice algebricamentechiuso se ogni polinomio non costante a coefficienti in $ K $ ha almeno una raidce in $ K $. Il campo $ \mathbb{C} $ e' algebricamente chiuso, mentre $ \mathbb{R} $ non lo e' (ad esempio, $ x^2 + 1 = 0 $).

\section{Anelli}
\dfn{Anello}{
  Un anello e' una terna $ (R, +, \cdot) $, dove $ R $ e' un insieme dotato di due operazioni binarie interne (somma e prodotto), tale che:
  \begin{itemize}
    \item $ (R, +) $ e' un \textit{gruppo abeliano}
    \item $ (R, \cdot) $ e' un \textit{monoide}:
      \begin{itemize}
      \item Associatività: $(a \cdot b) \cdot c = a \cdot (b \cdot c), \forall a, b, c \in R$
      \item Unità: $\exists 1 \in R \mid a \cdot 1 = 1 \cdot a = a, \forall a \in R$
      \end{itemize}
    \item Proprieta' distributiva (sx e dx)
  \end{itemize} 
}

Come conseguenza diretta, si ha che l'elemento neutro della somma diventa \textbf{elemento assorbente} ($ \forall a \in K. a \cdot 0 = 0 $). Infatti, se un prodotto e' 0 allora e' sicuro che almeno uno degli operandi e' 0.

\nt{
A differenza dei campi, in un anello generale non si richiede che il prodotto sia commutativo ($ a \cdot b $ può essere diverso da $ b \cdot a $), e non si richiede che ogni elemento non nullo abbia un inverso moltiplicativo (cioè non si può sempre "dividere").
}

\ex{Anello $ (\mathbb{Z}_6, +, \cdot) $}{
  Un ottimo esempio per comprendere la differenza tra un anello e un campo \`e l'insieme delle classi di resto modulo 6: 
\[
\mathbb{Z}_6 = \{\overline{0}, \overline{1}, \overline{2}, \overline{3}, \overline{4}, \overline{5}\}
\]

\subsubsection*{I Divisori dello Zero in $\mathbb{Z}_6$}

A differenza del campo $\mathbb{Z}_5$ analizzato in precedenza, in $\mathbb{Z}_6$ \textbf{non} tutti gli elementi non nulli possiedono un inverso moltiplicativo. Questo accade perch\'e 6 \`e un numero composto, il che porta alla comparsa dei cosiddetti \emph{divisori dello zero}.

Un divisore dello zero \`e un elemento non nullo che, moltiplicato per un altro elemento non nullo, restituisce zero. In $\mathbb{Z}_6$, se consideriamo gli elementi $\overline{2}$ e $\overline{3}$, osserviamo che:
\[
\overline{2} \cdot \overline{3} = \overline{6} \equiv \overline{0} \pmod{6}
\]
Sia $\overline{2}$ che $\overline{3}$ sono diversi da $\overline{0}$, ma il loro prodotto \`e $\overline{0}$. A causa di questa propriet\`a, \`e matematicamente impossibile che $\overline{2}$ o $\overline{3}$ ammettano un inverso moltiplicativo (nessun numero moltiplicato per $\overline{2}$ potr\`a mai dare $\overline{1}$).

\noindent
\textbf{Conclusione:} Poich\'e possiede divisori dello zero, la struttura $(\mathbb{Z}_6, +, \cdot)$ \`e un \textbf{anello commutativo}, ma non pu\`o essere un campo. Altri esempi classici di anelli includono l'anello dei polinomi o l'anello delle matrici quadrate $n \times n$ (che \`e un esempio di anello non commutativo).
}

\subsection{Proprieta' fondamentali}
Per manipolare gli anelli, definiamo alcune categorie di elementi e strutture interne fondamentali:

\begin{itemize}
    \item \textbf{Divisori dello zero:} Un elemento $a \neq 0$ si dice divisore dello zero se esiste un elemento $b \neq 0$ tale che $a \cdot b = 0$. Si osservi che nei campi questa eventualità non si verifica mai per definizione.
    \item \textbf{Elementi Invertibili (Unità):} Gli elementi di $R$ che possiedono un inverso moltiplicativo formano un gruppo rispetto all'operazione di prodotto, indicato con il simbolo $R^\times$ (o $U(R)$).
    \item \textbf{Ideali:} Un sottoinsieme $I \subseteq R$ è un \textbf{ideale sinistro} se è un sottogruppo additivo e ``assorbe'' il prodotto da sinistra: ovvero, per ogni $r \in R$ e ogni $x \in I$, si ha che $r \cdot x \in I$. Gli ideali rappresentano per gli anelli ciò che i sottogruppi normali rappresentano per i gruppi, permettendo la costruzione degli \textbf{anelli quoziente} $R/I$.
\end{itemize}

\section{Spazio Vettoriale}
\dfn{Spazio Vettoriale}{
Sia $K$ un campo (i cui elementi sono detti \textit{scalari}). Un insieme $V$ (i cui elementi sono detti \textit{vettori}) è un \textbf{spazio vettoriale su $K$} (o $K$-spazio vettoriale) se è dotato di due operazioni:
\begin{enumerate}
    \item \textbf{Somma interna:} un'operazione $+: V \times V \to V$ che rende $(V, +)$ un gruppo abeliano (commutativa, associativa, con elemento neutro $0_V$ e opposto per ogni vettore).
    \item \textbf{Prodotto per uno scalare:} un'operazione esterna $\cdot: K \times V \to V$ tale che, per ogni scalare $\alpha, \beta \in K$ e per ogni vettore $u, v \in V$, valgano i seguenti quattro assiomi:
    \begin{itemize}
        \item \textbf{Distributività rispetto alla somma vettoriale:} $\alpha \cdot (u + v) = (\alpha \cdot u) + (\alpha \cdot v)$
        \item \textbf{Distributività rispetto alla somma scalare:} $(\alpha + \beta) \cdot v = (\alpha \cdot v) + (\beta \cdot v)$
        \item \textbf{Compatibilità del prodotto:} $(\alpha \beta) \cdot v = \alpha \cdot (\beta \cdot v)$
        \item \textbf{Azione dell'identità scalare:} $1_K \cdot v = v$ (dove $1_K$ è l'elemento neutro moltiplicativo del campo $K$).
    \end{itemize}
\end{enumerate}
}

\ex{Piano Euclideo}{
  L'esempio pi\`u classico e geometricamente intuitivo \`e il piano euclideo, ovvero l'insieme delle coppie ordinate di numeri reali $\mathbb{R}^2$, considerato sul campo dei numeri reali $\mathbb{R}$.
\[
\mathbb{R}^2 = \{ (x, y) \mid x, y \in \mathbb{R} \}
\]

\subsubsection*{Le due operazioni fondamentali in $\mathbb{R}^2$}

Siano $\mathbf{u} = (x_1, y_1)$ e $\mathbf{v} = (x_2, y_2)$ due vettori in $\mathbb{R}^2$, e sia $c \in \mathbb{R}$ uno scalare. Le operazioni che rendono $\mathbb{R}^2$ uno spazio vettoriale sono definite componente per componente:

\begin{enumerate}
    \item \textbf{Addizione vettoriale:} La somma di due vettori produce un nuovo vettore in $\mathbb{R}^2$.
    \[
    \mathbf{u} + \mathbf{v} = (x_1 + x_2, y_1 + y_2)
    \]
    \item \textbf{Moltiplicazione per uno scalare:} Il prodotto di un vettore per un numero reale "scala" (allunga, accorcia o inverte) il vettore, producendo un risultato che resta in $\mathbb{R}^2$.
    \[
    c\mathbf{u} = c(x_1, y_1) = (cx_1, cy_1)
    \]
\end{enumerate}

\noindent
\textbf{Nota sugli Assiomi:} Queste due operazioni in $\mathbb{R}^2$ soddisfano rigorosamente tutti gli otto assiomi richiesti per gli spazi vettoriali (tra cui commutativit\`a e associativit\`a dell'addizione, esistenza del vettore nullo $\mathbf{0} = (0,0)$, esistenza del vettore opposto, e le varie propriet\`a distributive). Altri esempi molto utilizzati in algebra lineare includono lo spazio dei polinomi di grado $\le n$ o lo spazio delle matrici di dimensione $m \times n$.
}

\subsection{L'Importanza del Campo $K$ in Teoria delle Rappresentazioni}
Nel contesto dello studio delle rappresentazioni $\rho: G \to GL(V)$, le proprietà algebriche del campo $K$ determinano la validità dei teoremi fondamentali:

\begin{itemize}
    \item \textbf{Chiusura Algebrica (Esistenza degli autovalori):} Se lavoriamo su $K = \mathbb{C}$ (che è un campo algebricamente chiuso), il Teorema Fondamentale dell'Algebra ci garantisce che ogni endomorfismo lineare abbia sempre almeno un autovalore. Questa proprietà è il motore logico che fa funzionare il \textbf{Lemma di Schur} e ci permette di diagonalizzare l'azione del gruppo.
    \item \textbf{Caratteristica del Campo (Divisione per $|G|$):} Affinché sia valido il \textbf{Teorema di Maschke} (e le rappresentazioni siano completamente riducibili), è necessario che la caratteristica del campo, $\text{char}(K)$, non divida l'ordine del gruppo finito $|G|$. Solo sotto questa condizione l'elemento $|G| \cdot 1_K$ è invertibile in $K$, rendendo possibile l'operazione di ``media sul gruppo'' per costruire i proiettori equivarianti.
    \item \textbf{Dimensione Relativa:} La dimensione di uno spazio vettoriale dipende strettamente da $K$. Ad esempio, l'insieme dei numeri complessi $\mathbb{C}$ ha dimensione $1$ se inteso come $\mathbb{C}$-spazio vettoriale, ma possiede dimensione $2$ se lo strutturiamo come $\mathbb{R}$-spazio vettoriale (con base $\{1, i\}$).
\end{itemize}

\subsection{Prodotto Hermitiano}
\dfn{Coniugato di un Numero Complesso}{
Sia $z = a + ib$ un numero complesso, con $a, b \in \mathbb{R}$. Si definisce \textbf{coniugato} di $z$, e si indica con $\overline{z}$ (o talvolta $z^*$), il numero complesso:
$$ \overline{z} = a - ib $$
In termini geometrici, $\overline{z}$ è il simmetrico di $z$ rispetto all'asse delle ascisse nel piano di Argand-Gauss.
}

\nt{
\textbf{Proprietà Fondamentali:}
\begin{itemize}
    \item \textbf{Prodotto con il coniugato:} Il prodotto di un numero per il suo coniugato è sempre un reale non negativo, pari al quadrato del modulo: $z \cdot \overline{z} = a^2 + b^2 = |z|^2$.
    \item \textbf{Linearità:} Il coniugato della somma è la somma dei coniugati: $\overline{z + w} = \overline{z} + \overline{w}$.
    \item \textbf{Moltiplicatività:} Il coniugato del prodotto è il prodotto dei coniugati: $\overline{z \cdot w} = \overline{z} \cdot \overline{w}$.
    \item \textbf{Involuzione:} Coniugare due volte riporta al numero originale: $\overline{\overline{z}} = z$.
    \item \textbf{Caratterizzazione dei Reali:} Un numero è reale se e solo se coincide con il suo coniugato: $z = \overline{z} \iff z \in \mathbb{R}$.
\end{itemize}

\textbf{Ruolo nei Caratteri:}
Nella teoria delle rappresentazioni, quando calcoliamo il prodotto scalare tra due caratteri $\chi_1$ e $\chi_2$, usiamo il coniugato:
$$ \langle \chi_1, \chi_2 \rangle = \frac{1}{|G|} \sum_{g \in G} \chi_1(g) \overline{\chi_2(g)} $$
Questo garantisce che la "distanza" tra due rappresentazioni sia un valore sensato nel campo complesso.
}
\dfn{Prodotto Hermitiano}{
Sia $V$ uno spazio vettoriale su $\mathbb{C}$. Un \textbf{prodotto hermitiano} è una funzione $\langle \cdot, \cdot \rangle: V \times V \to \mathbb{C}$ che associa a ogni coppia di vettori $u, v$ uno scalare complesso, soddisfacendo le seguenti proprietà:
\begin{enumerate}
    \item \textbf{Linearità nel primo argomento:} $\langle a u + b w, v \rangle = a \langle u, v \rangle + b \langle w, v \rangle$.
    \item \textbf{Simmetria Hermitiana (Antisimmetria):} $\langle u, v \rangle = \overline{\langle v, u \rangle}$ (dove la barra indica il coniugato complesso).
    \item \textbf{Positività definita:} $\langle v, v \rangle \in \mathbb{R}$, $\langle v, v \rangle \ge 0$ e $\langle v, v \rangle = 0$ se e solo se $v = 0$.
\end{enumerate}
}

\nt{
\textbf{Perché è diverso dal prodotto scalare reale?}
\begin{itemize}
    \item \textbf{Il problema del coniugato:} Se usassimo la formula reale $\sum x_i y_i$ con i numeri complessi, potremmo avere un vettore non nullo con "lunghezza" zero (es. $v=(1, i) \to 1^2 + i^2 = 1 - 1 = 0$). Il coniugato nella proprietà (2) serve a garantire che la norma $\|v\|^2 = \langle v, v \rangle$ sia sempre un numero reale positivo.
    \item \textbf{Semi-linearità:} Nota che a causa della simmetria hermitiana, se tiri fuori uno scalare dal \textit{secondo} argomento, questo esce coniugato: $\langle u, \alpha v \rangle = \overline{\alpha} \langle u, v \rangle$.
    \item \textbf{Uso nel Modulo 2:} Lo usiamo per definire le \textbf{rappresentazioni unitarie}. Una matrice è unitaria se preserva questo prodotto. Grazie al ``trucco della media'', abbiamo dimostrato che ogni gruppo finito può essere rappresentato con matrici unitarie.
\end{itemize}
}
\ex{Esempio 1: Il Prodotto Hermitiano Standard in $\mathbb{C}^n$}{
Siano $u = (z_1, z_2, \dots, z_n)$ e $v = (w_1, w_2, \dots, w_n)$ due vettori in $\mathbb{C}^n$. Il prodotto hermitiano standard è definito come:
$$ \langle u, v \rangle = \sum_{i=1}^{n} z_i \overline{w_i} = z_1\overline{w_1} + z_2\overline{w_2} + \dots + z_n\overline{w_n} $$
Se prendiamo ad esempio $u = (1, i)$ e $v = (i, 1+i)$ in $\mathbb{C}^2$:
$$ \langle u, v \rangle = 1 \cdot (\overline{i}) + i \cdot (\overline{1+i}) = 1(-i) + i(1-i) = -i + i - i^2 = -i + i + 1 = 1 $$
Nota come, grazie al coniugato, il risultato finale è un numero che "tiene conto" della fase complessa.
}

\ex{Esempio 2: Prodotto di Frobenius su Matrici $M_n(\mathbb{C})$}{
Lo spazio delle matrici quadrate $n \times n$ può essere visto come uno spazio vettoriale. Il prodotto hermitiano tra due matrici $A$ e $B$ è definito tramite la \textbf{traccia}:
$$ \langle A, B \rangle = \text{Tr}(A B^*) = \text{Tr}(A \overline{B}^T) $$
Questo prodotto è fondamentale per dimostrare le relazioni di ortogonalità tra le entrate delle matrici delle rappresentazioni irriducibili.
}

\ex{Esempio 3: Prodotto tra Caratteri (Il più importante per te)}{
Sia $G$ un gruppo finito e siano $\chi_1, \chi_2: G \to \mathbb{C}$ i caratteri di due rappresentazioni. Lo spazio delle funzioni di classe possiede un prodotto hermitiano naturale definito "mediando" sul gruppo:
$$ \langle \chi_1, \chi_2 \rangle_G = \frac{1}{|G|} \sum_{g \in G} \chi_1(g) \overline{\chi_2(g)} $$
\textbf{Perché è fondamentale?} 
\begin{itemize}
    \item Se $\chi_1$ è irriducibile, allora $\langle \chi_1, \chi_1 \rangle = 1$ (norma unitaria).
    \item Se $\chi_1$ e $\chi_2$ sono irriducibili e non isomorfe, allora $\langle \chi_1, \chi_2 \rangle = 0$ (ortogonalità).
\end{itemize}
}
\section{Morfismi di Strutture Algebriche}
In algebra astratta, un morfismo fra due strutture $A, B$ e' una funzione che trasforma l'insieme di sostegno di $A$ nell'insieme di sostegno di $B$ (o in una sua parte) conservando determinate caratteristiche strutturali, in base alle quali si distinguono diversi morfismi.

\subsection{Omomorfismi}
E' un'applicazione fra strutture dello stesso tipo (Gruppi, Anelli, Campi, ecc...) che conserva le operazioni in esse definite
\dfn{Omomorfismo}{
Siano $A$ e $B$ due strutture algebriche dello stesso tipo. Una funzione $\phi: A \to B$ si dice \textbf{omomorfismo} se:
\[
  \forall f \text{ delle strutture}, \forall x_1,...,x_n. \phi(f_A(x_1,...,x_n) = f_B(\phi(x_1),...,\phi(x_n))
\]
Dove $ f_A, f_B $ rappresentano la funzione $ f $ nelle strutture $ A $ e $ B $ rispettivamente.

Se la struttura ha elementi particolari (unita', zeri, ...), questi vanno considerati come funzioni costanti con zero parametri. Ad esempio, siano $ e_A, e_B $ gli elementi neutri delle singole strutture, allora:
\[
  \phi(e_A) = e_B
\]
}

\subsubsection{Omomorfismi di gruppi}
Nei gruppi, gli omomorfismi sono "compatibili" con la struttura di gruppo. Ovvero, preserva sia gli elementi neutri che inversi:
\[
  \phi(a^{-1}) = [\phi(a)]^{-1}
\]
Inoltre, valgono le seguenti proprieta':
\begin{itemize}
    \item \textbf{Nucleo (Kernel):} $\ker(\phi) = \{x \in G \mid \phi(x) = e_H\}$. Il nucleo misura quanto l'omomorfismo "collassa" il gruppo di partenza ed è sempre un \textit{sottogruppo normale} di $G$ ($\ker(\phi) \trianglelefteq G$).
    \item \textbf{Immagine:} $\text{Im}(\phi) = \{\phi(x) \mid x \in G\}$. L'immagine rappresenta la porzione del codominio effettivamente raggiunta ed è sempre un sottogruppo di $H$ ($\text{Im}(\phi) \le H$).
\end{itemize}

Nel contesto del nostro corso, una \textbf{rappresentazione lineare} di un gruppo finito $G$ su uno spazio vettoriale $V$ (su un campo $K$) non è altro che un omomorfismo di gruppi:
$$ \rho: G \to GL(V) $$
dove $GL(V)$ è il gruppo degli automorfismi lineari (matrici quadrate invertibili) dello spazio $V$. Se $\ker(\rho) = \{e_G\}$, la rappresentazione non "perde" informazioni sul gruppo e si dice \textbf{fedele}.

\subsection{Isomorfismi}
\dfn{Isomorfismo}{
Un omomorfismo $\phi: A \to B$ si dice \textbf{isomorfismo} se la funzione $\phi$ è biunivoca (cioè iniettiva e suriettiva).
}

\subsubsection{Isomorfismi di gruppi}
Se $ A $ e $ B $ sono due gruppi, allora:
\begin{itemize}
    \item \textbf{Criterio di iniettività:} Un omomorfismo $\phi$ è iniettivo se e solo se $\ker(\phi) = \{e_G\}$.
    \item Se esiste un isomorfismo tra $G$ e $H$, i due gruppi si dicono isomorfi e si scrive $G \cong H$. Strutturalmente, sono indistinguibili dal punto di vista algebrico.
\end{itemize}

\subsection{Automorfismi}
\dfn{Automorfismo}{
Un \textbf{automorfismo} è un isomorfismo di una struttura in sé stessa, ovvero una mappa biunivoca $\phi: A \to A$ che preserva le operazioni.
}

\subsubsection{Automorfismi di gruppi}
Nel caso in cui la struttura di $ A $ e' di gruppo:
\begin{itemize}
    \item L'insieme di tutti gli automorfismi di un gruppo $G$, dotato dell'operazione di composizione di funzioni, forma un gruppo a sua volta, denotato con $\text{Aut}(G)$.
    \item \textbf{Automorfismi interni:} Fissato un elemento $g \in G$, la mappa di coniugio $\gamma_g(x) = gxg^{-1}$ è sempre un automorfismo di $G$. L'insieme di questi automorfismi forma un sottogruppo denotato con $\text{Inn}(G) \trianglelefteq \text{Aut}(G)$.
\end{itemize}

\section{L'Omomorfismo Determinante}

E' possibile definire la funzione del \textbf{determinante} come un omomorfismo di gruppi, vediamo come:
\dfn{Determinante}{
Sia $K$ un campo e $K^\times = K \setminus \{0\}$ il suo gruppo moltiplicativo (formato da tutti gli elementi non nulli di $K$ con l'operazione di prodotto).
L'applicazione \textbf{determinante} è una mappa:
$$ \det: GL(n, K) \to K^\times $$
che associa a ogni matrice invertibile $A$ il suo determinante $\det(A)$ (che è uno scalare in $K$). Poiché la matrice è invertibile, $\det(A) \neq 0$, quindi il codominio $K^\times$ è corretto.
}

Questa mappa è un omomorfismo di gruppi grazie al celebre \textbf{Teorema di Binet}, il quale garantisce che il determinante del prodotto di due matrici è uguale al prodotto dei loro determinanti:
$$ \det(A \cdot B) = \det(A) \cdot \det(B) \quad \forall A, B \in GL(n, K) $$
In altre parole, la mappa $\det$ preserva (e trasporta) l'operazione di moltiplicazione dal "complicato" gruppo delle matrici al "semplice" gruppo degli scalari.

\subsection*{Nucleo (Kernel) e Immagine}
Applicando le definizioni strutturali degli omomorfismi a questa mappa specifica, otteniamo due informazioni preziose:

\begin{itemize}
    \item \textbf{Immagine:} La mappa è \textbf{suriettiva}. Per ogni scalare $\lambda \in K^\times$, esiste sempre almeno una matrice in $GL(n, K)$ che ha $\lambda$ come determinante (basta prendere la matrice identità e sostituire il primo $1$ in alto a sinistra con $\lambda$). Quindi, $\text{Im}(\det) = K^\times$.
    
    \item \textbf{Nucleo:} Il nucleo è formato da tutte le matrici mappate nell'elemento neutro del codominio (che per la moltiplicazione in $K^\times$ è $1$).
    $$ \ker(\det) = \{ A \in GL(n, K) \mid \det(A) = 1 \} $$
    Questo insieme definisce il \textbf{Gruppo Speciale Lineare}, denotato con $SL(n, K)$. Poiché è il nucleo di un omomorfismo, $SL(n, K)$ è automaticamente un \textbf{sottogruppo normale} di $GL(n, K)$ ($SL(n, K) \trianglelefteq GL(n, K)$).
\end{itemize}

\subsection*{Conseguenza: Il Primo Teorema di Omomorfismo}
Per il Primo Teorema di Omomorfismo, il quoziente del dominio rispetto al nucleo è isomorfo all'immagine. Questo ci dà una bellissima identità strutturale:
$$ \frac{GL(n, K)}{SL(n, K)} \cong K^\times $$

\subsection*{Applicazione nel Modulo 2 (Caratteri e Rappresentazioni)}
Se possediamo una rappresentazione di un gruppo $G$, ovvero un omomorfismo $\rho: G \to GL(n, \mathbb{C})$, possiamo comporla con l'omomorfismo determinante per creare una nuova mappa:
$$ \det \circ \rho: G \to \mathbb{C}^\times $$
Poiché la composizione di due omomorfismi è ancora un omomorfismo, questa nuova mappa è a tutti gli effetti una \textbf{rappresentazione di grado 1} del gruppo $G$!
% \end{document}
