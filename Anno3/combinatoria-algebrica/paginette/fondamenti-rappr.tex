% \begin{document}
\chapter{Fondamenti della Teoria delle Rappresentazioni}
Questo corso si colloca all'intersezione tra \textbf{Algebra} (gruppi, anelli, campi, spazi vettoriali) e \textbf{Combinatoria} (conteggio). Lo scopo è rendere gli oggetti algebrici \textit{astratti} più \textit{concreti}, studiandoli attraverso le loro azioni lineari su spazi vettoriali.

L'idea di fondo è semplice: invece di studiare un gruppo astratto $G$ direttamente, lo "rappresentiamo" tramite matrici invertibili, che sono oggetti molto più maneggevoli.

\section{Rappresentazioni e Sottorappresentazioni}
Una \textbf{struttura algebrica astratta} — ad esempio un gruppo — è definita da un insieme e da operazioni che soddisfano degli assiomi.
\dfn{Struttura algebrica}{
    Si definisce struttura algebrica una coppia $(G, *)$ dove $G$ è un insieme e $*$ è una operazione binaria su $G$
}
\ex{Strutture algebriche}{
    \begin{itemize}
    \item  \textbf{Gruppo}: $(G, *)$ con $*$ binaria e associativa, $G$ con elemento neutro e ogni elemento ha un inverso
    \item \textbf{Anello}: $(G, +, \cdot)$ con $+$ e $\cdot$ binarie e associative, $G$ con elemento neutro per $+$ e ogni elemento ha un inverso per $+$
    \item \textbf{Campo}: $(G, +, \cdot)$ con $+$ e $\cdot$ binarie e associative, $G$ con elemento neutro per $+$ e ogni elemento ha un inverso per $+$
    \end{itemize}
}
%
Per "astratta", dal latino, "tirata fuori", si intende estrapolata dal suo contesto originale. La combinatoria quindi, vuole studiare le struttura algebriche e darne una rappresentazione più concreta.

Questa struttura, infatti, può sembrare "sospesa nel vuoto", decontestualizzata. L'idea delle rappresentazioni è di portarla \textbf{dentro} uno spazio concreto, dove possiamo agire con strumenti dell'algebra lineare.

Ad esempio, un gruppo può agire su uno spazio vettoriale tramite trasformazioni lineari. Questo lo "rappresenta" come un gruppo di matrici.
\dfn{
  Rappresentazione di un gruppo 
}{
        Sia $G$ un gruppo, $K$ un campo e $V$ uno spazio vettoriale su $K$. si definisce una rappresentazione di $G$ su $V$ è un omomorfismo 
        $$\rho: G \to GL(V)$$ 
}
\nt{
  La rappresentazione e' un omomorfismo, quindi:
  \[ 
   \forall g, h \in G.\ \rho(g) \rho(h) = \rho(gh)
  \]
}
Ricordiamo che $\mathrm{GL}(V)$ è il gruppo degli isomorfismi lineari di $V$ in sé (automorfismi lineari, o equivalentemente matrici invertibili se $V = K^n$).
\nt{
  In altre parole, $\rho$ è un'azione di $G$ su $V$ che non è solo per biiezioni (come in $\mathrm{Sym}(V)$), ma per \textbf{trasformazioni lineari invertibili}.
}

\dfn{Rappresentazione fedele}{
    Diciamo che la rappresentazione è fedele se $\rho$ è iniettiva, ovvero se $\rho(g) = \rho(h) \implies g = h$. 
}
\nt{
  Ovvero il nucleo di $ \rho $ contiene solo l'elemento neutro ($ ker(\rho) = \{e\} $) e $G$ si immerge come sottogruppo di $\mathrm{GL}(V)$.
}

\ex{Gruppo di Klein}{
  Consideriamo il gruppo di Klein $G = \{a, b, c, d\}$ con tavola di Cayley:

$$\begin{array}{c|cccc}
* & a & b & c & d \\
\hline
a & c & d & a & b \\
b & d & c & b & a \\
c & a & b & c & d \\
d & b & a & d & c
\end{array}$$

L'elemento neutro è $c$ (la riga $c$ è l'identità). Si tratta di $\mathbb{Z}/2\mathbb{Z} \times \mathbb{Z}/2\mathbb{Z}$, ogni elemento ha ordine 2.

Una rappresentazione su $V = \mathbb{R}^2$ è data da:

$$\rho(c) = \begin{pmatrix}1 & 0 \\ 0 & 1\end{pmatrix}, \quad \rho(a) = \begin{pmatrix}-1 & 0 \\ 0 & 1\end{pmatrix}, \quad \rho(b) = \begin{pmatrix}1 & 0 \\ 0 & -1\end{pmatrix}, \quad \rho(d) = \begin{pmatrix}-1 & 0 \\ 0 & -1\end{pmatrix}$$

Si può verificare che è un omomorfismo (ad es. $\rho(a)\rho(b) = \rho(d)$, che corrisponde a $a * b = d$ nella tavola).

Rispetto alla decomposizione $V = \langle e_1 \rangle \oplus \langle e_2 \rangle$, le due rette coordinate sono \textbf{stabili} sotto l'azione di $G$: $\rho(g) \cdot e_i$ è sempre nella direzione di $e_i$. Al contrario, la decomposizione $V = \langle(1,1)\rangle \oplus \langle(1,-1)\rangle$ \textbf{non è stabile}, perché le basi non vengono mandate in sé stesse dagli elementi del gruppo.
}


\dfn{Sottorappresentazione}{
  Sia $(V, \rho)$ una rappresentazione di $G$ su $V$, diciamo che un sottospazio vettoriale $U\subseteq V$ è una sottorappresentazione di $\rho$ se $\rho(g)U\subseteq U$ per ogni $g\in G$.
}
\nt{
    In altre parole $(U, \rho_{|U})$ è una rappresentazione di $G$ su $U$, dove $\rho_{|U}: G \to GL(U)$ è l'omomorfismo che mappa $g \to \rho(g)_{|U}$
}

\subsection{Rappresentazioni Riducibili e Decomponibili}
\dfn{Rappresentazione Irriducibile}{
Una rappresentazione $(\rho, V)$ di un gruppo $G$ si dice \textbf{irriducibile} se gli unici sottospazi di $V$ che siano $G$-invarianti (sottorappresentazioni) sono i sottospazi banali $\{0\}$ e $V$ stesso. 
In caso contrario, ovvero se esiste un sottospazio proprio $0 < U < V$ che e' una sottorappresentazione di $ \rho $, la rappresentazione si dice \textbf{riducibile}.
}

\ex{Rappresentazione di $C_4$ e Dipendenza dal Campo $K$}{
Si consideri il gruppo ciclico di ordine 4, $G = C_4 = \langle z \mid z^4 = 1 \rangle$, e la sua rappresentazione naturale su $V = \mathbb{R}^2$ definita mandando il generatore $z$ nella matrice di rotazione di $\pi/2$:
$$ \rho(z) = \begin{pmatrix} 0 & -1 \\ 1 & 0 \end{pmatrix} $$
L'obiettivo è analizzare la riducibilità di $\rho$ al variare del campo degli scalari $K$.

\textbf{1. Analisi sul campo reale ($K = \mathbb{R}$):}
Per determinare se esistono sottorappresentazioni proprie, cerchiamo sottospazi stabili di dimensione 1, il che equivale a cercare gli autovalori reali di $\rho(z)$. Il polinomio caratteristico è:
$$ p(\lambda) = \det(\rho(z) - \lambda I) = \det \begin{pmatrix} -\lambda & -1 \\ 1 & -\lambda \end{pmatrix} = \lambda^2 + 1 $$
Le radici di $p(\lambda)$ sono $\pm i$, le quali \textbf{non appartengono} a $\mathbb{R}$. Non esistendo autovalori reali, non esistono rette in $\mathbb{R}^2$ stabili sotto l'azione della rotazione.
\begin{center}
    \textit{Conclusione:} $\rho$ è \textbf{irriducibile} su $\mathbb{R}$.
\end{center}

\textbf{2. Analisi sul campo complesso ($K = \mathbb{C}$):}
Se estendiamo lo spazio vettoriale a $V_{\mathbb{C}} = \mathbb{C}^2$, gli autovalori $\lambda_1 = i$ e $\lambda_2 = -i$ sono ora ammissibili. Ad essi corrispondono i rispettivi autospazi (sottospazi di dimensione 1):
$$ V_i = \text{span}_{\mathbb{C}} \left\{ \begin{pmatrix} 1 \\ -i \end{pmatrix} \right\}, \quad V_{-i} = \text{span}_{\mathbb{C}} \left\{ \begin{pmatrix} 1 \\ i \end{pmatrix} \right\} $$
Questi sottospazi sono $G$-invarianti, poiché l'azione di $z$ (e delle sue potenze) si limita a moltiplicare i vettori per uno scalare ($\pm i$). Lo spazio si decompone nella somma diretta:
$$ V_{\mathbb{C}} = V_i \oplus V_{-i} $$
\begin{center}
    \textit{Conclusione:} $\rho$ è \textbf{riducibile} su $\mathbb{C}$.
\end{center}
}

\nt{
\textbf{Osservazioni sulla Riducibilità:}
\begin{itemize}
    \item Una rappresentazione di dimensione 1 è sempre irriducibile per motivi dimensionali (non esistono sottospazi propri non nulli).
    \item Il concetto di irriducibilità dipende dal campo $K$ (come visto nell'esempio delle rotazioni su $\mathbb{R}$ vs $\mathbb{C}$).
    \item Scomporre una rappresentazione riducibile in somma diretta di irriducibili è l'obiettivo principale del corso (completabilità garantita dal Teorema di Maschke).
\end{itemize}
}

\ex{Rappresentazione Naturale di $S_3$}{
Sia $G = S_3$ agente su $V = \mathbb{R}^3$ tramite permutazione delle coordinate:
$$ \sigma \cdot (x_1, x_2, x_3) = (x_{\sigma(1)}, x_{\sigma(2)}, x_{\sigma(3)}) $$
Questa rappresentazione è \textbf{riducibile} in quanto ammette i seguenti sottospazi $G$-invarianti propri:
\begin{enumerate}
    \item $U = \{ (t, t, t) \mid t \in \mathbb{R} \}$, sottospazio di dimensione 1 (retta diagonale). Poiché ogni permutazione scambia coordinate identiche, $U$ è puntualmente fisso.
    \item $W = \{ (x_1, x_2, x_3) \in \mathbb{R}^3 \mid x_1 + x_2 + x_3 = 0 \}$, sottospazio di dimensione 2 (piano iperortogonale a $U$). Poiché la somma è commutativa, permutare gli addendi non cambia il risultato zero.
\end{enumerate}
Si verifica facilmente che $U \cap W = \{0\}$, pertanto lo spazio si decompone nella somma diretta:
$$ V = U \oplus W $$
Dove $U$ è la rappresentazione banale e $W$ è la \textbf{rappresentazione standard} di $S_3$. Entrambe sono irriducibili.
}

\dfn{Rappresentazione Indecomponibile}{
Una rappresentazione $(\rho, V)$ si dice \textbf{indecomponibile} se non può essere scritta come somma diretta di due sottorappresentazioni proprie non banali. Ovvero, se $V = U \oplus W$ con $U, W$ sottorappresentazioni, allora necessariamente $U=0$ oppure $W=0$.
}

\nt{
\textbf{Irriducibile vs Indecomponibile:}
\begin{itemize}
    \item Ogni rappresentazione \textbf{irriducibile} è banalmente \textbf{indecomponibile} (se non ha sottospazi stabili, a maggior ragione non può essere somma diretta di sottospazi stabili).
    \item Il viceversa non è sempre vero: esistono rappresentazioni che possiedono sottospazi stabili (sono riducibili) ma che non ammettono un complemento stabile (sono indecomponibili).
    \item \textbf{Nel Modulo 2:} Grazie al Teorema di Maschke, lavorando su un campo di caratteristica 0 (come $\mathbb{C}$) e con gruppi finiti, ogni sottorappresentazione ammette un complemento stabile. Di conseguenza, in questo contesto i concetti di irriducibile e indecomponibile \textbf{coincidono}.
\end{itemize}
}

\ex{Blocco di Jordan}{
  Sia $G = (\mathbb{Z}, +)$, $V = \mathbb{C}^2$, e definiamo

$$\rho : \mathbb{Z} \longrightarrow \mathrm{GL}(\mathbb{C}^2), \qquad n \longmapsto \begin{pmatrix}1 & n \\ 0 & 1\end{pmatrix}.$$

Questo è un omomorfismo perché $\begin{pmatrix}1&n\\0&1\end{pmatrix}\begin{pmatrix}1&m\\0&1\end{pmatrix} = \begin{pmatrix}1&n+m\\0&1\end{pmatrix}$.

  Il sottospazio $U = \langle e_1 \rangle = \langle(1,0)\rangle$ è una sottorappresentazione propria (poiché $\rho(n) e_1 = e_1 \in U$ per ogni $n$), quindi $V$ \textit{non è irriducibile}.

Tuttavia, notiamo che per ogni $(x, y)$ con $y \neq 0$:

$$\rho(n)\begin{pmatrix}x\\y\end{pmatrix} = \begin{pmatrix}x + ny \\ y\end{pmatrix}$$

  che è linearmente indipendente da $(x,y)$ per $n$ generico. Quindi \textbf{non esiste} alcun $W \subseteq V$ con $V = U \oplus W$ stabile, e la rappresentazione è \textbf{indecomponibile}.

Questo è il tipico blocco di Jordan di dimensione 2. La situazione si verifica perché $G = \mathbb{Z}$ è un gruppo infinito. Vedremo che per gruppi \textbf{finiti} con $\mathrm{char}(K) = 0$ la situazione è radicalmente diversa.
}

\subsection{Teorema di Maschke}
Per procedere serve costruire un prodotto scalare $ G $-invariante:
\\
\textbf{Costruzione}: Sia $(\rho, V)$ una rappresentazione di un gruppo finito $G$ su uno spazio vettoriale complesso $V$ ($ K = \mathbb{C} $). Dato un prodotto hermitiano arbitrario $( \cdot, \cdot )$ su $V$, definiamo:
$$ \langle v, u \rangle_G := \frac{1}{|G|} \sum_{g \in G} (\rho(g)v, \rho(g)u) $$
per ogni $v, u \in V$.
\\
Si tratta della media sui trasformati di $(v,u)$ lungo tutto il gruppo. Il fatto che $G$ sia finito e $\mathrm{char}(K) = 0$ (quindi $|G| \neq 0$ in $K$) garantisce che la divisione abbia senso.

\mprop{Esistenza di un Prodotto Hermitiano $G$-invariante}{
Sia $(\rho, V)$ una rappresentazione di un gruppo finito $G$ su uno spazio vettoriale complesso $V$. Esiste sempre su $V$ un prodotto hermitiano $\langle \cdot, \cdot \rangle_G$ che sia $G$-invariante, ovvero tale che:
$$ \langle \rho(g)v, \rho(g)u \rangle_G = \langle v, u \rangle_G \quad \forall g \in G, \forall v, u \in V $$
}

\pf{Dimostrazione}{
Sia $H(v, u)$ un prodotto hermitiano arbitrario su $V$ (la cui esistenza è garantita dalla struttura di spazio vettoriale complesso). Consideriamo il prodotto scalare definito prima:
$$ \langle v, u \rangle_G := \frac{1}{|G|} \sum_{x \in G} (\rho(x)v, \rho(x)u) $$
Per verificare l'invarianza, applichiamo un elemento generico $h \in G$:
$$ \langle \rho(h)v, \rho(h)u \rangle_G = \frac{1}{|G|} \sum_{x \in G} (\rho(x)\rho(h)v, \rho(x)\rho(h)u) $$
Poiché $\rho$ è un omomorfismo, $\rho(x)\rho(h) = \rho(xh)$. Sostituendo:
$$ \langle \rho(h)v, \rho(h)u \rangle_G = \frac{1}{|G|} \sum_{x \in G} (\rho(xh)v, \rho(xh)u) $$
Al variare di $x$ in $G$, l'elemento $xh$ percorre tutti gli elementi del gruppo esattamente una volta (la traslazione a destra è una permutazione di $G$). Possiamo quindi effettuare il cambio di variabile $k = xh$, ottenendo:
$$ \langle \rho(h)v, \rho(h)u \rangle_G = \frac{1}{|G|} \sum_{k \in G} (\rho(k)v, \rho(k)u) = \langle v, u \rangle_G $$
Questo dimostra che il prodotto è $G$-invariante. Le proprietà di linearità, simmetria hermitiana e positività di $\langle \cdot, \cdot \rangle_G$ discendono direttamente dalle proprietà del prodotto hermitiano costruito prima.
}

\thm{Teorema di Maschke (Versione Sintetica)}{
Sia $G$ un gruppo finito e $(\rho, V)$ una rappresentazione di $G$ su un campo $K$ (con $\text{char}(K) \nmid |G|$). Allora ogni sottorappresentazione $U \subseteq V$ ammette un complemento $G$-invariante $W$, tale che 
$$V = U \oplus W$$
}

\cor{}{
Per $G$ finito e $\mathrm{char}(K) = 0$: indecomponibile $\Longleftrightarrow$ irriducibile.
}

\pf{Dimostrazione}{
Data una sottorappresentazione $U \subseteq V$, prendiamo il complemento ortogonale rispetto al prodotto $G$-invariante costruito prima:

$$U^\perp = \{v \in V \mid \langle v, u \rangle = 0 \;\forall\, u \in U\}.$$

Allora $V = U \oplus U^\perp$ come spazi vettoriali (proprietà standard dei prodotti hermitiani). È sufficiente mostrare che $U^\perp$ è $G$-stabile. Siano $v \in U^\perp$ e $u \in U$:

$$\langle \rho(g)v,\, u \rangle = \langle \rho(g)^{-1}\rho(g)v,\, \rho(g)^{-1}u \rangle = \langle v,\, \rho(g)^{-1}u \rangle = 0$$

dove l'ultimo passaggio usa che $\rho(g)^{-1}u \in U$ (perché $U$ è $G$-stabile) e $v \in U^\perp$.
}

\section{Algebra di Gruppo}
Vogliamo ora riformulare tutta la teoria in un linguaggio più algebrico: guardare gli elementi di $G$ come "scalari che agiscono su $V$".

\dfn{A-modulo sinistro}{
 Sia $A$ un anello con unità $1_A$ e $(V, +)$ un gruppo abeliano. $V$ è un $A$-modulo sinistro se esiste un'operazione $A \times V \to V$ tale che:
 \begin{itemize}
 \item $a(v + u) = av + au$ 
 \item $(a + b)v = av + bv$
 \item $(ab)v = a(bv)$
 \item $1_A \cdot v = v$
 \end{itemize}
}
Questa è esattamente la definizione di spazio vettoriale, ma con $A$ anello al posto di campo.

\dfn{Algebra di gruppo}{
  L'\textbf{algebra di gruppo} $K[G]$ è l'insieme delle combinazioni lineari formali degli elementi di $G$ a coefficienti in $K$:

$$K[G] = \left\{\sum_{g \in G} a_g \cdot g \;\bigg|\; a_g \in K\right\}.$$

Gli elementi di $G$ formano una $K$-base di $K[G]$. Il prodotto in $K[G]$ è definito estendendo per bilinearità il prodotto in $G$:

$$(a_{g_1} \cdot g_1)(a_{g_2} \cdot g_2) := (a_{g_1} \cdot a_{g_2}) \cdot (g_1 g_2)$$

dove il primo prodotto è in $K$ e il secondo è in $G$. Questo rende $K[G]$ un \textbf{anello} (non commutativo in generale).
}

\mprop{$ K[G] $-modulo sinistro come rappresentazione di G}{
Le seguenti condizioni sono equivalenti:
\begin{enumerate}
\item $V$ è una rappresentazione di $G$ (cioè esiste $\rho: G \to \mathrm{GL}(V)$).
\item $V$ è un $K[G]$-modulo sinistro.
\end{enumerate}
}

\pf{Dimostrazione}{
$(2) \Rightarrow (1)$: 
Sia $ V $ un $ K[G] $-modulo sinistro. Abbiamo quindi l'operazione:
\[
  \cdot: K[G] \times V \to V
\]
che rispetta certe proprieta'. Dato che $\forall g \in G. g \in K[G]$, possiamo definire per ogni elemento di $ G $ una funzione
\begin{align*}
  T_g: V &\to V\\
  v &\to g \cdot v 
\end{align*}
che quindi utilizza l'operazione definita sopra. Possiamo dimostrare che $ \forall g. T_g $ e' un'\textit{applicazione lineare invertibile}:
\begin{itemize}
\item $ K $-Linearita': usando le proprieta' dell'operazione del modulo possiamo dimostrare che
  \begin{itemize}
  \item $ \forall g \in G, \forall v,u \in V $ 
    \begin{align*}
      T_g(v+u) &= g \cdot (v+u)\\
      &= g\cdot v + g \cdot u \\
      &= T_g(v) + T_g(u)
    \end{align*}
  \item $ \forall g \in G, \forall v \in V, \forall k \in K $
    \begin{align*}
      T_g(kv) &= g \cdot ((ke) \cdot v) \\
      &= (g(ke)) \cdot v \\
      &= (kg) \cdot v \\
      &= k(g \cdot v) \\
      &= kT_g(v)
    \end{align*}
      dove $ e $ e' l'unita' di $ K[G] $
  \end{itemize}
\item Invertibilita': dimostriamo che $ \forall g \in G. T_g^{-1} $ e' l'applicazione lineare inversa di $ T_g $:
  \begin{align*}
    T_{g^{-1}}(T_g(u)) &= g^{-1} \cdot (g \cdot u) \\
    &= (g^{-1}g) \cdot u \\
    &= e \cdot u = u\\
  \end{align*}
\end{itemize}

$(1) \Rightarrow (2)$: Sia $ V $ una rappresentazione di $ G $, allora $ \exists \rho: G \to GL(V) $ omomorfismo. Vogliamo mostrare che $ V $ ha anche struttura di $ K[G] $-modulo sx, ovvero che $ \exists \cdot: K[G] \times V \to V $ con le proprieta' giuste. Possiamo definire questo prodotto come:
\[
  \left(\sum_{g \in G}a_g g\right) \cdot v \coloneq \sum_{g \in G} a_g \rho(g)(v)
\]
Si puo' dimostrare che l'operazione cosi' definita soddisfa tutti gli assiomi di modulo (lo puo' dimostrare la Marta Matrici se ne ha voglia), quindi $ V $ e' un $ K[G] $-modulo.
}

\nt{
  Una notazione un po' crazy che ha usato il prof è anche questa:
  \[
    K[G] = \left\{\sum_{g \in G} a_g \cdot e_g \;\bigg|\; a_g \in K\right\} 
  \]
  Dove $\{e_g\}_{g \in G}$ è la base formale indicizzata dagli elementi di $G$, ovvero $e_g = (0,\ldots,1,\ldots,0)$ con $1$ in posizione $g$. In questo caso $K[G]$ è un uno spazio vettoriale su $K$ con dimensione $|G|$, ed è anello esresso per bilinearità della moltiplicazione in $G$:
  \[
    e_g \cdot e_h =  e_{gh} \quad \forall g,h \in G
  \]
}

\ex{}{
  Sia $G= S_3$. Sia $v = 5e_{(12)}$ e $u=2e_{id}-3e_{(23)}$ in $K[S_3]$. Si ha:
  \[
    v \cdot u = (5e_{(12)}) \cdot (2e_{id}-3e_{(23)}) = 10e_{(12)\cdot id}-15e_{(12)\cdot (23)}= 12e_{(12)}-15e_{(132)}
  \]
}

\subsection{Rappresentazione Regolare}

L'algebra di gruppo $K[G]$ non è solo l'ambiente in cui vivono gli "scalari" della teoria, ma può essere essa stessa vista come uno spazio vettoriale su cui il gruppo agisce.

\dfn{Azione regolare a sinistra}{
    Si definisce \textbf{azione regolare a sinistra} di $G$ su $K[G]$ l'applicazione che associa a ogni coppia $(h, e_g)$ l'elemento della base corrispondente al prodotto nel gruppo:
    $$ \cdot : G \times K[G] \longrightarrow K[G] $$
    $$ h \cdot e_g \coloneqq e_{hg} \quad \forall h, g \in G $$
    Per estensione lineare, l'azione di un elemento $h \in G$ su un generico elemento $x = \sum_{g \in G} a_g e_g \in K[G]$ è definita come:
    $$ h \cdot \left( \sum_{g \in G} a_g e_g \right) \coloneqq \sum_{g \in G} a_g e_{hg} $$
}

\thm{Esistenza della rappresentazione regolare}{
    L'azione regolare a sinistra definisce una rappresentazione lineare di $G$ sullo spazio vettoriale $V = K[G]$. Esiste cioè un omomorfismo di gruppi:
    $$ \rho_{reg} : G \longrightarrow \mathrm{GL}(K[G]) $$
    che associa a ogni $g \in G$ un operatore lineare invertibile $\rho_{reg}(g)$. Tale rappresentazione è detta \textbf{rappresentazione regolare} di $G$.
}

\nt{
  A quanto pare si è deciso in via del tutto arbitraria che
  \[
    \rho_{reg}(g)h = e_{gh}
  \]
  Secondo me intendeva:
  \[
    \rho_{reg}(g)e_h = e_{gh}
  \]
  che ha piu' senso, in quanto la rappresentazione $ \rho_{reg} $ e' l'applicazione parziale dell'azione regolare a sinistra definita sopra:
  \[
    \rho_{reg}(g) \coloneq g \cdot \alpha
  \]
  dove $ \alpha $ e' un elemento fissato di $ K[G] $ (in teoria penso che esista un $ \rho_{reg} $ per ogni elemento $ \alpha \in K[G] $).
  \\

  Alla fine usiamo in modo interscambiabile $ h $ e $ e_h $ per rappresentare una base di $ K[G] $ quindi ha senso.
}

\pf{Dimostrazione}{
    Dobbiamo verificare che $\rho_{reg}$ soddisfi le condizioni di omomorfismo verso il gruppo degli operatori lineari invertibili:

    \begin{enumerate}
        \item \textbf{Linearità ed Invertibilità}: Per ogni $g \in G$, l'operatore $\rho_{reg}(g)$ permuta gli elementi della base $\{e_h\}_{h \in G}$ (poiché la moltiplicazione per $g$ a sinistra è una biiezione del gruppo in sé). Un operatore che permuta una base si estende univocamente a un isomorfismo lineare, dunque $\rho_{reg}(g) \in \mathrm{GL}(K[G])$.
        
        \item \textbf{Conservazione dell'identità}: Sia $1_G$ l'elemento neutro di $G$. Per ogni generatore $e_g$:
        $$ \rho_{reg}(1_G) e_g = e_{1_G \cdot g} = e_g $$
        Poiché l'operatore coincide con l'identità su tutti i vettori della base, si ha $\rho_{reg}(1_G) = \mathrm{id}_{K[G]}$.

        \item \textbf{Proprietà di omomorfismo}: Siano $h_1, h_2 \in G$. Verifichiamo che $\rho_{reg}(h_1 h_2) = \rho_{reg}(h_1) \circ \rho_{reg}(h_2)$ applicandoli a un generico $e_g$:
        \begin{itemize}
            \item Applicazione composta: $\rho_{reg}(h_1) \left( \rho_{reg}(h_2) e_g \right) = \rho_{reg}(h_1) e_{h_2 g} = e_{h_1(h_2 g)}$
            \item Applicazione diretta: $\rho_{reg}(h_1 h_2) e_g = e_{(h_1 h_2)g}$
        \end{itemize}
        Per l'associatività della legge di gruppo in $G$, si ha $h_1(h_2 g) = (h_1 h_2)g$, da cui l'uguaglianza dei vettori immagine. Per linearità, l'uguaglianza si estende a tutto lo spazio.
    \end{enumerate}
}

\nt{
    La rappresentazione regolare ha dimensione pari all'ordine del gruppo, $\dim(K[G]) = |G|$. In termini matriciali, ogni $\rho_{reg}(g)$ è rappresentato da una matrice di permutazione di taglia $|G| \times |G|$. Come vedremo, questa rappresentazione è fondamentale perché contiene ogni rappresentazione irriducibile di $G$ con molteplicità pari alla sua dimensione.
}

\ex{}{
  Sia $ G $ il gruppo ciclico di ordine 4 $ \langle z | z^4 = 1 \rangle $:
  \begin{enumerate}
  \item Usando il Teorema Fondamentale degli Omomorfismi di Anelli, dimostrare che
    \[
      K[G] \cong K[X]/(x^4 - 1)
    \]
  \item Usando il Teorema Cinese del Resto, decomporre per $ K = \mathbb{R} $ e $ K = \mathbb{C} $
  \end{enumerate}

  Sia $G = \langle z \mid z^4 = 1 \rangle = \{1, z, z^2, z^3\}$ il gruppo ciclico di ordine 4.\\

\textbf{1. Isomorfismo $K[G] \cong K[X]/(x^4 - 1)$}

Per dimostrare l'isomorfismo richiesto, utilizziamo il \textbf{Teorema Fondamentale degli Omomorfismi di Anelli}.

Definiamo l'applicazione $\phi: K[X] \to K[G]$ come l'unico omomorfismo di $K$-algebre che estende la mappa $X \mapsto z$. In termini espliciti, per ogni polinomio $P(X) = \sum_{i=0}^n a_i X^i \in K[X]$:
\[ \phi(P(X)) = \sum_{i=0}^n a_i z^i \]
Verifichiamo le proprietà necessarie:
\begin{itemize}
    \item \textbf{Suriettività}: Per definizione di anello di gruppo, ogni elemento di $K[G]$ è una combinazione lineare finita di elementi di $G$. Poiché $G = \{1, z, z^2, z^3\}$, un generico elemento di $K[G]$ è della forma $a_0 + a_1 z + a_2 z^2 + a_3 z^3$, che è esattamente l'immagine del polinomio $P(X) = a_0 + a_1 X + a_2 X^2 + a_3 X^3$. Quindi $\text{im}(\phi) = K[G]$.
    \item \textbf{Nucleo}: Un polinomio $P(X)$ appartiene a $\ker(\phi)$ se e solo se $P(z) = 0$. Sappiamo che $z^4 = 1$ nel gruppo $G$, quindi $z^4 - 1 = 0$ in $K[G]$. Ciò implica che il polinomio $x^4 - 1$ appartiene al nucleo, ovvero $(x^4 - 1) \subseteq \ker(\phi)$. 
    Poiché gli elementi $\{1, z, z^2, z^3\}$ sono linearmente indipendenti su $K$ (formano una base di $K[G]$), nessun polinomio di grado inferiore a 4 può annullarsi in $z$ (se non il polinomio nullo). Pertanto, il nucleo è esattamente l'ideale generato dal polinomio minimo di $z$, ovvero $\ker(\phi) = (x^4 - 1)$.
\end{itemize}
Per il Teorema Fondamentale degli Omomorfismi:
\[ K[X] / \ker(\phi) \cong \text{im}(\phi) \implies K[X] / (x^4 - 1) \cong K[G] \]

\textbf{2. Decomposizione tramite il Teorema Cinese del Resto}

Sfruttando l'isomorfismo precedente, studiamo la decomposizione di $K[X]/(x^4 - 1)$ per i casi $K = \mathbb{R}$ e $K = \mathbb{C}$. Il polinomio $x^4 - 1$ si scompone come $(x^2 - 1)(x^2 + 1) = (x - 1)(x + 1)(x^2 + 1)$.\\

\textbf{Caso $K = \mathbb{C}$}\\
In $\mathbb{C}[X]$, il polinomio $x^2 + 1$ è ulteriormente scomponibile in $(x - i)(x + i)$. Quindi:
\[ x^4 - 1 = (x - 1)(x + 1)(x - i)(x + i) \]
I fattori sono polinomi di primo grado distinti, quindi sono a due a due coprimi. Per il Teorema Cinese del Resto:
\begin{align*}
\mathbb{C}[G] \cong \frac{\mathbb{C}[X]}{(x - 1) \cdot (x + 1) \cdot (x - i) \cdot (x + i)} &\cong \frac{\mathbb{C}[X]}{(x - 1)} \times \frac{\mathbb{C}[X]}{(x + 1)} \times \frac{\mathbb{C}[X]}{(x - i)} \times \frac{\mathbb{C}[X]}{(x + i)} \\
&\cong \mathbb{C} \times \mathbb{C} \times \mathbb{C} \times \mathbb{C} \cong \mathbb{C}^4
\end{align*}\\

\textbf{Caso $K = \mathbb{R}$}\\
In $\mathbb{R}[X]$, il polinomio $x^2 + 1$ è irriducibile. La scomposizione in fattori coprimi è:
\[ x^4 - 1 = (x - 1)(x + 1)(x^2 + 1) \]
Applicando il Teorema Cinese del Resto:
\begin{align*}
\mathbb{R}[G] \cong \frac{\mathbb{R}[X]}{(x - 1) \cdot (x + 1) \cdot (x^2 + 1)} &\cong \frac{\mathbb{R}[X]}{(x - 1)} \times \frac{\mathbb{R}[X]}{(x + 1)} \times \frac{\mathbb{R}[X]}{(x^2 + 1)} \\
&\cong \mathbb{R} \times \mathbb{R} \times \mathbb{C}
\end{align*}
Dove abbiamo usato il fatto che $\mathbb{R}[X]/(x^2 + 1) \cong \mathbb{C}$.
}

\section{Secondo approccio al Lemma del Complemento}

\subsection{Morfismi di Rappresentazioni}

Si guardi tale definizione

\dfn{Morifsmo di rappresentazioni}{
  Siano $(V_1,\rho_1)$ e $(V_2,\rho_2)$ due rappresentazioni di $G$ su $K$. Un \textit{morifsmo di rappresentazioni} (o \textit{morfismo di $K[G]$-moduli}) è una applicazione $f: V_1 \to V_2$ tale che:
  \[
    f(\rho_1(g)v) = \rho_2(g)f(v) \quad \forall g \in G, \forall v \in V_1
  \]
}

\nt{
  Questa condizione equivale a dire
  \[
    f \circ \rho_1(g) = \rho_2(g) \circ f
  \]
  che possiamo riscrivere come
  \[
    f = \rho_2(g) \circ f \circ \rho_1(g)^{-1}
  \]
  componendo a dx $ \rho_1(g)^{-1} $. Quindi, nel caso in cui $ V_1 = V_2 = V $ (come nel Lemma di Maschke), $ f $ e' un morfismo di rappresentazioni sse e' $ G $-invariante.
}

\nt{
  Si noti il seguente diagrmmino di Luca Doci:
  \[
  \begin{array}{ccc}
    V_1 & \xrightarrow{f} & V_2 \\
    \rho_1(g)\downarrow & & \downarrow \rho_2(g) \\
    V_1 & \xrightarrow{f} & V_2 
  \end{array}
\]
il seguente il seguente diagramma commuta per ogni $g \in G$
}

Un morfismo biettivo di rappresentazioni si chiama isomorfismo di rappresentazioni

\mprop{Esercizio 3 -Nucleo e Immagine di un Morfismo sono Sottorappresentazioni}{
  Sia $f: V_1 \to V_2$ un morfismo di rappresentazioni di $G$. Allora:
  \begin{enumerate}
    \item $\ker(f)$ è una sottorappresentazione di $V_1$
    \item $\mathrm{im}(f)$ è una sottorappresentazione di $V_2$
  \end{enumerate}
}
\pf{Dimostrazione lasciata al giolapalma}{
  \begin{enumerate}
    \item Occorre mostrare che $\ker(f)$ è $G-stabile$, ovvero che per ogni $g \in G$ e $v \in \ker(f)$, si ha $\rho_1(g)v \in \ker(f)$.
    
    Sia quindi $v\in \ker(f)$, ovvero $f(v)=0$. Allora per ogni $g \in G$ si ha:
    \[
      f(\rho_1(g)v) = \rho_2(g)f(v) = \rho_2(g)0 = 0
    \]
    Quindi $ \rho_1(g)v \in \ker(f) $ 
    
    \item Occorre mostrare che $ \mathrm{im}(f) $ e' $ G-stabile $, ovvero che per ogni $ g \in G $ e $ v \in \mathrm{im}(f) $, si ha $ \rho_2(g)v \in \mathrm{im}(f) $.
    
    Sia quindi $ v \in \mathrm{im}(f) $, ovvero $ v = f(u) $ per qualche $ u \in V_1 $. Allora per ogni $ g \in G $ si ha:
    \[
      \rho_2(g)v = \rho_2(g)f(u) = f(\rho_1(g)u) \in \mathrm{im}(f)
    \]
  \end{enumerate}
  Luca moci skills
}
\subsection{Lemma del Complemento con i proiettori}
\dfn{Proiettore su un sottospazio}{
  Sia $V$ uno spazio vettoriale e $U \subseteq V$ un sottospazio. Un operatore $p \in \mathrm{End}(V)$ si dice \textit{proiettore su $U$} se:
  \begin{enumerate}
    \item $p \circ p = p$ (idempotenza)
    \item $\mathrm{im}(p) = U$ (l'immagine coincide con il sottospazio)
    \item $p|_U = id_U$, ovvero $p(u) = u$ per ogni $u \in U$
  \end{enumerate}
}

Adesso andremo a definire un proiettore $G$-invariante con il trucco della media vista in precedenza
TODO: ma vogliamo un proiettore G-invariante o un morfismo di rappresentazioni? O sono la stessa cosa?
\dfn{Proiettore $G$-invariante $P^G$}{
  Sia $V$ una rappresentazione di $G$ e $U \subseteq V$ una sottorappresentazione. Si dice \textit{proeittore $G$-invariante} la seguente funzione:
  \[
    p^G:V \to U \quad p^G(v) = \frac{1}{|G|} \sum_{g \in G} \rho(g)\circ p \circ \rho(g)^{-1}(v)
  \]
  dove $p$ e' un proiettore su $U$.
}

\mprop{Verifica del proiettore $G$-invariante}{
  Il proiettore $p^G$ appena definito è ben definito
}
\pf{Dimostrazione}{
  Dobbiamo verificare che $ p^G $ e' effettivamente un proiettore e che sia davvero $ G $-invariante:
  \begin{itemize}
    \item Verifica che $p^G$ è ancora un proiettore su $U$: dobbiamo verificare le tre condizioni per essere un proiettore:
      \begin{enumerate}
      \item Immagine: Poiché $U$ è $G$-stabile e $\mathrm{Im}(p) = U$, ogni termine $\rho(g) \circ p \circ \rho(g)^{-1}$ manda $V$ in $U$. Quindi $\mathrm{Im}(p^G) \subseteq U$. 
      \item Restrizione su $ U $: Se $u \in U$ si ha che $\forall g \in G.\ \rho(g)^{-1}u \in U$ ($U$ è stabile), quindi $p(\rho(g)^{-1}u) = \rho(g)^{-1}u$, e quindi $\rho(g)\,p\,\rho(g)^{-1} u = \rho(g)\rho(g)^{-1}u = u$. Facendo la media: $p^G(u) = u$.
      \end{enumerate}
    \item Verifica che $p^G$ è $G$-invariante: Dobbiamo mostrare che per ogni $h \in G$:

$$\rho(h) \circ p^G = p^G \circ \rho(h)$$

ovvero equivalentemente $\rho(h) \circ p^G \circ \rho(h)^{-1} = p^G$. Calcoliamo:

$$\rho(h) \circ p^G \circ \rho(h)^{-1} = \frac{1}{|G|} \sum_{g \in G} \rho(h) \circ \rho(g) \circ p \circ \rho(g)^{-1} \circ \rho(h)^{-1} = \frac{1}{|G|} \sum_{g \in G} \rho(hg) \circ p \circ \rho(hg)^{-1}.$$

      dove possiamo portare all'interno della sommatoria $ \rho(h) $ grazie alla linearita' della composizione (distributiva), e per la proprieta' degli omomorfismi $ \rho(hg)^{-1}=\rho(g^{-1}h^{-1})=\rho(g)^{-1} \circ \rho(h)^{-1} $ (gli operatori si scambiano).\\

Poiché $g \mapsto hg$ è una biiezione di $G$, la somma sull'indice $hg$ è uguale alla somma sull'indice $g$, e si ottiene esattamente $p^G$.
  \end{itemize}
}

Adesso si viene al lemma vero e proprio

\mlenma{Secondo approccio al Lemma di Maschke}{
Sia $(V, \rho)$ una rappresentazione di un gruppo finito $G$ su un campo $K$ tale che $\mathrm{char}(K) \nmid |G|$. Sia $U \subseteq V$ una sottorappresentazione e sia $p^G$ l'operatore di proiezione $G$-invariante definito prima. Allora:
\begin{enumerate}
    \item $\ker(p^G)$ è una sottorappresentazione di $V$;
    \item $V = U \oplus \ker(p^G)$.
\end{enumerate}
}
\pf{Dimostrazione}{
  Consideriamo la proiezione $ G $-invariante $ p^G: V \to U $.
  \begin{enumerate}
  \item Poiché $p^G$ è un morfismo di rappresentazioni (per i calcoli definiti nella nota), per la proposizione di prima $\ker(p^G)$ è una sottorappresentazione di $V$.
  \item Poiché $p^G$ è un proiettore con $\mathrm{Im}(p^G) = U$, grazie alle proprieta' dei proiettori si ha la decomposizione:
    $$V = U \oplus \ker(p^G)$$
    e $W := \ker(p^G)$ è il complemento $G$-stabile cercato.
  \end{enumerate}
}

Detto cio' possiamo tirare fuori un corollario molto utile
\cor{}{
  Sia $G$ un gruppo finito, $K$ un campo con $\mathrm{char}(K) = 0$ (o più in generale $\mathrm{char}(K) \nmid |G|$). Allora ogni rappresentazione $V$ di $G$ di dimensione finita è somma diretta di rappresentazioni irriducibili:

$$V \cong V_1 \oplus V_2 \oplus \cdots \oplus V_k$$

con ciascun $V_i$ irriducibile.
}
\pf{Dimostrazione}{
  Per induzione su $n = \dim V$:
  \begin{itemize}
    \item \textbf{Caso base}: $n = 0$. La rappresentazione è ${0}$, somma diretta vuota.
    \item \textbf{Passo induttivo}: Sia $n \geq 1$ e supponiamo il risultato vero per tutte le rappresentazioni di dimensione $< n$.
      \begin{itemize}
      \item Se $V$ è irriducibile, allora $V$ è già la decomposizione cercata (con un solo addendo).
        \item Se $V$ è riducibile, allora esiste una sottorappresentazione propria $U \subsetneq V$ con $U \neq {0}$. Per il Lemma di Maschke (nella versione con il proiettore $G$-equivariante), esiste una sottorappresentazione $W \subsetneq V$ tale che
          $$V = U \oplus W, \quad \dim(U) < n, \quad \dim(W) < n.$$
          Per ipotesi induttiva, sia $U$ che $W$ si decompongono in somma diretta di irriducibili. Concatenando le decomposizioni:
          $$V = U \oplus W = (U_1 \oplus \cdots \oplus U_r) \oplus (W_1 \oplus \cdots \oplus W_s).$$
      \end{itemize}
  \end{itemize}
}

\section{Lemma di Shur}
Questo è uno dei risultati più potenti e usati di tutta la teoria delle rappresentazioni. Dice che tra rappresentazioni irriducibili, i morfismi sono pochissimi e molto rigidi.
\mlenma{Shur}{
  Sia $G$ un gruppo finito, $K = \mathbb{C}$ (campo algebricamente chiuso con $\mathrm{char}(K) = 0$). Siano $(V_1, \rho_1)$ e $(V_2, \rho_2)$ rappresentazioni irriducibili di $G$, e sia $f: V_1 \to V_2$ un morfismo di rappresentazioni. Allora:
\begin{itemize}
\item Se $V_1 \not\cong V_2$ (non sono isomorfe), allora $f = 0$ (l'applicazione nulla).
  \item Se $V_1 = V_2 =: V$ (stesso spazio), allora $\exists, \lambda \in \mathbb{C}$ tale che $f = \lambda, \mathrm{Id}_V$ (omotetia).
\end{itemize}

In breve: tra irriducibili non isomorfe c'è solo il morfismo nullo; su una irriducibile ogni endomorfismo è uno scalare.
}

\pf{Dimostrazione}{
  \textbf{Parte 1}. Supponiamo $f \neq 0$.

Poiché $\ker f$ è una sottorappresentazione di $V_1$ (l'abbiamo dimostrato nella lezione precedente) e $V_1$ è irriducibile, $\ker f$ è o ${0}$ o $V_1$.

Se $\ker f = V_1$ allora $f = 0$, contraddizione. Quindi $\ker f = {0}$, cioè $f$ è iniettiva.

Analogamente, $\mathrm{Im} f$ è una sottorappresentazione di $V_2$, e $V_2$ è irriducibile, quindi $\mathrm{Im} f = {0}$ o $\mathrm{Im} f = V_2$.

Ma $f \neq 0$ implica $\mathrm{Im} f \neq {0}$, quindi $\mathrm{Im} f = V_2$, cioè $f$ è suriettiva.

Dunque $f$ è biettiva, ovvero è un isomorfismo di rappresentazioni, e $V_1 \cong V_2$. Contrappositivamente: se $V_1 \not\cong V_2$ allora $f = 0$.\\

\textbf{Parte 2}. Sia $f: V \to V$ un endomorfismo di rappresentazioni con $V$ irriducibile.

Siamo su $K = \mathbb{C}$, quindi $f$ ammette almeno un autovalore $\lambda \in \mathbb{C}$ (il polinomio caratteristico ha sempre radici complesse).

Consideriamo $f - \lambda, \mathrm{Id}_V : V \to V$. Questo è anch'esso un morfismo di rappresentazioni (differenza di morfismi è un morfismo), e per costruzione $\ker(f - \lambda, \mathrm{Id}_V) \neq {0}$ (contiene gli autovettori di $\lambda$).

Poiché $\ker(f - \lambda, \mathrm{Id}_V)$ è una sottorappresentazione di $V$ e $V$ è irriducibile, deve essere $\ker(f - \lambda, \mathrm{Id}_V) = V$, cioè $f - \lambda, \mathrm{Id}_V = 0$, ovvero:

$$f = \lambda, \mathrm{Id}_V.$$

Perché ci serve $K$ algebricamente chiuso? Per garantire l'esistenza dell'autovalore $\lambda$. Su $\mathbb{R}$ questo non è garantito: il polinomio caratteristico potrebbe non avere radici reali, e la parte 2 fallirebbe.
}

Da cui possiamo ricavare un corollario importante
\cor{}{
  Se $G$ è abeliano e $K = \mathbb{C}$, allora ogni rappresentazione irriducibile di $G$ ha dimensione 1.
}
\pf{Dimostrazione}{
  Dimostrazione. Per ogni $h \in G$, la mappa $\rho(h): V \to V$ è un morfismo di rappresentazioni (perché $G$ è abeliano: $\rho(h)\rho(g) = \rho(hg) = \rho(gh) = \rho(g)\rho(h)$). Per Schur, $\rho(h) = \lambda_h, \mathrm{Id}_V$ per qualche $\lambda_h \in \mathbb{C}$. Quindi ogni sottospazio di $V$ è stabile sotto $G$. Ma $V$ è irriducibile, quindi non ha sottospazi propri, il che forza $\dim V = 1$.
}

\section{Caratteri}

Abbiamo ora uno strumento potentissimo per \textbf{classificare} le rappresentazioni senza lavorare esplicitamente con le matrici: i \textbf{caratteri}.

\dfn{Traccia}{
Sia $V$ uno spazio vettoriale di dimensione finita e $f: V \to V$ un'applicazione lineare. In qualunque base di $V$, la matrice di $f$ è $A = (a_{ij})$. La traccia di $f$ è:

$$\mathrm{tr}(f) := \sum_i a_{ii}$$

cioè la somma degli elementi diagonali.
}

Le \textbf{proprieta' fondamentali} della traccia sono:
\begin{enumerate}
\item $\mathrm{tr}(\mathrm{Id}_V) = \dim V$ — è uno dei coefficienti del polinomio caratteristico.
\item \textbf{Non dipende dalla base}: se $A$ e $B = P^{-1}AP$ sono la stessa mappa in basi diverse, allora $\mathrm{tr}(B) = \mathrm{tr}(P^{-1}AP) = \mathrm{tr}(A)$ (per ogni $A$ invertibile, $\mathrm{tr}(PQ) = \mathrm{tr}(QP)$).
\item $\mathrm{tr}(f) = \sum_i \lambda_i$ dove $\lambda_i$ sono gli autovalori di $f$ contati con molteplicità (radici del polinomio caratteristico).
\end{enumerate}
La proprietà 2 è cruciale: la traccia è un \textbf{invariante} dell'operatore, indipendente dalla scelta della base.

\dfn{Carattere}{
  Sia $(V, \rho)$ una rappresentazione di $G$ su $\mathbb{C}$. Il carattere di $\rho$ è la funzione:

$$\chi_\rho : G \longrightarrow \mathbb{C}, \qquad g \longmapsto \mathrm{tr}(\rho(g))$$
}
\subsection{Proprieta' dei Caratteri}
Le proprieta' piu' interessanti dei caratteri sono:
\begin{enumerate}
\item $\chi_\rho(e) = \dim V$.
\item $\chi_\rho$ è costante sulle classi di coniugio di $G$:
  $$g' = hgh^{-1} \implies \chi_\rho(g') = \chi_\rho(g).$$
    Una funzione costante sulle classi di coniugio si chiama funzione di classe. I caratteri sono dunque funzioni di classe — questo è fondamentale: non dobbiamo calcolare $\chi$ su tutti gli elementi del gruppo, basta una volta per ogni classe di coniugio.
\item $\chi_\rho(g^{-1}) = \overline{\chi_\rho(g)}$ (il complesso coniugato), per $G$ finito.
\end{enumerate}

\nt{
  Gli autovalori di $\rho(g)$ sono radici dell'unità (perché $g^n = e$ per qualche $n$, quindi $\rho(g)^n = I$), quindi hanno modulo 1, e gli autovalori di $\rho(g^{-1}) = \rho(g)^{-1}$ sono i loro inversi, che coincidono con i coniugati.
}

\pf{Dimostrazione delle proprieta'}{
  Dimostriamo in ordine:
  \begin{enumerate}
  \item $\rho(e) = \mathrm{Id}_V$, e $\mathrm{tr}(\mathrm{Id}_V) = \dim V$.
    \item $\chi_\rho(hgh^{-1}) = \mathrm{tr}(\rho(h)\rho(g)\rho(h)^{-1}) = \mathrm{tr}(\rho(g)) = \chi_\rho(g)$, usando ciclicità della traccia.
  \end{enumerate}
}

\ex{Azione naturale di $ S_3 $ su $ \mathbb{C}^3 $}{
  Sia $G = S_3$ con la rappresentazione naturale su $V = \mathbb{C}^3$ con base ${e_1, e_2, e_3}$:

$$g \cdot (x_1, x_2, x_3) = (x_{g(1)}, x_{g(2)}, x_{g(3)})$$

Le classi di coniugio di $S_3$ sono tre: ${e}$, le trasposizioni ${(12),(13),(23)}$, i 3-cicli ${(123),(132)}$.\\

Calcolo di $\chi$ per ciascuna classe:
\begin{itemize}
\item $g = \mathrm{id}$: $\rho(e) = I_3$, quindi: $$\chi(e) = \mathrm{tr}(I_3) = 3$$
  \item $g = (12)$: la matrice permuta le prime due coordinate, fissa la terza: $$\rho((12)) = \begin{pmatrix}0&1&0\\1&0&0\\0&0&1\end{pmatrix}, \qquad \chi((12)) = 0+0+1 = 1$$

Lo stesso vale per tutte le trasposizioni: ciascuna fissa esattamente un indice, quindi la traccia è sempre 1.
\item $g = (123)$: la ciclica $1\mapsto 2\mapsto 3\mapsto 1$, nessun punto fisso: $$\rho((123)) = \begin{pmatrix}0&0&1\\1&0&0\\0&1&0\end{pmatrix}, \qquad \chi((123)) = 0+0+0 = 0$$

Lo stesso per $(132)$: nessun punto fisso, traccia 0.
\end{itemize}
\textbf{Regola generale}: per la rappresentazione di permutazione, $\chi(g) = $ numero di punti fissi di $g$.

TODO: fai tavola

I valori sono interi — non è un caso: vedremo che i caratteri di gruppi simmetrici sono sempre interi. I valori $3, 1, 0$ codificano interamente la struttura della rappresentazione. Nella prossima lezione vedremo come usarli per decomporre $V$ in irriducibili tramite le relazioni di ortogonalità tra caratteri.
}

% \end{document}
