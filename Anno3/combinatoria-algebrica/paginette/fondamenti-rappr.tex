% \begin{document}
\chapter{Fondamenti della Teoria delle Rappresentazioni}
\section{Robe per Informatici}
\subsection{Gruppi}
\dfn{Gruppo}
{
  E' una coppia $ (G, \cdot) $ dove:
  \begin{itemize}
  \item $ G $ e' un insieme non vuoto
  \item $ \cdot $ e' un'operazione $ G \times G \to G $ (chiusa su $ G $)
  \end{itemize}

  Che soddsfa gli assiomi:
  \begin{itemize}
  \item Associativita'
  \item Esistenza dell'elemento \textit{neutro}
  \item Esistenza dell'\textit{inverso} per ogni $ a \in G $
  \end{itemize}
}
\subsubsection{Proprieta' fondamentali}
\begin{itemize}
\item \textbf{Unicita'} dell'elemento inverso e neutro
\item \textbf{Inverso del prodotto}: $ (a \cdot b)^{-1} = b^{-1} \cdot a^{-1} $
\item \textbf{Legge della cancellazione}: $ a \cdot b = a \cdot c $ moltiplicando a sx per $ a^{-1} $ si ottiene
  $ b = c $
\end{itemize}
\dfn{Sottogruppo}{
  Dato un gruppo $ (G, \cdot) $, si dice che $ B $ e' un suo sottogruppo se:
  \begin{itemize}
    \item $ B \subset G $
    \item $ (B, \cdot) $ e' un gruppo 
  \end{itemize}
}
\dfn{Operazione di Coniugio}{
Sia $G$ un gruppo e siano $x, g \in G$. Si definisce \textbf{coniugio} di $x$ tramite $g$ l'operazione che associa ad $x$ l'elemento:
$$ x^g = gxg^{-1} $$
Due elementi $x, y \in G$ si dicono \textbf{coniugati} se esiste un elemento $g \in G$ tale che $y = gxg^{-1}$. Questa è una relazione di equivalenza che partiziona il gruppo in \textbf{classi di coniugio}.
}

\nt{
\textbf{Osservazioni e Proprietà:}
\begin{itemize}
    \item \textbf{Automorfismo Interno:} Per ogni $g \in G$, la mappa $\gamma_g: G \to G$ definita da $\gamma_g(x) = gxg^{-1}$ è un automorfismo del gruppo (chiamato automorfismo interno). Questo significa che il coniugio preserva tutte le proprietà algebriche dell'elemento (ad esempio, $x$ e $gxg^{-1}$ hanno sempre lo stesso ordine).
    \item \textbf{Nei Gruppi Abeliani:} Se $G$ è commutativo, il coniugio è banale: $gxg^{-1} = xgg^{-1} = x$. In questo caso, ogni elemento forma una classe di coniugio a sé stante.
    \item \textbf{Invarianza dei Caratteri:} Questa è la proprietà più importante per il Modulo 2. I caratteri di una rappresentazione sono \textbf{funzioni di classe}, ovvero assumono lo stesso valore su tutti gli elementi di una stessa classe di coniugio: $\chi(x) = \chi(gxg^{-1})$.
    \item \textbf{Legame con la Normalità:} Un sottogruppo $N$ è normale ($N \trianglelefteq G$) se e solo se è un'unione di classi di coniugio, ovvero se è "chiuso" rispetto all'operazione di coniugio.
\end{itemize}
}
\dfn{Sottogruppo Normale}{
Sia $G$ un gruppo e $N$ un suo sottogruppo ($N \le G$). Diciamo che $N$ è un \textbf{sottogruppo normale} di $G$, e si denota con il simbolo $N \trianglelefteq G$, se è invariante rispetto all'operazione di coniugio per qualsiasi elemento del gruppo. 
In formule, deve valere:
$$g n g^{-1} \in N \quad \forall n \in N, \forall g \in G$$
}

\nt{
\textbf{Condizioni Equivalenti}
Nella pratica algebrica, dire che $N \trianglelefteq G$ equivale a verificare una di queste due proprietà:
\begin{itemize}
    \item \textbf{Invarianza globale per coniugio:} $g N g^{-1} = N$ per ogni $g \in G$.
    \item \textbf{Coincidenza dei laterali:} I laterali sinistri coincidono sempre con i laterali destri. Ovvero, $gN = Ng$ per ogni $g \in G$. (Attenzione: questo non significa che gli elementi commutino individualmente, cioè $gn = ng$, ma che gli \textit{insiemi} risultanti siano identici).
\end{itemize}

\textbf{Esempi e Proprietà Fondamentali}
\begin{itemize}
    \item \textbf{Gruppi Abeliani:} Se il gruppo $G$ è commutativo (come i gruppi ciclici o il Gruppo di Klein $V_4$), allora \textit{ogni} suo sottogruppo è banalmente normale, poiché $gng^{-1} = ngg^{-1} = n$.
    \item \textbf{Nucleo di un Omomorfismo:} Il nucleo di un qualsiasi omomorfismo $\phi: G \to H$ è sempre un sottogruppo normale di $G$ ($\ker(\phi) \trianglelefteq G$).
    \item \textbf{Il Centro del Gruppo:} Il centro $Z(G)$, contenendo gli elementi che commutano con tutto, è sempre un sottogruppo normale di $G$.
\end{itemize}

\textbf{Il Fine Ultimo: Il Gruppo Quoziente}
La normalità è la condizione necessaria e sufficiente per poter definire un'operazione coerente sull'insieme dei laterali $\{gN \mid g \in G\}$. Solo se $N \trianglelefteq G$, il prodotto $(aN) \cdot (bN) = (ab)N$ è ben definito. 
Questo ci permette di creare il \textbf{Gruppo Quoziente} $G/N$, una struttura fondamentale che "semplifica" il gruppo di partenza collassando tutto il sottogruppo $N$ nell'elemento neutro.
}

\subsubsection{Teoremi principali}
\thm{Lagrange}{
  Se $ G $ e' un gruppo finito e $ H $ un suo sottogruppo, allora la cardinalita' degli elementi di $ G $ divide esattamente 
}

\dfn{Il Gruppo Simmetrico $Sym(V)$}{
Sia $V$ uno spazio vettoriale (considerato qui come un semplice insieme di punti). Il \textbf{Gruppo Simmetrico} di $V$, denotato con $Sym(V)$ o $\text{Perm}(V)$, è l'insieme di tutte le funzioni biunivoche (permutazioni) $f: V \to V$. 
Sotto l'operazione di composizione di funzioni, $Sym(V)$ forma un gruppo.
}

\nt{
\textbf{Distinzione tra $Sym(V)$ e $GL(V)$:}
\begin{itemize}
    \item \textbf{Natura delle trasformazioni:} Mentre $Sym(V)$ contiene \textit{qualsiasi} funzione biettiva (anche quelle che "rimescolano" i vettori in modo selvaggio e non lineare), il gruppo $GL(V)$ è il sottogruppo di $Sym(V)$ costituito solo dalle trasformazioni che sono anche \textbf{lineari}.
    \item \textbf{Inclusione:} $GL(V) \le Sym(V)$. In termini di Teoria delle Rappresentazioni, diciamo che una rappresentazione è un'azione di $G$ su $V$ tale che l'immagine dell'omomorfismo non sia semplicemente in $Sym(V)$, ma sia contenuta interamente in $GL(V)$.
    \item \textbf{Esempio concettuale:} Se $V = \mathbb{R}^2$, una funzione che sposta il vettore $(1,1)$ in $(2,2)$ e il vettore $(2,2)$ in $(5,0)$ può appartenere a $Sym(V)$ (se è biettiva), ma non potrà mai appartenere a $GL(V)$ perché non rispetta la proporzionalità ($\text{linearità}$).
    \item \textbf{Il "filtro" della Rappresentazione:} Quando scriviamo $\rho: G \to GL(V)$, stiamo imponendo che ogni simmetria del gruppo $G$ agisca sullo spazio $V$ rispettando la sua struttura vettoriale (somma e prodotto per scalare), non solo come un semplice rimescolamento di punti.
\end{itemize}
}

\dfn{Azione di un Gruppo}{
Sia $G$ un gruppo e $X$ un insieme non vuoto. Un' \textbf{azione} (a sinistra) di $G$ su $X$ è una funzione $\cdot: G \times X \to X$ che associa a ogni coppia $(g, x)$ un elemento $g \cdot x \in X$, tale che siano soddisfatti i seguenti assiomi:
\begin{enumerate}
    \item \textbf{Identità:} $e \cdot x = x$ per ogni $x \in X$ (dove $e$ è l'elemento neutro di $G$).
    \item \textbf{Compatibilità:} $(gh) \cdot x = g \cdot (h \cdot x)$ per ogni $g, h \in G$ e $x \in X$.
\end{enumerate}
}

\nt{
\textbf{Concetti Chiave e Proprietà:}
\begin{itemize}
    \item \textbf{Omomorfismo di Permutazione:} Un'azione di $G$ su $X$ è equivalente a un omomorfismo di gruppi $\phi: G \to \text{Sym}(X)$. In questo senso, ogni elemento del gruppo viene visto come una permutazione degli elementi di $X$.
    \item \textbf{Orbita:} L'orbita di un elemento $x \in X$ è l'insieme $G \cdot x = \{g \cdot x \mid g \in G\}$. Le orbite formano una partizione dell'insieme $X$.
    \item \textbf{Stabilizzatore:} Lo stabilizzatore di $x \in X$ è il sottogruppo $G_x = \{g \in G \mid g \cdot x = x\}$. Contiene tutti gli elementi del gruppo che "lasciano fermo" $x$.
    \item \textbf{Teorema Orbita-Stabilizzatore:} Se $G$ è finito, la cardinalità dell'orbita di $x$ è data dal numero di laterali dello stabilizzatore: $|G \cdot x| = |G| / |G_x|$.
    \item \textbf{Dall'Azione alla Rappresentazione:} Se l'insieme $X$ è uno spazio vettoriale $V$ e l'azione è lineare (cioè $g \cdot (v+w) = g \cdot v + g \cdot w$ e $g \cdot (\lambda v) = \lambda (g \cdot v)$), allora l'azione è esattamente una \textbf{rappresentazione lineare} di $G$.
\end{itemize}
}
\subsection{Gruppi ciclici}
\dfn{Gruppo Ciclico}{
Un gruppo $(G, \cdot)$ si dice \textbf{ciclico} se esiste un elemento $g \in G$, detto \textbf{generatore}, tale che ogni elemento di $G$ possa essere espresso come potenza intera di $g$:
$$G = \langle g \rangle = \{g^k \mid k \in \mathbb{Z}\}$$
}

\dfn{Gruppo Simmetrico $S_n$ e Classi di Coniugio}{
Il \textbf{gruppo simmetrico} $S_n$ è l'insieme di tutte le permutazioni di un insieme di $n$ elementi distinti. L'operazione del gruppo è la composizione di funzioni e l'ordine è $|S_n| = n!$.
Due permutazioni $\sigma, \tau \in S_n$ appartengono alla stessa \textbf{classe di coniugio} se e solo se hanno la stessa \textbf{struttura ciclica}, ovvero se presentano lo stesso numero di cicli della stessa lunghezza nella loro scomposizione in cicli disgiunti.
}

\nt{
\textbf{Proprietà e Relazione con le Partizioni:}
\begin{itemize}
    \item \textbf{Corrispondenza biunivoca:} Le classi di coniugio di $S_n$ sono in corrispondenza biunivoca con le \textbf{partizioni} dell'intero $n$. Una partizione $\lambda = (\lambda_1, \lambda_2, \dots, \lambda_k)$ tale che $\sum \lambda_i = n$ definisce univocamente una classe di coniugio.
    \item \textbf{Numero di Rappresentazioni:} In virtù della teoria generale, il numero di rappresentazioni irriducibili distinte di $S_n$ su $\mathbb{C}$ è esattamente uguale al numero di partizioni $p(n)$.
    \item \textbf{Esempio $S_3$ ($n=3$):} Le partizioni di 3 sono:
    \begin{itemize}
        \item $(1,1,1) \to$ Identità (cicli di lunghezza 1).
        \item $(2,1) \to$ Trasposizioni $\{(12), (13), (23)\}$.
        \item $(3) \to$ 3-cicli $\{(123), (132)\}$.
    \end{itemize}
    \item \textbf{Rappresentazioni Notevoli:} Ogni $S_n$ possiede sempre almeno due rappresentazioni di grado 1: la \textit{banale} ($\chi(g)=1$ per ogni $g$) e la \textit{segnatura} ($\chi(g)=\text{sgn}(g)$).
    \item \textbf{Diagrammi di Young:} Le rappresentazioni irriducibili di $S_n$ vengono classificate e costruite tramite i \textbf{Diagrammi di Young}, che sono la rappresentazione grafica delle partizioni di $n$.
\end{itemize}
}

\subsubsection*{Classificazione e Struttura}
I gruppi ciclici sono classificati in base al loro ordine $|G|$:
\begin{itemize}
    \item Se $|G| = \infty$, allora $G \cong (\mathbb{Z}, +)$.
    \item Se $|G| = n < \infty$, allora $G \cong (\mathbb{Z}/n\mathbb{Z}, +)$, ovvero il gruppo delle classi di resto modulo $n$.
\end{itemize}

\subsubsection*{Proprietà Fondamentali}
\begin{enumerate}
    \item \textbf{Abelianità:} Ogni gruppo ciclico è abeliano. Infatti, $g^a \cdot g^b = g^{a+b} = g^{b+a} = g^b \cdot g^a$.
    \item \textbf{Sottogruppi:} Ogni sottogruppo di un gruppo ciclico è a sua volta ciclico.
    \item \textbf{Teorema dei Divisori:} Se $G$ è un gruppo ciclico di ordine $n$, allora per ogni divisore $d$ di $n$ esiste un unico sottogruppo $H \le G$ tale che $|H| = d$.
    \item \textbf{Generatori:} Un elemento $g^k$ di un gruppo ciclico d'ordine $n$ è un generatore di $G$ se e solo se $\gcd(k, n) = 1$. Il numero di tali generatori è dato dalla funzione $\varphi(n)$ di Eulero.
\end{enumerate}

\subsection{Centro di un Gruppo}
\dfn{Centro di un Gruppo}{
Sia $G$ un gruppo. Il \textbf{centro} di $G$, tipicamente denotato con $Z(G)$, è l'insieme di tutti gli elementi di $G$ che commutano con ogni elemento del gruppo stesso. In simboli:
$$ Z(G) = \{ z \in G \mid z \cdot g = g \cdot z, \forall g \in G \} $$
}

\nt{
\textbf{Proprietà fondamentali ed Esempi:}
\begin{itemize}
    \item \textbf{È un sottogruppo:} L'elemento neutro $e$ commuta con tutto, quindi $e \in Z(G)$. Essendo chiuso rispetto al prodotto e all'inverso, costituisce un sottogruppo a tutti gli effetti ($Z(G) \le G$).
    \item \textbf{È un sottogruppo normale:} Poiché ogni elemento $z \in Z(G)$ commuta con tutti i $g \in G$, la coniugazione lo lascia invariato: $g z g^{-1} = z g g^{-1} = z \in Z(G)$. Di conseguenza, $Z(G) \trianglelefteq G$.
    \item \textbf{Casi limite:} Se $G$ è abeliano, il centro coincide con tutto il gruppo ($Z(G) = G$). Se invece $Z(G) = \{e\}$, si dice che il gruppo ha centro banale (es. il gruppo simmetrico $S_n$ per $n \ge 3$).
    \item \textbf{Applicazione in Combinatoria Algebrica:} Il centro del gruppo generale lineare $GL(V)$ è costituito esattamente dalle matrici scalari non nulle ($Z = \{\lambda I \mid \lambda \in K^\times\}$). Questo fatto è il motore logico della dimostrazione del \textbf{Lemma di Schur}.
\end{itemize}
}

\subsection*{Il Gruppo Lineare Generale $GL(V)$}

\dfn{Definizione Formale}{
Sia $V$ uno spazio vettoriale su un campo $K$. Il \textbf{Gruppo Lineare Generale} di $V$, denotato con $GL(V)$ o $\text{Aut}(V)$, è l'insieme di tutti gli \textbf{automorfismi lineari} dello spazio $V$ (ovvero, tutte le applicazioni lineari biunivoche $f: V \to V$).

La struttura $(GL(V), \circ)$ forma un gruppo dove l'operazione interna è la \textbf{composizione di funzioni}:
\begin{itemize}
    \item \textbf{Chiusura:} La composizione di due automorfismi lineari è ancora un automorfismo lineare.
    \item \textbf{Elemento neutro:} L'applicazione identica $\text{id}_V$ (tale che $\text{id}_V(v) = v, \forall v \in V$).
    \item \textbf{Inverso:} L'applicazione lineare inversa $f^{-1}$, che esiste sempre ed è unica poiché $f$ è una biiezione.
\end{itemize}
}

\subsubsection*{Rappresentazione Matriciale $GL(n, K)$}
Se lo spazio vettoriale $V$ ha dimensione finita $n$, fissata una base di $V$, ogni isomorfismo lineare può essere rappresentato univocamente da una matrice quadrata di ordine $n$. Di conseguenza, $GL(V)$ è isomorfo al gruppo delle matrici invertibili a coefficienti in $K$:
$$ GL(n, K) = \{ A \in M_{n}(K) \mid \det(A) \neq 0 \} $$
In questa veste, l'operazione del gruppo diventa la \textbf{moltiplicazione riga per colonna} tra matrici e l'elemento neutro è la matrice identità $I_n$.

\subsubsection*{Proprietà Fondamentali}
\begin{enumerate}
    \item \textbf{Non Abelianità:} Se la dimensione $n \ge 2$, il gruppo $GL(V)$ è tipicamente \textbf{non commutativo} (poiché il prodotto di matrici non commuta in generale).
    \item \textbf{Il Centro del Gruppo:} Il centro $Z(GL(V))$ (ovvero l'insieme degli elementi che commutano con ogni altro elemento del gruppo) è costituito esattamente dalle \textbf{matrici scalari} non nulle: $Z = \{\lambda I_n \mid \lambda \in K^\times\}$. Questo fatto è il motore logico del \textbf{Lemma di Schur}.
    \item \textbf{Sottogruppo Speciale Lineare:} Il nucleo dell'omomorfismo determinante ($\det: GL(n, K) \to K^\times$) forma un importante sottogruppo normale di $GL(n, K)$, chiamato \textit{Gruppo Speciale Lineare} $SL(n, K)$, composto da tutte e sole le matrici con determinante uguale a $1$.
\end{enumerate}

\subsubsection*{Il Ruolo Centrale nel Modulo 2}
Questa definizione è il perno del corso di Combinatoria Algebrica. Definire una rappresentazione di un gruppo finito astratto $G$ su uno spazio vettoriale $V$ significa esattamente stabilire un omomorfismo:
$$ \rho: G \to GL(V) $$
Stiamo, di fatto, "traducendo" la struttura moltiplicativa del gruppo astratto $G$ in operazioni tra matrici invertibili, permettendoci così di sfruttare tutta la potenza dell'Algebra Lineare (autovalori, traccia, diagonalizzazione) per studiare le simmetrie del gruppo.

\subsection{Campo}
\dfn{Campo}{
Un \textbf{campo} $ K $ e' una struttura algebrica dotata di due operazioni:
\begin{itemize}
  \item \textit{Somma}: t.c. $ (K, +) $ e' un \textit{gruppo abeliano} (con elem neutro 0)
  \item \textit{Prodotto}: t.c. $ (K \setminus \{0\}, \cdot ) $ e' un \textit{gruppo abeliano} (con elem neutro 1)
  \item Vale la proprieta'  distributiva del prodotto rispetto alla somma
\end{itemize}
}

Come conseguenza diretta, si ha che l'elemento neutro della somma diventa \textbf{elemento assorbente} ($ \forall a \in K. a \cdot 0 = 0 $). Infatti, se un prodotto e' 0 allora e' sicuro che almeno uno degli operandi e' 0.

\subsubsection{Proprieta' Fondamentali}
\textbf{Caratteristica del Campo} $ (char(K)) $: E' il piu' piccol intero positivo $ p $ tale che sommando l'elemento identita' 1 a se' stesso $ p $ volte si ottiene 0 (l'elemento neutro) (cioe' $ 1 + ... + 1 $ = 0). Se non esiste un valore $ p $ (non si ritorna mai all'elemento neutro) allora $ char(K) = 0 $.

\textbf{Chiusura Algebrica}: Un campo $ K $ si dice algebricamentechiuso se ogni polinomio non costante a coefficienti in $ K $ ha almeno una raidce in $ K $. Il campo $ \mathbb{C} $ e' algebricamente chiuso, mentre $ \mathbb{R} $ non lo e' (ad esempio, $ x^2 + 1 = 0 $).

\subsection{Anelli}
\dfn{Anello}{
  Un anello e' una terna $ (R, +, \cdot) $, dove $ R $ e' un insieme dotato di due operazioni binarie interne (somma e prodotto), tale che:
  \begin{itemize}
    \item $ (R, +) $ e' un \textit{gruppo abeliano}
    \item $ (R, \cdot) $ e' un \textit{monoide}:
      \begin{itemize}
      \item Associatività: $(a \cdot b) \cdot c = a \cdot (b \cdot c), \forall a, b, c \in R$
      \item Unità: $\exists 1 \in R \mid a \cdot 1 = 1 \cdot a = a, \forall a \in R$
      \end{itemize}
    \item Proprieta' distributiva (sx e dx)
  \end{itemize} 
}

Come conseguenza diretta, si ha che l'elemento neutro della somma diventa \textbf{elemento assorbente} ($ \forall a \in K. a \cdot 0 = 0 $). Infatti, se un prodotto e' 0 allora e' sicuro che almeno uno degli operandi e' 0.

\nt{
A differenza dei campi, in un anello generale non si richiede che il prodotto sia commutativo ($ a \cdot b $ può essere diverso da $ b \cdot a $), e non si richiede che ogni elemento non nullo abbia un inverso moltiplicativo (cioè non si può sempre "dividere").
}

\subsubsection{Proprieta' fondamentali}
Per manipolare gli anelli, definiamo alcune categorie di elementi e strutture interne fondamentali:

\begin{itemize}
    \item \textbf{Divisori dello zero:} Un elemento $a \neq 0$ si dice divisore dello zero se esiste un elemento $b \neq 0$ tale che $a \cdot b = 0$. Si osservi che nei campi questa eventualità non si verifica mai per definizione.
    \item \textbf{Elementi Invertibili (Unità):} Gli elementi di $R$ che possiedono un inverso moltiplicativo formano un gruppo rispetto all'operazione di prodotto, indicato con il simbolo $R^\times$ (o $U(R)$).
    \item \textbf{Ideali:} Un sottoinsieme $I \subseteq R$ è un \textbf{ideale sinistro} se è un sottogruppo additivo e ``assorbe'' il prodotto da sinistra: ovvero, per ogni $r \in R$ e ogni $x \in I$, si ha che $r \cdot x \in I$. Gli ideali rappresentano per gli anelli ciò che i sottogruppi normali rappresentano per i gruppi, permettendo la costruzione degli \textbf{anelli quoziente} $R/I$.
\end{itemize}

\subsection{Spazio Vettoriale}
\dfn{Spazio Vettoriale}{
Sia $K$ un campo (i cui elementi sono detti \textit{scalari}). Un insieme $V$ (i cui elementi sono detti \textit{vettori}) è un \textbf{spazio vettoriale su $K$} (o $K$-spazio vettoriale) se è dotato di due operazioni:
\begin{enumerate}
    \item \textbf{Somma interna:} un'operazione $+: V \times V \to V$ che rende $(V, +)$ un gruppo abeliano (commutativa, associativa, con elemento neutro $0_V$ e opposto per ogni vettore).
    \item \textbf{Prodotto per uno scalare:} un'operazione esterna $\cdot: K \times V \to V$ tale che, per ogni scalare $\alpha, \beta \in K$ e per ogni vettore $u, v \in V$, valgano i seguenti quattro assiomi:
    \begin{itemize}
        \item \textbf{Distributività rispetto alla somma vettoriale:} $\alpha \cdot (u + v) = (\alpha \cdot u) + (\alpha \cdot v)$
        \item \textbf{Distributività rispetto alla somma scalare:} $(\alpha + \beta) \cdot v = (\alpha \cdot v) + (\beta \cdot v)$
        \item \textbf{Compatibilità del prodotto:} $(\alpha \beta) \cdot v = \alpha \cdot (\beta \cdot v)$
        \item \textbf{Azione dell'identità scalare:} $1_K \cdot v = v$ (dove $1_K$ è l'elemento neutro moltiplicativo del campo $K$).
    \end{itemize}
\end{enumerate}
}

\subsection*{L'Importanza del Campo $K$ in Teoria delle Rappresentazioni}
Nel contesto dello studio delle rappresentazioni $\rho: G \to GL(V)$, le proprietà algebriche del campo $K$ determinano la validità dei teoremi fondamentali:

\begin{itemize}
    \item \textbf{Chiusura Algebrica (Esistenza degli autovalori):} Se lavoriamo su $K = \mathbb{C}$ (che è un campo algebricamente chiuso), il Teorema Fondamentale dell'Algebra ci garantisce che ogni endomorfismo lineare abbia sempre almeno un autovalore. Questa proprietà è il motore logico che fa funzionare il \textbf{Lemma di Schur} e ci permette di diagonalizzare l'azione del gruppo.
    \item \textbf{Caratteristica del Campo (Divisione per $|G|$):} Affinché sia valido il \textbf{Teorema di Maschke} (e le rappresentazioni siano completamente riducibili), è necessario che la caratteristica del campo, $\text{char}(K)$, non divida l'ordine del gruppo finito $|G|$. Solo sotto questa condizione l'elemento $|G| \cdot 1_K$ è invertibile in $K$, rendendo possibile l'operazione di ``media sul gruppo'' per costruire i proiettori equivarianti.
    \item \textbf{Dimensione Relativa:} La dimensione di uno spazio vettoriale dipende strettamente da $K$. Ad esempio, l'insieme dei numeri complessi $\mathbb{C}$ ha dimensione $1$ se inteso come $\mathbb{C}$-spazio vettoriale, ma possiede dimensione $2$ se lo strutturiamo come $\mathbb{R}$-spazio vettoriale (con base $\{1, i\}$).
\end{itemize}

\section{Omomorfismi, Isomorfismi e Automorfismi}

\subsection*{1. Omomorfismo di Gruppi}
Siano $(G, \cdot)$ e $(H, *)$ due gruppi. Una funzione $\phi: G \to H$ si dice \textbf{omomorfismo} se preserva l'operazione di gruppo, ovvero se:
$$ \phi(x \cdot y) = \phi(x) * \phi(y) \quad \forall x, y \in G $$
Da questa definizione derivano due proprietà strutturali fondamentali:
\begin{itemize}
    \item $\phi(e_G) = e_H$ (l'elemento neutro viene mappato nell'elemento neutro).
    \item $\phi(x^{-1}) = [\phi(x)]^{-1}$ (l'inverso viene mappato nell'inverso).
\end{itemize}

\textbf{Strutture associate a un omomorfismo:}
\begin{itemize}
    \item \textbf{Nucleo (Kernel):} $\ker(\phi) = \{x \in G \mid \phi(x) = e_H\}$. Il nucleo misura quanto l'omomorfismo "collassa" il gruppo di partenza ed è sempre un \textit{sottogruppo normale} di $G$ ($\ker(\phi) \trianglelefteq G$).
    \item \textbf{Immagine:} $\text{Im}(\phi) = \{\phi(x) \mid x \in G\}$. L'immagine rappresenta la porzione del codominio effettivamente raggiunta ed è sempre un sottogruppo di $H$ ($\text{Im}(\phi) \le H$).
\end{itemize}

\subsection*{2. Isomorfismo}
Un omomorfismo $\phi: G \to H$ si dice \textbf{isomorfismo} se la funzione $\phi$ è biunivoca (cioè iniettiva e suriettiva).
\begin{itemize}
    \item \textbf{Criterio di iniettività:} Un omomorfismo $\phi$ è iniettivo se e solo se $\ker(\phi) = \{e_G\}$.
    \item Se esiste un isomorfismo tra $G$ e $H$, i due gruppi si dicono isomorfi e si scrive $G \cong H$. Strutturalmente, sono indistinguibili dal punto di vista algebrico.
\end{itemize}

\subsection*{3. Automorfismo}
Un \textbf{automorfismo} è un isomorfismo di un gruppo in sé stesso, ovvero una mappa biunivoca $\phi: G \to G$ che preserva le operazioni.
\begin{itemize}
    \item L'insieme di tutti gli automorfismi di un gruppo $G$, dotato dell'operazione di composizione di funzioni, forma un gruppo a sua volta, denotato con $\text{Aut}(G)$.
    \item \textbf{Automorfismi interni:} Fissato un elemento $g \in G$, la mappa di coniugio $\gamma_g(x) = gxg^{-1}$ è sempre un automorfismo di $G$. L'insieme di questi automorfismi forma un sottogruppo denotato con $\text{Inn}(G) \trianglelefteq \text{Aut}(G)$.
\end{itemize}

\section*{L'Omomorfismo Determinante}

\subsection*{Definizione}
Sia $K$ un campo e $K^\times = K \setminus \{0\}$ il suo gruppo moltiplicativo (formato da tutti gli elementi non nulli di $K$ con l'operazione di prodotto).
L'applicazione \textbf{determinante} è una mappa:
$$ \det: GL(n, K) \to K^\times $$
che associa a ogni matrice invertibile $A$ il suo determinante $\det(A)$ (che è uno scalare in $K$). Poiché la matrice è invertibile, $\det(A) \neq 0$, quindi il codominio $K^\times$ è corretto.

\subsection*{La Proprietà di Omomorfismo}
Questa mappa è un omomorfismo di gruppi grazie al celebre \textbf{Teorema di Binet}, il quale garantisce che il determinante del prodotto di due matrici è uguale al prodotto dei loro determinanti:
$$ \det(A \cdot B) = \det(A) \cdot \det(B) \quad \forall A, B \in GL(n, K) $$
In altre parole, la mappa $\det$ preserva (e trasporta) l'operazione di moltiplicazione dal "complicato" gruppo delle matrici al "semplice" gruppo degli scalari.

\subsection*{Nucleo (Kernel) e Immagine}
Applicando le definizioni strutturali degli omomorfismi a questa mappa specifica, otteniamo due informazioni preziose:

\begin{itemize}
    \item \textbf{Immagine:} La mappa è \textbf{suriettiva}. Per ogni scalare $\lambda \in K^\times$, esiste sempre almeno una matrice in $GL(n, K)$ che ha $\lambda$ come determinante (basta prendere la matrice identità e sostituire il primo $1$ in alto a sinistra con $\lambda$). Quindi, $\text{Im}(\det) = K^\times$.
    
    \item \textbf{Nucleo:} Il nucleo è formato da tutte le matrici mappate nell'elemento neutro del codominio (che per la moltiplicazione in $K^\times$ è $1$).
    $$ \ker(\det) = \{ A \in GL(n, K) \mid \det(A) = 1 \} $$
    Questo insieme definisce il \textbf{Gruppo Speciale Lineare}, denotato con $SL(n, K)$. Poiché è il nucleo di un omomorfismo, $SL(n, K)$ è automaticamente un \textbf{sottogruppo normale} di $GL(n, K)$ ($SL(n, K) \trianglelefteq GL(n, K)$).
\end{itemize}

\subsection*{Conseguenza: Il Primo Teorema di Omomorfismo}
Per il Primo Teorema di Omomorfismo, il quoziente del dominio rispetto al nucleo è isomorfo all'immagine. Questo ci dà una bellissima identità strutturale:
$$ \frac{GL(n, K)}{SL(n, K)} \cong K^\times $$

\subsection*{Applicazione nel Modulo 2 (Caratteri e Rappresentazioni)}
Se possediamo una rappresentazione di un gruppo $G$, ovvero un omomorfismo $\rho: G \to GL(n, \mathbb{C})$, possiamo comporla con l'omomorfismo determinante per creare una nuova mappa:
$$ \det \circ \rho: G \to \mathbb{C}^\times $$
Poiché la composizione di due omomorfismi è ancora un omomorfismo, questa nuova mappa è a tutti gli effetti una \textbf{rappresentazione di grado 1} del gruppo $G$!

\subsection*{Collegamento con la Teoria delle Rappresentazioni}
Nel contesto del nostro corso, una \textbf{rappresentazione lineare} di un gruppo finito $G$ su uno spazio vettoriale $V$ (su un campo $K$) non è altro che un omomorfismo di gruppi:
$$ \rho: G \to GL(V) $$
dove $GL(V)$ è il gruppo degli automorfismi lineari (matrici quadrate invertibili) dello spazio $V$. Se $\ker(\rho) = \{e_G\}$, la rappresentazione non "perde" informazioni sul gruppo e si dice \textbf{fedele}.

\section{Struttura algebrica}
Cosa vuol dire combinatoria algebrica?

Vuol dire studiare strutture algebriche (gruppi, anelli, campi, etc.) attraverso la combinatoria.
\dfn{Struttura algebrica}{
    Si definisce struttura algebrica una coppia $(G, *)$ dove $G$ è un insieme e $*$ è una operazione binaria su $G$
}
%
\ex{Strutture algebriche}{
    \begin{itemize}
    \item  \textbf{Gruppo}: $(G, *)$ con $*$ binaria e associativa, $G$ con elemento neutro e ogni elemento ha un inverso
    \item \textbf{Anello}: $(G, +, \cdot)$ con $+$ e $\cdot$ binarie e associative, $G$ con elemento neutro per $+$ e ogni elemento ha un inverso per $+$
    \item \textbf{Campo}: $(G, +, \cdot)$ con $+$ e $\cdot$ binarie e associative, $G$ con elemento neutro per $+$ e ogni elemento ha un inverso per $+$
    \end{itemize}
}
%
Per "astratta", dal latino, "tirata fuori", si intende estrapolata dal suo contesto originale. La combinatoria quindi, vuole studiare le struttura algebriche e darne una rappresentazione più concreta.
%
%
\dfn{
  Rappresentazione di un gruppo 
}{
        Sia $G$ un gruppo, $K$ un campo e $V$ uno spazio vettoriale su $K$. si definisce una rappresentazione di $G$ su $V$ è un omomorfismo $\rho: G \to GL(V)$ tale che $\rho(g) \rho(h) = \rho(gh)$ per ogni $g, h \in G$.
}
%
\nt{
    In altre parole è un'azione (omomorfismo è un $G \to $ ) di $G$ su $V$ con immagine $GL(V)$ tramite isomorfismi 
}
%
\dfn{}{
    Diciamo che la rappresentazione è fedele se $\rho$ è iniettivo, ovvero se $\rho(g) = \rho(h) \implies g = h$
}
\nt{
  Ovvero il nucleo di $ \rho $ contiene solo l'elemento neutro ($ ker(\rho) = \{e\} $)
}
%
\dfn{}{
    Sia $(V, \rho)$ una rappresentazione di $G$ su $V$, diciamo che un sottospazio vettoriale $U\subseteq V$ è una sottorappresentazione di $\rho$ se $\rho(g)U\subseteq U$ per ogni $g\in G$
}
\nt{
    In altre parole $(U, \rho_{|U})$ è una rappresentazione di $G$ su $U$, dove $\rho_{|U}: G \to GL(U)$ è l'omomorfismo che mappa $g \to \rho(g)_{|U}$
}
%
\ex{Rappresentazione di $C_4$ e Dipendenza dal Campo $K$}{
Si consideri il gruppo ciclico di ordine 4, $G = C_4 = \langle z \mid z^4 = 1 \rangle$, e la sua rappresentazione naturale su $V = \mathbb{R}^2$ definita mandando il generatore $z$ nella matrice di rotazione di $\pi/2$:
$$ \rho(z) = \begin{pmatrix} 0 & -1 \\ 1 & 0 \end{pmatrix} $$
L'obiettivo è analizzare la riducibilità di $\rho$ al variare del campo degli scalari $K$.

\textbf{1. Analisi sul campo reale ($K = \mathbb{R}$):}
Per determinare se esistono sottorappresentazioni proprie, cerchiamo sottospazi stabili di dimensione 1, il che equivale a cercare gli autovalori reali di $\rho(z)$. Il polinomio caratteristico è:
$$ p(\lambda) = \det(\rho(z) - \lambda I) = \det \begin{pmatrix} -\lambda & -1 \\ 1 & -\lambda \end{pmatrix} = \lambda^2 + 1 $$
Le radici di $p(\lambda)$ sono $\pm i$, le quali \textbf{non appartengono} a $\mathbb{R}$. Non esistendo autovalori reali, non esistono rette in $\mathbb{R}^2$ stabili sotto l'azione della rotazione.
\begin{center}
    \textit{Conclusione:} $\rho$ è \textbf{irriducibile} su $\mathbb{R}$.
\end{center}

\textbf{2. Analisi sul campo complesso ($K = \mathbb{C}$):}
Se estendiamo lo spazio vettoriale a $V_{\mathbb{C}} = \mathbb{C}^2$, gli autovalori $\lambda_1 = i$ e $\lambda_2 = -i$ sono ora ammissibili. Ad essi corrispondono i rispettivi autospazi (sottospazi di dimensione 1):
$$ V_i = \text{span}_{\mathbb{C}} \left\{ \begin{pmatrix} 1 \\ -i \end{pmatrix} \right\}, \quad V_{-i} = \text{span}_{\mathbb{C}} \left\{ \begin{pmatrix} 1 \\ i \end{pmatrix} \right\} $$
Questi sottospazi sono $G$-invarianti, poiché l'azione di $z$ (e delle sue potenze) si limita a moltiplicare i vettori per uno scalare ($\pm i$). Lo spazio si decompone nella somma diretta:
$$ V_{\mathbb{C}} = V_i \oplus V_{-i} $$
\begin{center}
    \textit{Conclusione:} $\rho$ è \textbf{riducibile} su $\mathbb{C}$.
\end{center}
}
%
\dfn{Rappresentazione Irriducibile}{
Una rappresentazione $(\rho, V)$ di un gruppo $G$ si dice \textbf{irriducibile} se gli unici sottospazi di $V$ che siano $G$-invarianti (sottorappresentazioni) sono i sottospazi banali $\{0\}$ e $V$ stesso. 
In caso contrario, ovvero se esiste un sottospazio proprio $0 < U < V$ tale che $\rho(g)u \in U$ per ogni $u \in U$ e $g \in G$, la rappresentazione si dice \textbf{riducibile}.
}

\nt{
\textbf{Osservazioni sulla Riducibilità:}
\begin{itemize}
    \item Una rappresentazione di dimensione 1 è sempre irriducibile per motivi dimensionali (non esistono sottospazi propri non nulli).
    \item Il concetto di irriducibilità dipende dal campo $K$ (come visto nell'esempio delle rotazioni su $\mathbb{R}$ vs $\mathbb{C}$).
    \item Scomporre una rappresentazione riducibile in somma diretta di irriducibili è l'obiettivo principale del corso (completabilità garantita dal Teorema di Maschke).
\end{itemize}
}

\ex{Rappresentazione Naturale di $S_3$}{
Sia $G = S_3$ agente su $V = \mathbb{R}^3$ tramite permutazione delle coordinate:
$$ \sigma \cdot (x_1, x_2, x_3) = (x_{\sigma^{-1}(1)}, x_{\sigma^{-1}(2)}, x_{\sigma^{-1}(3)}) $$
Questa rappresentazione è \textbf{riducibile} in quanto ammette i seguenti sottospazi $G$-invarianti propri:
\begin{enumerate}
    \item $U = \{ (t, t, t) \mid t \in \mathbb{R} \}$, sottospazio di dimensione 1 (retta diagonale). Poiché ogni permutazione scambia coordinate identiche, $U$ è puntualmente fisso.
    \item $W = \{ (x_1, x_2, x_3) \in \mathbb{R}^3 \mid x_1 + x_2 + x_3 = 0 \}$, sottospazio di dimensione 2 (piano iperortogonale a $U$). Poiché la somma è commutativa, permutare gli addendi non cambia il risultato zero.
\end{enumerate}
Si verifica facilmente che $U \cap W = \{0\}$, pertanto lo spazio si decompone nella somma diretta:
$$ V = U \oplus W $$
Dove $U$ è la rappresentazione banale e $W$ è la \textbf{rappresentazione standard} di $S_3$. Entrambe sono irriducibili.
}

\dfn{Prodotto Hermitiano $G$-invariante}{
Sia $(\rho, V)$ una rappresentazione di un gruppo finito $G$ su uno spazio vettoriale complesso $V$. Dato un prodotto hermitiano arbitrario $\langle \cdot, \cdot \rangle$ su $V$, definiamo il \textbf{prodotto hermitiano $G$-mediato} (o $G$-invariante) come:
$$ \langle v, u \rangle_G := \frac{1}{|G|} \sum_{g \in G} \langle \rho(g)v, \rho(g)u \rangle $$
per ogni $v, u \in V$. Tale prodotto soddisfa la condizione di invarianza:
$$ \langle \rho(h)v, \rho(h)u \rangle_G = \langle v, u \rangle_G \quad \forall h \in G $$
}

\mprop{Esistenza di un Prodotto Hermitiano $G$-invariante}{
Sia $(\rho, V)$ una rappresentazione di un gruppo finito $G$ su uno spazio vettoriale complesso $V$. Esiste sempre su $V$ un prodotto hermitiano $\langle \cdot, \cdot \rangle_G$ che sia $G$-invariante, ovvero tale che:
$$ \langle \rho(g)v, \rho(g)u \rangle_G = \langle v, u \rangle_G \quad \forall g \in G, \forall v, u \in V $$
}

\pf{Dimostrazione}{
Sia $H(v, u)$ un prodotto hermitiano arbitrario su $V$ (la cui esistenza è garantita dalla struttura di spazio vettoriale complesso). Definiamo il nuovo prodotto facendo la media sui trasformati degli elementi tramite il gruppo:
$$ \langle v, u \rangle_G := \frac{1}{|G|} \sum_{x \in G} H(\rho(x)v, \rho(x)u) $$
Per verificare l'invarianza, applichiamo un elemento generico $h \in G$:
$$ \langle \rho(h)v, \rho(h)u \rangle_G = \frac{1}{|G|} \sum_{x \in G} H(\rho(x)\rho(h)v, \rho(x)\rho(h)u) $$
Poiché $\rho$ è un omomorfismo, $\rho(x)\rho(h) = \rho(xh)$. Sostituendo:
$$ \langle \rho(h)v, \rho(h)u \rangle_G = \frac{1}{|G|} \sum_{x \in G} H(\rho(xh)v, \rho(xh)u) $$
Al variare di $x$ in $G$, l'elemento $xh$ percorre tutti gli elementi del gruppo esattamente una volta (la traslazione a destra è una permutazione di $G$). Possiamo quindi effettuare il cambio di variabile $k = xh$, ottenendo:
$$ \langle \rho(h)v, \rho(h)u \rangle_G = \frac{1}{|G|} \sum_{k \in G} H(\rho(k)v, \rho(k)u) = \langle v, u \rangle_G $$
Questo dimostra che il prodotto è $G$-invariante. Le proprietà di linearità, simmetria hermitiana e positività di $\langle \cdot, \cdot \rangle_G$ discendono direttamente dalle proprietà di $H$.
}



\dfn{Rappresentazione Indecomponibile}{
Una rappresentazione $(\rho, V)$ si dice \textbf{indecomponibile} se non può essere scritta come somma diretta di due sottorappresentazioni proprie non banali. Ovvero, se $V = U \oplus W$ con $U, W$ sottorappresentazioni, allora necessariamente $U=0$ oppure $W=0$.
}

\nt{
\textbf{Irriducibile vs Indecomponibile:}
\begin{itemize}
    \item Ogni rappresentazione \textbf{irriducibile} è banalmente \textbf{indecomponibile} (se non ha sottospazi stabili, a maggior ragione non può essere somma diretta di sottospazi stabili).
    \item Il viceversa non è sempre vero: esistono rappresentazioni che possiedono sottospazi stabili (sono riducibili) ma che non ammettono un complemento stabile (sono indecomponibili).
    \item \textbf{Nel Modulo 2:} Grazie al Teorema di Maschke, lavorando su un campo di caratteristica 0 (come $\mathbb{C}$) e con gruppi finiti, ogni sottorappresentazione ammette un complemento stabile. Di conseguenza, in questo contesto i concetti di irriducibile e indecomponibile \textbf{coincidono}.
\end{itemize}
}

\mprop{Teorema di Maschke (Versione Sintetica)}{
Sia $G$ un gruppo finito e $(\rho, V)$ una rappresentazione di $G$ su un campo $K$ (con $\text{char}(K) \nmid |G|$). Allora ogni sottorappresentazione $U \subseteq V$ ammette un complemento $G$-invariante $W$, tale che $V = U \oplus W$.
}

\pf{Dimostrazione}{
L'esistenza di $W$ è garantita dalla costruzione di un proiettore $G$-equivariante ottenuto tramite il \textbf{trucco della media}. Sia $\pi: V \to U$ una proiezione lineare arbitraria. Definiamo:
$$ \tilde{\pi} = \frac{1}{|G|} \sum_{g \in G} \rho(g) \pi \rho(g)^{-1} $$
Si dimostra che $\tilde{\pi}$ è un morfismo di rappresentazioni e che $\ker(\tilde{\pi})$ è il complemento stabile $W$ cercato.
}

\ex{}{
    $V = C^2$
    $(Z, +)$
%
    \[
        \rho(n) = \begin{pmatrix} 1 & n \\ 0 & 1 \end{pmatrix}
    \]
%
    è un omomorfismo perché 
    \[
        \rho(n) \rho(m) = \begin{pmatrix} 1 & n \\ 0 & 1 \end{pmatrix} \begin{pmatrix} 1 & m \\ 0 & 1 \end{pmatrix} = \begin{pmatrix} 1 & n+m \\ 0 & 1 \end{pmatrix} = \rho(n+m)
    \]
} 
%
\nt{
    \[
        e_i = \binom{1}{0}, e_2 = \binom{0}{1} \quad       
    \]
}
%
\mlenma{}{
    Sia $G$ un gruppo finito. E sia $V$ una sottorappresentazione (su $C$) allora ogni sottorappresentazione $U \subseteq V$ ammette un complementare $W \subseteq V$ tale che $V = U \oplus W$ e $W$ è una sottorappresentazione. Quindi indecomponibile $\implies$ irriducibile
}
%
\pf{Dimostrazione}{
    Data una sottorappresentazione $U \subseteq V$ consideriamo $U^\perp = \{ v \in V | \langle v, u \rangle = 0 \forall u \in U \}$, dove $\langle \cdot, \cdot \rangle$ è un prodotto scalare hermitiano su $V$ (quello G inviariante), chiaramente $V = U \oplus U^\perp$ e $U^\perp$ è una sottorappresentazione perché $\forall v \in U^\perp, \forall g \in G, \forall u \in U$ abbiamo $\langle \rho(g)v, u \rangle = \langle v, \rho(g^{-1})u \rangle = 0$ perché $u \in U$ e $\rho(g^{-1})u \in U$
}
%
\dfn{}{
    $A$ anello con $1_A, (V, +)$ gruppo abeliano, è un A-modulo sinistro se $A\times V \to V$ tale che
    :
    \begin{itemize}
        \item $(a+b)v = av + bv$
        \item $a(v+w) = av + aw$
        \item $a(bv) = (ab)v$
        \item $1_A v = v$
    \end{itemize}
}
%
\dfn{$K[G]$}{Combinazioni linerari formali di elementi di $G$ con coefficienti in $K$ = $\left\{ \sum_{g \in G} a_g e_g | a_g \in K, supp(a_g) < \infty \right\}$
%
con prodotto $a_g, e_g$
}
%
%
% \end{document}
