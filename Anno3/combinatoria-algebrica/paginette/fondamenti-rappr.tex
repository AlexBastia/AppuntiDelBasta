% \begin{document}
\chapter{Fondamenti della Teoria delle Rappresentazioni}
\section{Struttura algebrica}
Cosa vuol dire combinatoria algebrica?

Vuol dire studiare strutture algebriche (gruppi, anelli, campi, etc.) attraverso la combinatoria.
\dfn{Struttura algebrica}{
    Si definisce struttura algebrica una coppia $(G, *)$ dove $G$ è un insieme e $*$ è una operazione binaria su $G$
}
\ex{Strutture algebriche}{
    \begin{itemize}
    \item  \textbf{Gruppo}: $(G, *)$ con $*$ binaria e associativa, $G$ con elemento neutro e ogni elemento ha un inverso
    \item \textbf{Anello}: $(G, +, \cdot)$ con $+$ e $\cdot$ binarie e associative, $G$ con elemento neutro per $+$ e ogni elemento ha un inverso per $+$
    \item \textbf{Campo}: $(G, +, \cdot)$ con $+$ e $\cdot$ binarie e associative, $G$ con elemento neutro per $+$ e ogni elemento ha un inverso per $+$
    \end{itemize}
}
%
Per "astratta", dal latino, "tirata fuori", si intende estrapolata dal suo contesto originale. La combinatoria quindi, vuole studiare le struttura algebriche e darne una rappresentazione più concreta.
%
%
\dfn{
  Rappresentazione di un gruppo 
}{
        Sia $G$ un gruppo, $K$ un campo e $V$ uno spazio vettoriale su $K$. si definisce una rappresentazione di $G$ su $V$ è un omomorfismo $\rho: G \to GL(V)$ tale che $\rho(g) \rho(h) = \rho(gh)$ per ogni $g, h \in G$.
}
%
\nt{
    In altre parole è un'azione (omomorfismo è un $G \to $ ) di $G$ su $V$ con immagine $GL(V)$ tramite isomorfismi 
}
%
\dfn{}{
    Diciamo che la rappresentazione è fedele se $\rho$ è iniettivo, ovvero se $\rho(g) = \rho(h) \implies g = h$
}
\nt{
  Ovvero il nucleo di $ \rho $ contiene solo l'elemento neutro ($ ker(\rho) = \{e\} $)
}
%
\dfn{}{
    Sia $(V, \rho)$ una rappresentazione di $G$ su $V$, diciamo che un sottospazio vettoriale $U\subseteq V$ è una sottorappresentazione di $\rho$ se $\rho(g)U\subseteq U$ per ogni $g\in G$
}
\nt{
    In altre parole $(U, \rho_{|U})$ è una rappresentazione di $G$ su $U$, dove $\rho_{|U}: G \to GL(U)$ è l'omomorfismo che mappa $g \to \rho(g)_{|U}$
}
%
\ex{}{
    \[
        G = \langle r | r^4 = i \rangle
    \]
    Può essere rappresentato tramite rotazione di $V = R^2$. Ovvero $\rho(r) = \begin{pmatrix} \cos(\frac{\pi}{2}) & -\sin(\frac{\pi}{2}) \\ \sin(\frac{\pi}{2}) & \cos(\frac{\pi}{2}) \end{pmatrix}$ ha sottorappresentazioni proprie? 
   
}
%
\dfn{R}{
    Una rappresenrazione è riducibile se ammette una sottorappresentazione propria (ovvero $U \neq \{0\}$ e $U \neq V$)
}
%
\ex{}{
    Sia $G$ il gruppo simmetrico $G=S_3, V=K^3$ rappresentazione matriciale. $\sigma \in S_3, (x_1, x_2, x_3) \in K^3$ 
    \[
        \rho(\sigma)(x_1, x_2, x_3) = (x_{\sigma(1)}, x_{\sigma(2)}, x_{\sigma(3)})
    \]
    Ad esempio 
    \[
        \rho((123))(7, -4, 5/2)= (-4, 5/2, 7)
    \]
}
%
%
\ex{Esercizio}{ Consideriamo i sottospazi $U=\{(t,t,t), t \in K\}$ e $W=\{(x_1, x_2, x_3),\in V | x_1 + x_2 + x_3 = 0\}$, mostrare che $U$ e $W$ sono sottorappresentazioni riducibili di $\rho$
}
\pf{Sol}{
    $V = U \oplus W$
}
%
\dfn{}{
    Una rappresentazione si dice indecomponibile se, ogni volta che $V = U \oplus W$ allora $U$ e $W$ sono sottorappresentazioni, abbiamo $U= V$ e $W=    \{0\}$ o viceversa
}
%
\mprop{}{
    Chiaramente irriducibile $\implies$ indecomponibile
}
%
%
\nt{
        vale il viceversa?   in generale n perché potrebbe esserci un completamentare
}
%
\ex{}{
    $V = C^2$
    $(Z, +)$
%
    \[
        \rho(n) = \begin{pmatrix} 1 & n \\ 0 & 1 \end{pmatrix}
    \]
%
    è un omomorfismo perché 
    \[
        \rho(n) \rho(m) = \begin{pmatrix} 1 & n \\ 0 & 1 \end{pmatrix} \begin{pmatrix} 1 & m \\ 0 & 1 \end{pmatrix} = \begin{pmatrix} 1 & n+m \\ 0 & 1 \end{pmatrix} = \rho(n+m)
    \]
} 
%
\nt{
    \[
        e_i = \binom{1}{0}, e_2 = \binom{0}{1} \quad       
    \]
}
%
\mlenma{}{
    Sia $G$ un gruppo finito. E sia $V$ una sottorappresentazione (su $C$) allora ogni sottorappresentazione $U \subseteq V$ ammette un complementare $W \subseteq V$ tale che $V = U \oplus W$ e $W$ è una sottorappresentazione. Quindi indecomponibile $\implies$ irriducibile
}
%
\pf{Dimostrazione}{
    Data una sottorappresentazione $U \subseteq V$ consideriamo $U^\perp = \{ v \in V | \langle v, u \rangle = 0 \forall u \in U \}$, dove $\langle \cdot, \cdot \rangle$ è un prodotto scalare hermitiano su $V$ (quello G inviariante), chiaramente $V = U \oplus U^\perp$ e $U^\perp$ è una sottorappresentazione perché $\forall v \in U^\perp, \forall g \in G, \forall u \in U$ abbiamo $\langle \rho(g)v, u \rangle = \langle v, \rho(g^{-1})u \rangle = 0$ perché $u \in U$ e $\rho(g^{-1})u \in U$
}
%
\dfn{}{
    $A$ anello con $1_A, (V, +)$ gruppo abeliano, è un A-modulo sinistro se $A\times V \to V$ tale che
    :
    \begin{itemize}
        \item $(a+b)v = av + bv$
        \item $a(v+w) = av + aw$
        \item $a(bv) = (ab)v$
        \item $1_A v = v$
    \end{itemize}
}
%
\dfn{$K[G]$}{Combinazioni linerari formali di elementi di $G$ con coefficienti in $K$ = $\left\{ \sum_{g \in G} a_g e_g | a_g \in K, supp(a_g) < \infty \right\}$
%
con prodotto $a_g, e_g$
}
%
%
% \end{document}
