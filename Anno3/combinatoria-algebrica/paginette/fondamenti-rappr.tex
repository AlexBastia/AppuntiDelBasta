% \begin{document}
\chapter{Fondamenti della Teoria delle Rappresentazioni}
Questo corso si colloca all'intersezione tra \textbf{Algebra} (gruppi, anelli, campi, spazi vettoriali) e \textbf{Combinatoria} (conteggio). Lo scopo è rendere gli oggetti algebrici \textit{astratti} più \textit{concreti}, studiandoli attraverso le loro azioni lineari su spazi vettoriali.

L'idea di fondo è semplice: invece di studiare un gruppo astratto $G$ direttamente, lo "rappresentiamo" tramite matrici invertibili, che sono oggetti molto più maneggevoli.

\section{Rappresentazioni e Sottorappresentazioni}
Una \textbf{struttura algebrica astratta} — ad esempio un gruppo — è definita da un insieme e da operazioni che soddisfano degli assiomi.
\dfn{Struttura algebrica}{
    Si definisce struttura algebrica una coppia $(G, *)$ dove $G$ è un insieme e $*$ è una operazione binaria su $G$
}
\ex{Strutture algebriche}{
    \begin{itemize}
    \item  \textbf{Gruppo}: $(G, *)$ con $*$ binaria e associativa, $G$ con elemento neutro e ogni elemento ha un inverso
    \item \textbf{Anello}: $(G, +, \cdot)$ con $+$ e $\cdot$ binarie e associative, $G$ con elemento neutro per $+$ e ogni elemento ha un inverso per $+$
    \item \textbf{Campo}: $(G, +, \cdot)$ con $+$ e $\cdot$ binarie e associative, $G$ con elemento neutro per $+$ e ogni elemento ha un inverso per $+$
    \end{itemize}
}
%
Per "astratta", dal latino, "tirata fuori", si intende estrapolata dal suo contesto originale. La combinatoria quindi, vuole studiare le struttura algebriche e darne una rappresentazione più concreta.

Questa struttura, infatti, può sembrare "sospesa nel vuoto", decontestualizzata. L'idea delle rappresentazioni è di portarla \textbf{dentro} uno spazio concreto, dove possiamo agire con strumenti dell'algebra lineare.

Ad esempio, un gruppo può agire su uno spazio vettoriale tramite trasformazioni lineari. Questo lo "rappresenta" come un gruppo di matrici.
\dfn{
  Rappresentazione di un gruppo 
}{
        Sia $G$ un gruppo, $K$ un campo e $V$ uno spazio vettoriale su $K$. si definisce una rappresentazione di $G$ su $V$ è un omomorfismo 
        $$\rho: G \to GL(V)$$ 
}
\nt{
  La rappresentazione e' un omomorfismo, quindi:
  \[ 
   \forall g, h \in G.\ \rho(g) \rho(h) = \rho(gh)
  \]
}
Ricordiamo che $\mathrm{GL}(V)$ è il gruppo degli isomorfismi lineari di $V$ in sé (automorfismi lineari, o equivalentemente matrici invertibili se $V = K^n$).
\nt{
  In altre parole, $\rho$ è un'azione di $G$ su $V$ che non è solo per biiezioni (come in $\mathrm{Sym}(V)$), ma per \textbf{trasformazioni lineari invertibili}.
}

\dfn{Rappresentazione fedele}{
    Diciamo che la rappresentazione è fedele se $\rho$ è iniettiva, ovvero se $\rho(g) = \rho(h) \implies g = h$. 
}
\nt{
  Ovvero il nucleo di $ \rho $ contiene solo l'elemento neutro ($ ker(\rho) = \{e\} $) e $G$ si immerge come sottogruppo di $\mathrm{GL}(V)$.
}

\ex{Gruppo di Klein}{
  Consideriamo il gruppo di Klein $G = \{a, b, c, d\}$ con tavola di Cayley:

$$\begin{array}{c|cccc}
* & a & b & c & d \\
\hline
a & c & d & a & b \\
b & d & c & b & a \\
c & a & b & c & d \\
d & b & a & d & c
\end{array}$$

L'elemento neutro è $c$ (la riga $c$ è l'identità). Si tratta di $\mathbb{Z}/2\mathbb{Z} \times \mathbb{Z}/2\mathbb{Z}$, ogni elemento ha ordine 2.

Una rappresentazione su $V = \mathbb{R}^2$ è data da:

$$\rho(c) = \begin{pmatrix}1 & 0 \\ 0 & 1\end{pmatrix}, \quad \rho(a) = \begin{pmatrix}-1 & 0 \\ 0 & 1\end{pmatrix}, \quad \rho(b) = \begin{pmatrix}1 & 0 \\ 0 & -1\end{pmatrix}, \quad \rho(d) = \begin{pmatrix}-1 & 0 \\ 0 & -1\end{pmatrix}$$

Si può verificare che è un omomorfismo (ad es. $\rho(a)\rho(b) = \rho(d)$, che corrisponde a $a * b = d$ nella tavola).

Rispetto alla decomposizione $V = \langle e_1 \rangle \oplus \langle e_2 \rangle$, le due rette coordinate sono \textbf{stabili} sotto l'azione di $G$: $\rho(g) \cdot e_i$ è sempre nella direzione di $e_i$. Al contrario, la decomposizione $V = \langle(1,1)\rangle \oplus \langle(1,-1)\rangle$ \textbf{non è stabile}, perché le basi non vengono mandate in sé stesse dagli elementi del gruppo.
}


\dfn{Sottorappresentazione}{
  Sia $(V, \rho)$ una rappresentazione di $G$ su $V$, diciamo che un sottospazio vettoriale $U\subseteq V$ è una sottorappresentazione di $\rho$ se $\rho(g)U\subseteq U$ per ogni $g\in G$.
}
\nt{
    In altre parole $(U, \rho_{|U})$ è una rappresentazione di $G$ su $U$, dove $\rho_{|U}: G \to GL(U)$ è l'omomorfismo che mappa $g \to \rho(g)_{|U}$
}

\subsection{Rappresentazioni Riducibili e Decomponibili}
\dfn{Rappresentazione Irriducibile}{
Una rappresentazione $(\rho, V)$ di un gruppo $G$ si dice \textbf{irriducibile} se gli unici sottospazi di $V$ che siano $G$-invarianti (sottorappresentazioni) sono i sottospazi banali $\{0\}$ e $V$ stesso. 
In caso contrario, ovvero se esiste un sottospazio proprio $0 < U < V$ che e' una sottorappresentazione di $ \rho $, la rappresentazione si dice \textbf{riducibile}.
}

\ex{Rappresentazione di $C_4$ e Dipendenza dal Campo $K$}{
Si consideri il gruppo ciclico di ordine 4, $G = C_4 = \langle z \mid z^4 = 1 \rangle$, e la sua rappresentazione naturale su $V = \mathbb{R}^2$ definita mandando il generatore $z$ nella matrice di rotazione di $\pi/2$:
$$ \rho(z) = \begin{pmatrix} 0 & -1 \\ 1 & 0 \end{pmatrix} $$
L'obiettivo è analizzare la riducibilità di $\rho$ al variare del campo degli scalari $K$.

\textbf{1. Analisi sul campo reale ($K = \mathbb{R}$):}
Per determinare se esistono sottorappresentazioni proprie, cerchiamo sottospazi stabili di dimensione 1, il che equivale a cercare gli autovalori reali di $\rho(z)$. Il polinomio caratteristico è:
$$ p(\lambda) = \det(\rho(z) - \lambda I) = \det \begin{pmatrix} -\lambda & -1 \\ 1 & -\lambda \end{pmatrix} = \lambda^2 + 1 $$
Le radici di $p(\lambda)$ sono $\pm i$, le quali \textbf{non appartengono} a $\mathbb{R}$. Non esistendo autovalori reali, non esistono rette in $\mathbb{R}^2$ stabili sotto l'azione della rotazione.
\begin{center}
    \textit{Conclusione:} $\rho$ è \textbf{irriducibile} su $\mathbb{R}$.
\end{center}

\textbf{2. Analisi sul campo complesso ($K = \mathbb{C}$):}
Se estendiamo lo spazio vettoriale a $V_{\mathbb{C}} = \mathbb{C}^2$, gli autovalori $\lambda_1 = i$ e $\lambda_2 = -i$ sono ora ammissibili. Ad essi corrispondono i rispettivi autospazi (sottospazi di dimensione 1):
$$ V_i = \text{span}_{\mathbb{C}} \left\{ \begin{pmatrix} 1 \\ -i \end{pmatrix} \right\}, \quad V_{-i} = \text{span}_{\mathbb{C}} \left\{ \begin{pmatrix} 1 \\ i \end{pmatrix} \right\} $$
Questi sottospazi sono $G$-invarianti, poiché l'azione di $z$ (e delle sue potenze) si limita a moltiplicare i vettori per uno scalare ($\pm i$). Lo spazio si decompone nella somma diretta:
$$ V_{\mathbb{C}} = V_i \oplus V_{-i} $$
\begin{center}
    \textit{Conclusione:} $\rho$ è \textbf{riducibile} su $\mathbb{C}$.
\end{center}
}

\nt{
\textbf{Osservazioni sulla Riducibilità:}
\begin{itemize}
    \item Una rappresentazione di dimensione 1 è sempre irriducibile per motivi dimensionali (non esistono sottospazi propri non nulli).
    \item Il concetto di irriducibilità dipende dal campo $K$ (come visto nell'esempio delle rotazioni su $\mathbb{R}$ vs $\mathbb{C}$).
    \item Scomporre una rappresentazione riducibile in somma diretta di irriducibili è l'obiettivo principale del corso (completabilità garantita dal Teorema di Maschke).
\end{itemize}
}

\ex{Rappresentazione Naturale di $S_3$}{
Sia $G = S_3$ agente su $V = \mathbb{R}^3$ tramite permutazione delle coordinate:
$$ \sigma \cdot (x_1, x_2, x_3) = (x_{\sigma(1)}, x_{\sigma(2)}, x_{\sigma(3)}) $$
Questa rappresentazione è \textbf{riducibile} in quanto ammette i seguenti sottospazi $G$-invarianti propri:
\begin{enumerate}
    \item $U = \{ (t, t, t) \mid t \in \mathbb{R} \}$, sottospazio di dimensione 1 (retta diagonale). Poiché ogni permutazione scambia coordinate identiche, $U$ è puntualmente fisso.
    \item $W = \{ (x_1, x_2, x_3) \in \mathbb{R}^3 \mid x_1 + x_2 + x_3 = 0 \}$, sottospazio di dimensione 2 (piano iperortogonale a $U$). Poiché la somma è commutativa, permutare gli addendi non cambia il risultato zero.
\end{enumerate}
Si verifica facilmente che $U \cap W = \{0\}$, pertanto lo spazio si decompone nella somma diretta:
$$ V = U \oplus W $$
Dove $U$ è la rappresentazione banale e $W$ è la \textbf{rappresentazione standard} di $S_3$. Entrambe sono irriducibili.
}

\dfn{Rappresentazione Indecomponibile}{
Una rappresentazione $(\rho, V)$ si dice \textbf{indecomponibile} se non può essere scritta come somma diretta di due sottorappresentazioni proprie non banali. Ovvero, se $V = U \oplus W$ con $U, W$ sottorappresentazioni, allora necessariamente $U=0$ oppure $W=0$.
}

\nt{
\textbf{Irriducibile vs Indecomponibile:}
\begin{itemize}
    \item Ogni rappresentazione \textbf{irriducibile} è banalmente \textbf{indecomponibile} (se non ha sottospazi stabili, a maggior ragione non può essere somma diretta di sottospazi stabili).
    \item Il viceversa non è sempre vero: esistono rappresentazioni che possiedono sottospazi stabili (sono riducibili) ma che non ammettono un complemento stabile (sono indecomponibili).
    \item \textbf{Nel Modulo 2:} Grazie al Teorema di Maschke, lavorando su un campo di caratteristica 0 (come $\mathbb{C}$) e con gruppi finiti, ogni sottorappresentazione ammette un complemento stabile. Di conseguenza, in questo contesto i concetti di irriducibile e indecomponibile \textbf{coincidono}.
\end{itemize}
}

\ex{Blocco di Jordan}{
  Sia $G = (\mathbb{Z}, +)$, $V = \mathbb{C}^2$, e definiamo

$$\rho : \mathbb{Z} \longrightarrow \mathrm{GL}(\mathbb{C}^2), \qquad n \longmapsto \begin{pmatrix}1 & n \\ 0 & 1\end{pmatrix}.$$

Questo è un omomorfismo perché $\begin{pmatrix}1&n\\0&1\end{pmatrix}\begin{pmatrix}1&m\\0&1\end{pmatrix} = \begin{pmatrix}1&n+m\\0&1\end{pmatrix}$.

  Il sottospazio $U = \langle e_1 \rangle = \langle(1,0)\rangle$ è una sottorappresentazione propria (poiché $\rho(n) e_1 = e_1 \in U$ per ogni $n$), quindi $V$ \textit{non è irriducibile}.

Tuttavia, notiamo che per ogni $(x, y)$ con $y \neq 0$:

$$\rho(n)\begin{pmatrix}x\\y\end{pmatrix} = \begin{pmatrix}x + ny \\ y\end{pmatrix}$$

  che è linearmente indipendente da $(x,y)$ per $n$ generico. Quindi \textbf{non esiste} alcun $W \subseteq V$ con $V = U \oplus W$ stabile, e la rappresentazione è \textbf{indecomponibile}.

Questo è il tipico blocco di Jordan di dimensione 2. La situazione si verifica perché $G = \mathbb{Z}$ è un gruppo infinito. Vedremo che per gruppi \textbf{finiti} con $\mathrm{char}(K) = 0$ la situazione è radicalmente diversa.
}

\subsection{Teorema di Maschke}
Per procedere serve costruire un prodotto scalare $ G $-invariante:
\\
\textbf{Costruzione}: Sia $(\rho, V)$ una rappresentazione di un gruppo finito $G$ su uno spazio vettoriale complesso $V$ ($ K = \mathbb{C} $). Dato un prodotto hermitiano arbitrario $( \cdot, \cdot )$ su $V$, definiamo:
$$ \langle v, u \rangle_G := \frac{1}{|G|} \sum_{g \in G} (\rho(g)v, \rho(g)u) $$
per ogni $v, u \in V$.
\\
Si tratta della media sui trasformati di $(v,u)$ lungo tutto il gruppo. Il fatto che $G$ sia finito e $\mathrm{char}(K) = 0$ (quindi $|G| \neq 0$ in $K$) garantisce che la divisione abbia senso.

\mprop{Esistenza di un Prodotto Hermitiano $G$-invariante}{
Sia $(\rho, V)$ una rappresentazione di un gruppo finito $G$ su uno spazio vettoriale complesso $V$. Esiste sempre su $V$ un prodotto hermitiano $\langle \cdot, \cdot \rangle_G$ che sia $G$-invariante, ovvero tale che:
$$ \langle \rho(g)v, \rho(g)u \rangle_G = \langle v, u \rangle_G \quad \forall g \in G, \forall v, u \in V $$
}

\pf{Dimostrazione}{
Sia $H(v, u)$ un prodotto hermitiano arbitrario su $V$ (la cui esistenza è garantita dalla struttura di spazio vettoriale complesso). Consideriamo il prodotto scalare definito prima:
$$ \langle v, u \rangle_G := \frac{1}{|G|} \sum_{x \in G} (\rho(x)v, \rho(x)u) $$
Per verificare l'invarianza, applichiamo un elemento generico $h \in G$:
$$ \langle \rho(h)v, \rho(h)u \rangle_G = \frac{1}{|G|} \sum_{x \in G} (\rho(x)\rho(h)v, \rho(x)\rho(h)u) $$
Poiché $\rho$ è un omomorfismo, $\rho(x)\rho(h) = \rho(xh)$. Sostituendo:
$$ \langle \rho(h)v, \rho(h)u \rangle_G = \frac{1}{|G|} \sum_{x \in G} (\rho(xh)v, \rho(xh)u) $$
Al variare di $x$ in $G$, l'elemento $xh$ percorre tutti gli elementi del gruppo esattamente una volta (la traslazione a destra è una permutazione di $G$). Possiamo quindi effettuare il cambio di variabile $k = xh$, ottenendo:
$$ \langle \rho(h)v, \rho(h)u \rangle_G = \frac{1}{|G|} \sum_{k \in G} (\rho(k)v, \rho(k)u) = \langle v, u \rangle_G $$
Questo dimostra che il prodotto è $G$-invariante. Le proprietà di linearità, simmetria hermitiana e positività di $\langle \cdot, \cdot \rangle_G$ discendono direttamente dalle proprietà del prodotto hermitiano costruito prima.
}

\thm{Teorema di Maschke (Versione Sintetica)}{
Sia $G$ un gruppo finito e $(\rho, V)$ una rappresentazione di $G$ su un campo $K$ (con $\text{char}(K) \nmid |G|$). Allora ogni sottorappresentazione $U \subseteq V$ ammette un complemento $G$-invariante $W$, tale che 
$$V = U \oplus W$$
}

\cor{}{
Per $G$ finito e $\mathrm{char}(K) = 0$: indecomponibile $\Longleftrightarrow$ irriducibile.
}

\pf{Dimostrazione}{
Data una sottorappresentazione $U \subseteq V$, prendiamo il complemento ortogonale rispetto al prodotto $G$-invariante costruito prima:

$$U^\perp = \{v \in V \mid \langle v, u \rangle = 0 \;\forall\, u \in U\}.$$

Allora $V = U \oplus U^\perp$ come spazi vettoriali (proprietà standard dei prodotti hermitiani). È sufficiente mostrare che $U^\perp$ è $G$-stabile. Siano $v \in U^\perp$ e $u \in U$:

$$\langle \rho(g)v,\, u \rangle = \langle \rho(g)^{-1}\rho(g)v,\, \rho(g)^{-1}u \rangle = \langle v,\, \rho(g)^{-1}u \rangle = 0$$

dove l'ultimo passaggio usa che $\rho(g)^{-1}u \in U$ (perché $U$ è $G$-stabile) e $v \in U^\perp$.
}

\section{Algebra di Gruppo}
Vogliamo ora riformulare tutta la teoria in un linguaggio più algebrico: guardare gli elementi di $G$ come "scalari che agiscono su $V$".

\dfn{A-modulo sinistro}{
 Sia $A$ un anello con unità $1_A$ e $(V, +)$ un gruppo abeliano. $V$ è un $A$-modulo sinistro se esiste un'operazione $A \times V \to V$ tale che:
 \begin{itemize}
 \item $a(v + u) = av + au$ 
 \item $(a + b)v = av + bv$
 \item $(ab)v = a(bv)$
 \item $1_A \cdot v = v$
 \end{itemize}
}
Questa è esattamente la definizione di spazio vettoriale, ma con $A$ anello al posto di campo.

\dfn{Algebra di gruppo}{
  L'\textbf{algebra di gruppo} $K[G]$ è l'insieme delle combinazioni lineari formali degli elementi di $G$ a coefficienti in $K$:

$$K[G] = \left\{\sum_{g \in G} a_g \cdot g \;\bigg|\; a_g \in K\right\}.$$

Gli elementi di $G$ formano una $K$-base di $K[G]$. Il prodotto in $K[G]$ è definito estendendo per bilinearità il prodotto in $G$:

$$(a_{g_1} \cdot g_1)(a_{g_2} \cdot g_2) := (a_{g_1} \cdot a_{g_2}) \cdot (g_1 g_2)$$

dove il primo prodotto è in $K$ e il secondo è in $G$. Questo rende $K[G]$ un \textbf{anello} (non commutativo in generale).
}

\mprop{$ K[G] $-modulo sinistro come rappresentazione di G}{
Le seguenti condizioni sono equivalenti:
\begin{enumerate}
\item $V$ è una rappresentazione di $G$ (cioè esiste $\rho: G \to \mathrm{GL}(V)$).
\item $V$ è un $K[G]$-modulo sinistro.
\end{enumerate}
}

\pf{Dimostrazione}{
$(2) \Rightarrow (1)$: l'inclusione $G \hookrightarrow K[G]$ (ogni $g$ è un elemento di base) restringe l'azione di $K[G]$ su $V$ a un'azione di $G$ per isomorfismi lineari, cioè fornisce $\rho: G \to \mathrm{GL}(V)$.

$(1) \Rightarrow (2)$: si estende l'azione per $K$-linearità:

$$\left(\sum_{g \in G} a_g \cdot g\right) \cdot v := \sum_{g \in G} a_g \cdot \rho(g)(v)$$

e si verifica che questo soddisfa gli assiomi di modulo.
}
% \end{document}
