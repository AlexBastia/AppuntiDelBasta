% \begin{document}
\chapter{Fondamenti della Teoria delle Rappresentazioni}
\section{Struttura algebrica}
Cosa vuol dire combinatoria algebrica?

Vuol dire studiare strutture algebriche (gruppi, anelli, campi, etc.) attraverso la combinatoria.
\dfn{Struttura algebrica}{
    Si definisce struttura algebrica una coppia $(G, *)$ dove $G$ è un insieme e $*$ è una operazione binaria su $G$
}
\ex{Strutture algebriche}{
    \begin{itemize}
    \item  \textbf{Gruppo}: $(G, *)$ con $*$ binaria e associativa, $G$ con elemento neutro e ogni elemento ha un inverso
    \item \textbf{Anello}: $(G, +, \cdot)$ con $+$ e $\cdot$ binarie e associative, $G$ con elemento neutro per $+$ e ogni elemento ha un inverso per $+$
    \item \textbf{Campo}: $(G, +, \cdot)$ con $+$ e $\cdot$ binarie e associative, $G$ con elemento neutro per $+$ e ogni elemento ha un inverso per $+$
    \end{itemize}
}
%
Per "astratta", dal latino, "tirata fuori", si intende estrapolata dal suo contesto originale. La combinatoria quindi, vuole studiare le struttura algebriche e darne una rappresentazione più concreta.
%
%
\dfn{
  Rappresentazione di un gruppo 
}{
        Sia $G$ un gruppo, $K$ un campo e $V$ uno spazio vettoriale su $K$. si definisce una rappresentazione di $G$ su $V$ è un omomorfismo $\rho: G \to GL(V)$ tale che $\rho(g) \rho(h) = \rho(gh)$ per ogni $g, h \in G$.
}
%
\nt{
    In altre parole è un'azione (omomorfismo è un $G \to $ ) di $G$ su $V$ con immagine $GL(V)$ tramite isomorfismi 
}
%
\dfn{}{
    Diciamo che la rappresentazione è fedele se $\rho$ è iniettivo, ovvero se $\rho(g) = \rho(h) \implies g = h$
}
\nt{
  Ovvero il nucleo di $ \rho $ contiene solo l'elemento neutro ($ ker(\rho) = \{e\} $)
}
%
\dfn{}{
    Sia $(V, \rho)$ una rappresentazione di $G$ su $V$, diciamo che un sottospazio vettoriale $U\subseteq V$ è una sottorappresentazione di $\rho$ se $\rho(g)U\subseteq U$ per ogni $g\in G$
}
\nt{
    In altre parole $(U, \rho_{|U})$ è una rappresentazione di $G$ su $U$, dove $\rho_{|U}: G \to GL(U)$ è l'omomorfismo che mappa $g \to \rho(g)_{|U}$
}
%
\ex{Rappresentazione di $C_4$ e Dipendenza dal Campo $K$}{
Si consideri il gruppo ciclico di ordine 4, $G = C_4 = \langle z \mid z^4 = 1 \rangle$, e la sua rappresentazione naturale su $V = \mathbb{R}^2$ definita mandando il generatore $z$ nella matrice di rotazione di $\pi/2$:
$$ \rho(z) = \begin{pmatrix} 0 & -1 \\ 1 & 0 \end{pmatrix} $$
L'obiettivo è analizzare la riducibilità di $\rho$ al variare del campo degli scalari $K$.

\textbf{1. Analisi sul campo reale ($K = \mathbb{R}$):}
Per determinare se esistono sottorappresentazioni proprie, cerchiamo sottospazi stabili di dimensione 1, il che equivale a cercare gli autovalori reali di $\rho(z)$. Il polinomio caratteristico è:
$$ p(\lambda) = \det(\rho(z) - \lambda I) = \det \begin{pmatrix} -\lambda & -1 \\ 1 & -\lambda \end{pmatrix} = \lambda^2 + 1 $$
Le radici di $p(\lambda)$ sono $\pm i$, le quali \textbf{non appartengono} a $\mathbb{R}$. Non esistendo autovalori reali, non esistono rette in $\mathbb{R}^2$ stabili sotto l'azione della rotazione.
\begin{center}
    \textit{Conclusione:} $\rho$ è \textbf{irriducibile} su $\mathbb{R}$.
\end{center}

\textbf{2. Analisi sul campo complesso ($K = \mathbb{C}$):}
Se estendiamo lo spazio vettoriale a $V_{\mathbb{C}} = \mathbb{C}^2$, gli autovalori $\lambda_1 = i$ e $\lambda_2 = -i$ sono ora ammissibili. Ad essi corrispondono i rispettivi autospazi (sottospazi di dimensione 1):
$$ V_i = \text{span}_{\mathbb{C}} \left\{ \begin{pmatrix} 1 \\ -i \end{pmatrix} \right\}, \quad V_{-i} = \text{span}_{\mathbb{C}} \left\{ \begin{pmatrix} 1 \\ i \end{pmatrix} \right\} $$
Questi sottospazi sono $G$-invarianti, poiché l'azione di $z$ (e delle sue potenze) si limita a moltiplicare i vettori per uno scalare ($\pm i$). Lo spazio si decompone nella somma diretta:
$$ V_{\mathbb{C}} = V_i \oplus V_{-i} $$
\begin{center}
    \textit{Conclusione:} $\rho$ è \textbf{riducibile} su $\mathbb{C}$.
\end{center}
}
%
\dfn{Rappresentazione Irriducibile}{
Una rappresentazione $(\rho, V)$ di un gruppo $G$ si dice \textbf{irriducibile} se gli unici sottospazi di $V$ che siano $G$-invarianti (sottorappresentazioni) sono i sottospazi banali $\{0\}$ e $V$ stesso. 
In caso contrario, ovvero se esiste un sottospazio proprio $0 < U < V$ tale che $\rho(g)u \in U$ per ogni $u \in U$ e $g \in G$, la rappresentazione si dice \textbf{riducibile}.
}

\nt{
\textbf{Osservazioni sulla Riducibilità:}
\begin{itemize}
    \item Una rappresentazione di dimensione 1 è sempre irriducibile per motivi dimensionali (non esistono sottospazi propri non nulli).
    \item Il concetto di irriducibilità dipende dal campo $K$ (come visto nell'esempio delle rotazioni su $\mathbb{R}$ vs $\mathbb{C}$).
    \item Scomporre una rappresentazione riducibile in somma diretta di irriducibili è l'obiettivo principale del corso (completabilità garantita dal Teorema di Maschke).
\end{itemize}
}

\ex{Rappresentazione Naturale di $S_3$}{
Sia $G = S_3$ agente su $V = \mathbb{R}^3$ tramite permutazione delle coordinate:
$$ \sigma \cdot (x_1, x_2, x_3) = (x_{\sigma^{-1}(1)}, x_{\sigma^{-1}(2)}, x_{\sigma^{-1}(3)}) $$
Questa rappresentazione è \textbf{riducibile} in quanto ammette i seguenti sottospazi $G$-invarianti propri:
\begin{enumerate}
    \item $U = \{ (t, t, t) \mid t \in \mathbb{R} \}$, sottospazio di dimensione 1 (retta diagonale). Poiché ogni permutazione scambia coordinate identiche, $U$ è puntualmente fisso.
    \item $W = \{ (x_1, x_2, x_3) \in \mathbb{R}^3 \mid x_1 + x_2 + x_3 = 0 \}$, sottospazio di dimensione 2 (piano iperortogonale a $U$). Poiché la somma è commutativa, permutare gli addendi non cambia il risultato zero.
\end{enumerate}
Si verifica facilmente che $U \cap W = \{0\}$, pertanto lo spazio si decompone nella somma diretta:
$$ V = U \oplus W $$
Dove $U$ è la rappresentazione banale e $W$ è la \textbf{rappresentazione standard} di $S_3$. Entrambe sono irriducibili.
}

\dfn{Prodotto Hermitiano $G$-invariante}{
Sia $(\rho, V)$ una rappresentazione di un gruppo finito $G$ su uno spazio vettoriale complesso $V$. Dato un prodotto hermitiano arbitrario $\langle \cdot, \cdot \rangle$ su $V$, definiamo il \textbf{prodotto hermitiano $G$-mediato} (o $G$-invariante) come:
$$ \langle v, u \rangle_G := \frac{1}{|G|} \sum_{g \in G} \langle \rho(g)v, \rho(g)u \rangle $$
per ogni $v, u \in V$. Tale prodotto soddisfa la condizione di invarianza:
$$ \langle \rho(h)v, \rho(h)u \rangle_G = \langle v, u \rangle_G \quad \forall h \in G $$
}

\mprop{Esistenza di un Prodotto Hermitiano $G$-invariante}{
Sia $(\rho, V)$ una rappresentazione di un gruppo finito $G$ su uno spazio vettoriale complesso $V$. Esiste sempre su $V$ un prodotto hermitiano $\langle \cdot, \cdot \rangle_G$ che sia $G$-invariante, ovvero tale che:
$$ \langle \rho(g)v, \rho(g)u \rangle_G = \langle v, u \rangle_G \quad \forall g \in G, \forall v, u \in V $$
}

\pf{Dimostrazione}{
Sia $H(v, u)$ un prodotto hermitiano arbitrario su $V$ (la cui esistenza è garantita dalla struttura di spazio vettoriale complesso). Definiamo il nuovo prodotto facendo la media sui trasformati degli elementi tramite il gruppo:
$$ \langle v, u \rangle_G := \frac{1}{|G|} \sum_{x \in G} H(\rho(x)v, \rho(x)u) $$
Per verificare l'invarianza, applichiamo un elemento generico $h \in G$:
$$ \langle \rho(h)v, \rho(h)u \rangle_G = \frac{1}{|G|} \sum_{x \in G} H(\rho(x)\rho(h)v, \rho(x)\rho(h)u) $$
Poiché $\rho$ è un omomorfismo, $\rho(x)\rho(h) = \rho(xh)$. Sostituendo:
$$ \langle \rho(h)v, \rho(h)u \rangle_G = \frac{1}{|G|} \sum_{x \in G} H(\rho(xh)v, \rho(xh)u) $$
Al variare di $x$ in $G$, l'elemento $xh$ percorre tutti gli elementi del gruppo esattamente una volta (la traslazione a destra è una permutazione di $G$). Possiamo quindi effettuare il cambio di variabile $k = xh$, ottenendo:
$$ \langle \rho(h)v, \rho(h)u \rangle_G = \frac{1}{|G|} \sum_{k \in G} H(\rho(k)v, \rho(k)u) = \langle v, u \rangle_G $$
Questo dimostra che il prodotto è $G$-invariante. Le proprietà di linearità, simmetria hermitiana e positività di $\langle \cdot, \cdot \rangle_G$ discendono direttamente dalle proprietà di $H$.
}



\dfn{Rappresentazione Indecomponibile}{
Una rappresentazione $(\rho, V)$ si dice \textbf{indecomponibile} se non può essere scritta come somma diretta di due sottorappresentazioni proprie non banali. Ovvero, se $V = U \oplus W$ con $U, W$ sottorappresentazioni, allora necessariamente $U=0$ oppure $W=0$.
}

\nt{
\textbf{Irriducibile vs Indecomponibile:}
\begin{itemize}
    \item Ogni rappresentazione \textbf{irriducibile} è banalmente \textbf{indecomponibile} (se non ha sottospazi stabili, a maggior ragione non può essere somma diretta di sottospazi stabili).
    \item Il viceversa non è sempre vero: esistono rappresentazioni che possiedono sottospazi stabili (sono riducibili) ma che non ammettono un complemento stabile (sono indecomponibili).
    \item \textbf{Nel Modulo 2:} Grazie al Teorema di Maschke, lavorando su un campo di caratteristica 0 (come $\mathbb{C}$) e con gruppi finiti, ogni sottorappresentazione ammette un complemento stabile. Di conseguenza, in questo contesto i concetti di irriducibile e indecomponibile \textbf{coincidono}.
\end{itemize}
}

\mprop{Teorema di Maschke (Versione Sintetica)}{
Sia $G$ un gruppo finito e $(\rho, V)$ una rappresentazione di $G$ su un campo $K$ (con $\text{char}(K) \nmid |G|$). Allora ogni sottorappresentazione $U \subseteq V$ ammette un complemento $G$-invariante $W$, tale che $V = U \oplus W$.
}

\pf{Dimostrazione}{
L'esistenza di $W$ è garantita dalla costruzione di un proiettore $G$-equivariante ottenuto tramite il \textbf{trucco della media}. Sia $\pi: V \to U$ una proiezione lineare arbitraria. Definiamo:
$$ \tilde{\pi} = \frac{1}{|G|} \sum_{g \in G} \rho(g) \pi \rho(g)^{-1} $$
Si dimostra che $\tilde{\pi}$ è un morfismo di rappresentazioni e che $\ker(\tilde{\pi})$ è il complemento stabile $W$ cercato.
}

\ex{}{
    $V = C^2$
    $(Z, +)$
%
    \[
        \rho(n) = \begin{pmatrix} 1 & n \\ 0 & 1 \end{pmatrix}
    \]
%
    è un omomorfismo perché 
    \[
        \rho(n) \rho(m) = \begin{pmatrix} 1 & n \\ 0 & 1 \end{pmatrix} \begin{pmatrix} 1 & m \\ 0 & 1 \end{pmatrix} = \begin{pmatrix} 1 & n+m \\ 0 & 1 \end{pmatrix} = \rho(n+m)
    \]
} 
%
\nt{
    \[
        e_i = \binom{1}{0}, e_2 = \binom{0}{1} \quad       
    \]
}
%
\mlenma{}{
    Sia $G$ un gruppo finito. E sia $V$ una sottorappresentazione (su $C$) allora ogni sottorappresentazione $U \subseteq V$ ammette un complementare $W \subseteq V$ tale che $V = U \oplus W$ e $W$ è una sottorappresentazione. Quindi indecomponibile $\implies$ irriducibile
}
%
\pf{Dimostrazione}{
    Data una sottorappresentazione $U \subseteq V$ consideriamo $U^\perp = \{ v \in V | \langle v, u \rangle = 0 \forall u \in U \}$, dove $\langle \cdot, \cdot \rangle$ è un prodotto scalare hermitiano su $V$ (quello G inviariante), chiaramente $V = U \oplus U^\perp$ e $U^\perp$ è una sottorappresentazione perché $\forall v \in U^\perp, \forall g \in G, \forall u \in U$ abbiamo $\langle \rho(g)v, u \rangle = \langle v, \rho(g^{-1})u \rangle = 0$ perché $u \in U$ e $\rho(g^{-1})u \in U$
}
%
\dfn{}{
    $A$ anello con $1_A, (V, +)$ gruppo abeliano, è un A-modulo sinistro se $A\times V \to V$ tale che
    :
    \begin{itemize}
        \item $(a+b)v = av + bv$
        \item $a(v+w) = av + aw$
        \item $a(bv) = (ab)v$
        \item $1_A v = v$
    \end{itemize}
}
%
\dfn{$K[G]$}{Combinazioni linerari formali di elementi di $G$ con coefficienti in $K$ = $\left\{ \sum_{g \in G} a_g e_g | a_g \in K, supp(a_g) < \infty \right\}$
%
con prodotto $a_g, e_g$
}
%
%
% \end{document}
