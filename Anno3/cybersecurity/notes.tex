\documentclass{report}

\input{../LatexTemp/preamble.tex}
\input{../LatexTemp/macros}
\input{../LatexTemp/letterfonts}

\usepackage[utf8]{inputenc}
\usepackage{fontawesome}
\usepackage{amsmath}
\usepackage{algorithm}
\usepackage{algpseudocode}

\newcommand{\dperp}{\mathrel{\bot\!\!\!\bot}}
\renewcommand{\P}{\mathbb{P}}
\newcommand{\p}{p_{(X,Y)}(x,y)}
\renewcommand{\ss}{\mathcal{S}_X\times\mathcal{S}_Y}
\newcommand{\xy}{(X,Y)}



\usetikzlibrary{arrows.meta, positioning, automata}


\setlength{\parindent}{0pt}

\title{\Huge{Fondamenti di Cybersecurity}\\Appunti}
\author{\huge{Giovanni "Qua' Qua' dancer" Palma} 
\\e\\\huge{Alex "Morbidelli WhatsApp" Basta}}
\date{}
\pagenumbering{gobble}

\begin{document}

\maketitle
\newpage% or \cleardoublepage
% \pdfbookmark[<level>]{<title>}{<dest>}
\pdfbookmark[section]{\contentsname}{toc}
\tableofcontents

\pagebreak

Ci sara' una domanda sul lab

\ex{}{
  Quali opzioni ho per crackare una password?
}

% \begin{document}
\chapter{Key Exchange}
% \end{document}

% \begin{document}
\chapter{Modular Arithmetic}
% \end{document}

% \begin{document}
\chapter{Asymmetric Criptography}
% \end{document}

% \begin{document}
\chapter{IPsec and TLS}
% \end{document}

% \begin{document}
\chapter{Access Control}
% \end{document}

% \begin{document}
\chapter{Exploits and Patches}
% \end{document}


\end{document}
