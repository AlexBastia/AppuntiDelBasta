\chapter{Probabilistic approach}
\section{core idea}
we have two main points of views:
\begin{itemize}
    \item \textbf{traditional view}: we wanna to approximate a function $f:X\to X$
    \item \textbf{Probabilist view}: we wanna compute probablilities: $p: P( Y\mid  X)$
\end{itemize}
\subsection{Probs basics}
\subsubsection{Random variables}
A random variables $X$ represents an oyt come about which we're ncertain
\ex{Random variables}{
\begin{itemize}
    \item $X=$\texttt{true} if a randomly drawn stdent is male
    \item $X=$ first name of the student
    \item $X=$\texttt{true} if a randomly drawn stdent have the same birthday
\end{itemize}
}
 
Formal def:
\dfn{Probs variables}{
    the set $\Omega$ of the possible outcomes is called the sample space. It is said random variable a measurable function over $\Omega$:
    \begin{itemize}
        \item Discrete: $\Omega \to \{m,f\}$
        \item Continuos: $\Omega\to \mathbb{R}$
    \end{itemize}
}
 
\dfn{Probs def}{
    it is defined $P(X)$ is the fraction of times $X$ is true in repeated runs of the same experiment.
}

\nt{
    The definition requires that all samples 
}

Pay attention:
\wc{
    bad examples
}{
    Sample space, let $\Omega$ be a space made the possibile sum:
    \[
        \Omega = \{2,3,4,\dots, 12\}
    \]
    Problem: not all sums are equally likely! It should be:
    \[
        \begin{array}{c}
            P(sum = 2) = 1/11\\
            P(sum = 7) = 1/11
        \end{array}
    \]
    but in reality:
    \begin{itemize}
        \item Sum = 2: can only happen one way: $(1,1)$
        \item Sum = 7: can happen six ways:$(1,6), (2,5), (3,4), (4,3), (5,2), (6,1)$
    \end{itemize}

    so 
    \[
        P(sum = 2) \neq P(sum = 7)
    \]
}

A correct approach is 
\clm{correct approach}{}{
    Be $\Omega = {(1,1), (1,2), (1,3), ..., (6,5), (6,6)}$, where $|\Omega|=36 $outcomes

    each pair has equally probability = $\frac{1}{36}$
    
    Now here is a correctly computing:
    \[
        \begin{array}{c}
            P(sum = 2) = \frac{|{(1,1)}|}{36} = \frac{1}{36}\\
            P(sum = 7) = \frac{|{(1,6), (2,5), (3,4), (4,3), (5,2), (6,1)}|}{36} = \frac{6}{36}
        \end{array}
    \]
}

\subsubsection{The Axioms of Probability Theory}
These are the fundamental rules that make probability a "reasonable theory of uncertainty":

\ax{Axioms of probability theory}{
    \begin{align}
        &\text{(1) Non-negativity: } && 0 \leq P(A) \leq 1 \quad \text{for all events } A. \\
        &\text{(2) Normalization: } && P(\Omega) = 1. \\
        &\text{(3) Countable additivity: } && 
        \text{If } A_1, A_2, \dots \text{ are disjoint, then } 
        P\!\left(\bigcup_{i=1}^\infty A_i\right) = \sum_{i=1}^\infty P(A_i).
    \end{align}
}

Then:

\cor{consequences of the axioms}{
\begin{itemize}
    \item Monotonicity: If $A \subseteq B$, then $P(A) \le P(B)$
    \item Union rule (for two events): $P(A \cup B) = P(A) + P(B) - P(A \cap B)$
    \item $P(True) = 1$
    \item $P(False) = 0$
\end{itemize}
}

\begin{center}
    \includegraphics[width=5cm]{probs_ax.png}
\end{center}

\subsubsection{Derivied theorems}
\cor{Complement Rule
}{
    \[
        P(\lnot A) = 1- P(A)
    \]
}
\pf{Dm}{
\[
P(A \cup \neg A) = P(A) + P(\neg A) - P(A \cap \neg A)
\]

But: 
\[
P(A \cup \neg A) = P(\text{True}) = 1
\quad \text{and} \quad
P(A \cap \neg A) = P(\text{False}) = 0
\]

Therefore:
\[
1 = P(A) + P(\neg A) - 0
\quad \implies \quad
P(\neg A) = 1 - P(A) \qed
\]

}

\cor{Partition Rule}{
    \[
        P(A) = P(A \cap B) + P(A \cap \neg B)
    \]
}
\pf{Proof}{
    \[
\begin{aligned}
A &= A \cap (B \cup \neg B) &\text{[since $B \cup \neg B$ is always True]}\\
  &= (A \cap B) \cup (A \cap \neg B) &\text{[distributive law]}
\end{aligned}
\]

Hence,
\[
\begin{aligned}
P(A) &= P((A \cap B) \cup (A \cap \neg B)) \\
     &= P(A \cap B) + P(A \cap \neg B) - P((A \cap B) \cap (A \cap \neg B)) \\
     &= P(A \cap B) + P(A \cap \neg B) - P(\text{False}) \\
     &= P(A \cap B) + P(A \cap \neg B)
\end{aligned}
\]

}
\subsubsection{Multivalued Discrete Random Variables}
\dfn{
    k-value Discrete Random Variables
}{
    A random variable $A$ is \textit{$k$-valued discrete} if it takes exactly one value from 
    \[
    \{\nu_1, \nu_2, \dots, \nu_k\}.
    \]
}

\mprop{Key proprieties}{
    \begin{enumerate}
    \item \textbf{Mutual exclusivity:} For $i \neq j$,
    \[
        P(A = \nu_i \cap A = \nu_j) = 0
    \]

    \item \textbf{Exhaustiveness:}
    \[
        P(A = \nu_1 \cup A = \nu_2 \cup \dots \cup A = \nu_k) = 1
    \]
\end{enumerate}

}

\subsubsection{Conditional Probability}
\dfn{Conditional probs}{
    The Conditional probs of the event $A$ \textit{given} the event $B$ is defined as the quantity
    \begin{center}
        \begin{math}
            P(A\mid B) = \frac{P(A\cap B)}{P(B)}            
        \end{math}
    \end{center}
}

\cor{Cahin roule}{
    \[
        P(A\cap B) = P(B)P(A\mid  B)= P(A) P(B\mid A)
    \]
}
\subsubsection{Independent Events}
\dfn{Independent Events}{
    Events $A$ and $B$ are independent when:
    \begin{center}
        \begin{math}
            P(A\mid  B) = P(B)
        \end{math}
    \end{center}
}
(Meaning: B provides no information about A.)
\cor{consequences}{
    \begin{itemize}
        \item $P(A\cap b) = P(A)P(B)$ (from chail roule)
        \item $P(B|A) = P(B)$ (symmetry)
    \end{itemize}
}

\subsubsection{Bayes' Rule: The Heart of Probabilistic ML (ok chat... really?)}
\thm{Bayes's roule}{
    Now we have Bayes roule
    \begin{center}
        \begin{math}
            P(A\mid  B) = \frac{P(A)  P(B\mid A)}{P(B)}
        \end{math}
    \end{center}
}
\pf{Proof}{
    It's true by the chain roule that: $P(A \cap B) = P(B)  P(A\mid B)$. It's true also the reverse case $P(A \cap B) = P(B) · P(A\mid B)$.

    Since both expressions equal $P(A \mid B)$, they must equal each other:
    \[
        P(A)  P(B\mid A) = P(B)  P(A\mid B) 
    \]
    that it's equal to
    \[
        P(A\mid B) =\frac{[P(A)  P(B\mid A)]}{P(B)}
    \]
}
\ex{The trousers problem}{
    Setup:
    \begin{itemize}
        \item 60\% of students are boys, 40\% are girls
        \item girls wear in the same number skirt and trousers
        \item boys only wear trousers
    \end{itemize}
    If we see a student wearing trousers, what is the probability that is a girl?
}
\pf{Solution}{
    The probs a priori that a strudent is a girl is
    \[
        P(G) = \frac{2}{5}
    \]
    the probability that a student wears trousers is
    \[
        P(T) = \frac{1}{5} + \frac{3}{5} = \frac{4}{5}
    \]
    the probability that a student wear trousers, given that the student is a girl, is
    \[
        P(T\mid G) = 1/2
    \]
    So 
    \[
        P(G\mid T) =\frac{p(G)p(T\mid G)}{P(T)}=\frac{2/5\cdot 1/2}{4/5} = 1/4
    \]
}

\paragraph{Machine Learning Form}
\paragraph{Machine Learning Form}
For discrete $Y$ with values $\{y_1, y_2, \ldots, y_m\}$ and $X$ with values $\{x_1, x_2, \ldots, x_n\}$:
\[
P(Y = y_i \mid X = x_j) = \frac{P(Y = y_i) \cdot P(X = x_j \mid Y = y_i)}{P(X = x_j)}
\]

\textbf{Expanding the denominator:}
\begin{align*}
P(X = x_j) &= \sum_{i} P(X = x_j, Y = y_i) \quad \text{[sum over all $Y$ values]} \\
           &= \sum_{i} P(Y = y_i) \cdot P(X = x_j \mid Y = y_i) \quad \text{[chain rule]}
\end{align*}

\textbf{Complete Bayes' Rule:}
\[
P(Y = y_i \mid X = x_j) = \frac{P(Y = y_i) \cdot P(X = x_j \mid Y = y_i)}{\sum_{i} P(Y = y_i) \cdot P(X = x_j \mid Y = y_i)}
\]

\textbf{Terminology:}
\[
\underbrace{P(Y \mid X)}_{\text{posterior}} = \frac{\overbrace{P(X \mid Y)}^{\text{likelihood}} \cdot \overbrace{P(Y)}^{\text{prior}}}{\underbrace{P(X)}_{\text{marginal}}}
\]

\begin{itemize}
    \item \textbf{Posterior} $P(Y \mid X)$: What we want -- probability of $Y$ given observed $X$
    \item \textbf{Likelihood} $P(X \mid Y)$: How likely is $X$ if $Y$ is true?
    \item \textbf{Prior} $P(Y)$: What we believed before seeing $X$
    \item \textbf{Marginal} $P(X)$: Overall probability of observing $X$ (normalization constant)
\end{itemize}

\textbf{Alternative form:}
\[
\text{Posterior} = \frac{\text{Likelihood} \cdot \text{Prior}}{\text{Marginal Likelihood}}
\]
where:
\[
\text{Marginal} = \sum_{Y} P(X \mid Y) \cdot P(Y)
\]

The term ``marginal'' means we've \textbf{marginalized} (integrated/summed) over $Y$.