\chapter{Probabilistic approach}
\section{core idea}
we have two main points of views:
\begin{itemize}
    \item \textbf{traditional view}: we wanna to approximate a function $f:X\to X$
    \item \textbf{Probabilist view}: we wanna compute probablilities: $p: P( Y\| X)$
\end{itemize}
\subsection{Probs basics}
\subsubsection{Random variables}
A random variables $X$ represents an oyt come about which we're ncertain
\ex{Random variables}{
\begin{itemize}
    \item $X=$\texttt{true} if a randomly drawn stdent is male
    \item $X=$ first name of the student
    \item $X=$\texttt{true} if a randomly drawn stdent have the same birthday
\end{itemize}
}
 
Formal def:
\dfn{Probs variables}{
    the set $\Omega$ of the possible outcomes is called the sample space. It is said random variable a measurable function over $\Omega$:
    \begin{itemize}
        \item Discrete: $\Omega \to \{m,f\}$
        \item Continuos: $\Omega\to \mathbb{R}$
    \end{itemize}
}
 
\dfn{Probs def}{
    it is defined $P(X)$ is the fraction of times $X$ is true in repeated runs of the same experiment.
} 
