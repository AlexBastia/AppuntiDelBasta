%\begin{document}
\chapter{Maths for Quantum Mechanics}
\section{Why linear algebra?}
\subsection{Classical Physics}
Physical quantities:
\begin{itemize}
\item \textbf{Single-valued}
\item \textbf{Continuous}
\end{itemize}

So they can be mapped as a \textbf{continuous function}.

\subsection{Quantum}
In the quantum world, physical values can also be:
\begin{itemize}
\item \textbf{Probabilistic}
\item \textbf{Discrete}
\end{itemize}

So we can't use a continuous function.

\subsubsection{Probabilisticness}
We can represent each possible outcome of our value with a mathematical object $ M_i $, and we need to combine these giving each a specific probability.

\[
a_1M_1 \cdot a_2M_2 \cdot ... \cdot a_nM_n
\]
where $ a_i $ encodes the likelyhood of its associated value to occur, and $ '\cdot' $ is an operation that combines the values. Informally, we can see how this structure is similar to a \textbf{linear equation}, we'll come back to that. 

\subsubsection{Discreteness}
What mathematical object can we use to extract discrete values? Following our linear combination lead, we can think of representing physical quantities as \textbf{matrices} (linear operators), from which we can think to extract the possible discrete values.

\section{Kets and Wavefunctions}
\dfn{Vector Space}{
  A set of objects (called vectors) that satisfy the following conditions $ \forall u,v,w $ vectors and $ \forall a,b \in \mathbb{C} $:
  \begin{itemize}
  \item $ u+v $ is a vector
  \item $ au $ is a vector
  \item $ -u $ is a vector
  \item 0 vector exists
  \item $ 1u = u $
  \item $ u + (v + w) = (u + v) + w $
  \item $ u + v = v + u $
  \item $ a(bu) = (ab)u $
  \item $ a(u + v) = au + av $
  \item $ (a + b)u = au + bu $
  \end{itemize}
}
\nt{
  This definition doesn't mention arrows or lists of numbers or arrows, we can use any tipe of object that follows the structure.
}

Let's try to represent our particle as a vector in a vector space, a \textbf{quantum state} $ \psi $:
\dfn{Quantum State}{
  Vector that holds all the physical properties of a particle, which should all be extractable at any moment in time, as well as all the associated probabilities. Usually represented in bra-ket notation as:
  \[
    \ket{\psi}
  \]
}

We want our state to represent a linear combination of all possible outcomes of a measurement:
\[
  \ket{\psi} = a_1\ket{E_1} + ... + a_n\ket{E_n}
\]
Where $ E_1,...,E_n $ are all the possible values of the particle's energy, and the coefficients have something to do with the probability of the outcome. This is called a \textbf{superposition}.

But the state shoud hold values for \textbf{all} physical properties, so--looking at angular momentum as an example--we must also have that:
\[
\ket{\psi} = b_1\ket{L_1} + ... + b_n\ket{L_n}
\]
So the same state also encodes a different linear combination.

\subsubsection{Infinite sets}
It's possible for the set of discrete values of a property to be infinite.
Take for example the energy: we can theoretically keep adding more and more without a limit, making the set of all possible values still discrete, but infinite:

\[
  \ket{\psi} = \sum_{i = 1}^{\infty} a_iE_i
\]

\nt{
  There are still some problems that can arise by allowing infinity without proper constraints, as we'll explore in the next chapter.
}

Furthermore, some properties like position, have continuous values. These can't be represented with a normal linear sum.

Let's say that a particle's position can be in any point on the continuous line between $ a $ and $ b $. In this case we need a continuous sum, which can be performed with an integral:
\[
  \ket{\psi} = \int_{a}^{b} c(x)\ket{x} dx
\]
Note that instead of a list of coefficients for each value we now have a function $ c(x) $ that encodes probability for each continuous value. This function is called a \textbf{wavefunction}.

\section{Hilbert Space}
\section{Inner products}
\section{Dirac deltas and Wavefunction inner products}
\section{Bra-ket notation}
\section{Observables as operators}
\section{Calculating Probabilities}
\section{Hermitian operators}
\section{Commutator and the Uncertainty principle}
\section{Unitary operators}
\section{Classical Generators}
\section{Shrodinger equation}
\section{Momentum operator}

%\end{document}
