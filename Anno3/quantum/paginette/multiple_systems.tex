% \begin{document}
\chapter{Multiple Systems}
\section{Classical states}
Suppose we have just two systems:
\begin{itemize}
\item $ X $ is a system with classical state set $ \Sigma $
\item $ Y $ is a system with classical state set $ \Gamma $
\end{itemize}

we can place them side-by-side to obtain a single classical system.

\dfn{Compound classical system (pair)}{
  The system created by joining two classical states $ X, Y $:
  \[
  XY
  \]
}

\qs{}{
  What are the classical states of $ (X,Y) $?
}

The classical state set of $ (X,Y) $ is the \textit{Cartesian product} of the two sets:
\[
  \Sigma \times \Gamma = \{(a,b): a \in \Sigma and b \in \Gamma \}
\]

\nt{
  We're making the assumption that we know witch system is witch, and there's no confusion between them.
}

This generalizes well for $ n $ systems:
\dfn{Compound classical system}{
  Given $ n $ classical systems $ X_1,...,X_n $, the \textbf{compound system} formed by them is written as an $ n $-tuple:
  \[
    (X_1,...,X_n)
  \]
  or as a string:
  \[
  X_1...X_n
  \]
}
\mprop{}{
  The classical state set of $ X_1...X_n $ with states $ \Sigma_1,...,\Sigma_n $ is the set:
  \[
  \Sigma_1 \times ... \times \Sigma_n
  \]
}

\nt{
  As per convention, cartesian products of states are ordered \textit{lexicographically}, and significance decreases from left to right.
}

\section{Probabilistic states}
A probability is associated to each Cartesian product of the classical state sets of the individual systems.

\dfn{Statistical independence}{
  Given a probabilistic state of $ (X,Y) $, we say that $ X $ and $ Y $ are independent if, $ \forall a \in \Sigma, \forall b \in \Gamma $:
  \[
    Pr((X,Y) = (a,b)) = Pr(X=a)Pr(Y=b)
  \]
}
By expressing the state of $ (X,Y) $ as a vector:
\[
  |\pi\rangle = \sum_{(a,b)\in \Sigma \times \Gamma} p_{ab}|ab\rangle
\]

% \end{document}
