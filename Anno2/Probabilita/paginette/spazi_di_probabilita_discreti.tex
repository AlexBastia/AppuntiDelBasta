% \begin{document}

\chapter{Spazi di probabilità discreti}
\section{Concetti introduttivi}
Innanzi tutto andiamo a definire che cosa intendiamo per \textit{esperimento aleatorio, esito, probabilità}

Con la dicitura esperimento \textit{aleatorio indicheremo} qualunque fenomeno (fisico, economico, sociale, \dots ) il cui esito non sia determinabile con certezza a priori. Il nostro obiettivo è di fornire una descrizione matematica di un esperimento aleatorio, definendo un modello probabilistico, un \textit{esito} invece è un ipotetico risultato di un'esperimento aleatorio sulla base di un cosiddetto \textit{spazio campionario} un insieme che contiene tutti gli esiti possibili dell’esperimento

\esempio{
    \begin{itemize}
        \item \textbf{Esperimento aleatorio:} Lancio di un dado.
        \item \textbf{Spazio campionario:} $\Omega = \{1, 2, 3, 4, 5, 6\}$.
        \item \textbf{Esito:} $4$.
    \end{itemize}
}

\nt{
  In casi piu' complessi ci saranno vari sotto-esperimenti aleatori, come 10 lanci di un dato.
}

Adesso forniamo vere e priorie definizioni
\dfn{evento}{
    Si definisce \textbf{evento} un'affermazione riguardante l'ipotetico esito univoco dell'esperimento, di cui si può affermare con certezza se è vero o falso una volta noto l'esito
}
\ex{}{
    Esper. aleatorio: Lancio del dado\\
    $A = \text{"esce un numero pari"}$
}

\dfn{Spazio camipionario}{
  Si chiama \textit{spazio campionario} un qualunque insieme $ \Omega $ che contiene tutti gli esiti dell'esperimento aleatorio.
}

Notare che non dice "tutti e solo tutti", quindi \textbf{tutti} gli insiemi a cui appartengono gli esiti sono degli spazi campionari.

\ex{Lancio dado}{
  Possiamo porre come spazio campionario:
  \[
  \Omega = \{1,2,3,4,5,6\}
  \]
  ma anche
  \[
  \Omega = \mathbb{R}
  \]
}

\dfn{Esiti favorevoli}{
  Esiti per cui un evento e' vero sono detti esiti favorevoli.
}

\dfn{Evento in termine di insiemi}{
  Un evento si puo' definire anche come il sottoinsieme dello spazio campionario $ \Omega $ formato da tutti gli esiti favorevoli dell'evento.
}

\ex{}{
$ \Omega = \{1,2,3,4,5,6\} \implies A = \text{"esce un numero pari"} = \{2,4,6\} = \text{evento}$
 } 

 \nt{
   La definizione insiemistica di evento dipende dallo spazio campionario definito, mentre l'insieme degli esiti favorevoli e' fisso e rappresenta l'insieme evento di cardinalita' maggiore possibile (esiti favorevoli di $A \subseteq \Omega $).
 }

\dfn{}{
  \begin{itemize}
    \item $ \Omega $ e' l'evento \textit{certo}
    \item $ \emptyset $ e' l'evento \textit{impossibile}
    \item $ \omega \in \Omega $ e' un evento \textit{elementare} ($ A = \{\omega\} $)
  \end{itemize}
}
\ex{}{
  Lancio un dado.
  \[
  A = \text{"esce un numero naturale tra 1 e 6"}
  \]
  \[
  B = \text{"esce un numero maggiore di 6"}
  \]
  \[
  C = \text{"esce il numero 3"}
  \]
  \begin{itemize}
    \item $ \Omega = \{1,2,3,4,5,6\} \implies A = \Omega (evento certo), B = \emptyset (evento impossibile), C = \{3\} $
    \item $ \Omega = \mathbb{R} \implies A = \{1,2,3,4,5,6\} \neq \Omega (evento quasi certo), B = (6,+\infty) (evento quasi impossibile), C = \{3\} $
  \end{itemize}
}

\section{Regole del calcolo probabilistico}
Ad ogni relazione logica possiamo associare un'operazione insiemistica:

\begin{center}
  \begin{tabular}{c|c}
    Connettivi Logici & Connettvi Insiemistici\\
    \hline
    $ A \lor B $ & $ A \cup B $ \\
    $ A \land B $ & $ A \cap B $ \\
    $ \neg A $ & $ A^{c} $ \\
    $ A \implies B $ & $ A \subseteq B $ \\
    $ A \iff B $ & $ A = B $
\end{tabular}
\end{center}

\nt{
  Nella prima colonna, $ A $ e $ B $ sono eventi come affermazioni, mentre nella colonna di destra sono degli insiemi.
}


\subsection{Assiomi della probabilita'}
Poniamo tre assimi fondamentali da cui possiamo partire per derivare tutte le operazioni e proprieta' che ci servono:

\nt{
  Per noi tutti i sottoinsiemi di $ \Omega $ sono eventi (anche se non sara' sempre cosi)
}
\dfn{Assioma 1}{
  A ciascun sottoinsieme $ A $ di $ \Omega $ ($ \forall A \in \powerset(\Omega) $) e' assegnato un numero che chiamo $ \mathbb{P}(A) \in [0,1] $. Questo numero si chiama \textit{probabilita'} di $ A $.
}

\nt{
  Quindi, per il primo assioma, esiste una funzione probabilita' $ \mathbb{P}(A): \powerset (\Omega) \to [0,1] $.
}

\dfn{Assioma 2}{
  Dato $ \Omega $ spazio campionario:
  \[
   \mathbb{P}(\Omega) = 1 
  \]
}

\dfn{Assioma 3: Proprieta' di sigma addittivita'}{
  Sia $ (A_n)_{n \in \mathbb{N}} $ una sequenza di insiemi (eventi) tali che $\forall i \neq j. A_i \cap A_j = \emptyset  $ (insiemi disgiunti) si ha che:
  \[
  \mathbb{P}\left(\bigcup_{i=1}^{\infty}  A_i\right) = \sum_{i=1}^{+\infty}\mathbb{P}(A_i)
  \]
  Quindi la probabilita' di un unione infinita di eventi \textbf{disgiunti} e' uguale alla somma delle probabilita' dei singoli eventi.
}


\dfn{Probabilita' discreta}{
  Chiamo probabilita' discreta una funzione probabilita' $ \mathbb{P} $ a valori discreti, ovvero che puo' assumere un numero finito o al piu' numerabile di valori fra 0 e 1.
}

Vediamo una tale probabilita':

\dfn{Delta di Dirac}{
  Sia $ \Omega = \mathbb{R}, x_0 \in \mathbb{R} $, allora si chiama delta di Dirac centrato in $ x_0 $ la funzione:
\[
  \begin{aligned} 
    \delta_{x_0}: \powerset(\mathbb{R}) &\to [0,1]\\
    A &\mapsto \delta_{x_0}(A) = \begin{cases}
    1 & x_0 \in A\\
    0 & x_0 \notin A
    \end{cases}
  \end{aligned}
\]
}

Notare che per definizione, la funzione di Dirac e' una probabilita' discreta, dato che soddisfa tutti gli assiomi (ma non molto utile dato che assume solo due valori). Pero', tramite le delta di Dirac siamo in grado di costruire qualunque altra probabilita' discreta:

Sia $ \Omega = \mathbb{R}$. Prendiamo un numero contabile $ n $ di eventi singoletto $ x_1,x_2,...,x_n \in \mathbb{R} $ a cui corrispondono $ p_1,p_2,...,p_n \in \mathbb{R} $ tale che:
\[
 \forall i = 1,...,n.\ p_i \in [0,1], \qquad  \sum_{i=1}^{n} p_i = 1 
\]
Definiamo la funzione:
\[
\begin{aligned}
  \mathbb{P}: \powerset(\Omega) &\to [0,1]\\
  A &\mapsto \sum_{i=1}^{n} p_i \delta_{x_i}(A)
\end{aligned}
\]
$ \mathbb{P} $ e' una combinazione lineare di delta di Dirac. Essendo una combinazione convessa, $ \mathbb{P} \in [0,1] $ e si puo' dimostrare che soddisfa gli altri due assiomi, quindi e' una probabilita' discreta! Variando le $ x $ e le $ p $ e' possibile generare qualsiasi funzione $ \mathbb{P} $ discreta.

\ex{}{
  $ \Omega = \{1,2,3,4,5,6\},\ \forall i = 1,...,6.\ x_i = i,\ p_i = \frac{1}{6} $, la funzione $ \mathbb{P} $ associata e':
  \[
   P(A) = \sum_{ i=1}^{6} \frac{1}{6} d_{x_i}(A)
  \]
  \[
    A = \{1,2,3,4,5,6\} \implies P(A) = 1 
  \]
  \[
    B = (6,+\infty) \implies P(B) = 0
  \]
  \[
    C = \{1,2,3,4,5,6,7,8,9\} \implies P(C) = 1
  \]
}

\dfn{}{
  Si chiama evento quasi certo un evento $ A.\ P(A) = 1 $
}

\dfn{}{
  Si chiama evento quasi impossibile un evento $ A.\ P(A) = 0 $
}

Posso allargare $ \Omega $ quanto voglio perche' tanto fuori dall'insieme minimo che comprende tutti gli eventi possibili le probabilita' che aggiugo sono quasi impossibili e quindi hanno probabilita' 0 e non cambiano il valore totale della somma.

\subsection{Conseguenze degli assiomi}

\thm{}{
  Sia $ \Omega $ spazio campionario e $ \mathbb{P} $ probabilita' su $ \Omega $ ($ (\Omega, \mathbb{P}) $ e' uno spazio di probabilita' con $ \mathbb{P}: \powerset(A) \to [0,1] $). Dagli assiomi 1,2,3 deduciamo le cose seguenti:
  \begin{enumerate}
  \item $ \mathbb{P}(\emptyset) = 0 $
  \item \textbf{Addittivita' finita}: $ (A_i)_{i = 1,...,n}.\ \forall i \neq j.A_i \cap A_j = \emptyset \implies \mathbb{P}(\bigcup_{i=1}^{n} A_i) = \sum_{i=1}^{n} \mathbb{P}(A_i) $
  \item $\mathbb{P}(A^c) = 1 - \mathbb{P}(A)$
  \item \textbf{Monotonia}: $ A \subseteq B \implies \mathbb{P}(A) \leq \mathbb{P}(B) $
  \end{enumerate}
}

\pf{Dimostriamo}{
  \begin{enumerate}
    \item Devo mostrare che $ \mathbb{P}(\emptyset) = 0$. Per semplicita' definiamo $ p \coloneq \mathbb{P}(\emptyset) $ Uso l'assioma 3 con la successione $ (A_n)_{n \in \mathbb{N}}.\ \forall i \in \mathbb{N}. A_i = \emptyset $, che sono tutti eventi disgiunti, quindi $ \mathbb{P}(\bigcup_{i=1}^{\infty} A_i) = \sum_{i=1}^{\infty} P(A_i) = \sum_{i=1}^{\infty} p $. Inoltre:

\[
\bigcup_{i=1}^{\infty} A_i = \emptyset
\]

Quindi:
\[
p = \sum_{i=1}^{\infty} p = \begin{cases}
0 & p=0\\
  +\infty & p \in (0,1]
\end{cases}
\]
L'equazione e' soddisfatta solo per $ p = 0 $.

\item Suppongo di avere una sequenza finita disgiunta $ A_n $, devo dimostrare che $ \mathbb{P}(\bigcup_{i=1}^{n} A_i) = \sum_{i=1}^{n} \mathbb{P}(A_i) $. Definisco $ (B_i)_{i \in \mathbb{N}} $ tale che $ B_i = A_i \forall i = 1,...,n $ e $ \forall i > n. B_i = \emptyset $. Usando l'assioma 3:
  \[
    \mathbb{P}\left(\bigcup_{i=1}^{\infty} B_i\right) = \sum_{i=1}^{\infty} \mathbb{P}(B_i) = \sum_{i=1}^{n} A_i 
  \]
\item Devo dimostrare che $ \mathbb{P}(A^{c}) = 1 - \mathbb{P}(A) $. Per definizione di complemento $ A^{c} + A = \Omega $ e i due insiemi sono disgiunti. Per l'addittivita' $ \mathbb{P}(A^{c}) + \mathbb{P}(A) = \mathbb{P}(A^{c}+A) = \mathbb{P}(\Omega) = 1 $ per l'assioma 2.
\item Assumiamo che $ A \subseteq B $. Dimostriamo che $ \mathbb{P}(A) \leq \mathbb{P}(B) $. Per ipotesi $ B = A \cup (B\setminus A) $, quindi per addittivita' $ \mathbb{P}(B) = \mathbb{P}(A) + \mathbb{P}(B\setminus A) $. Per assioma 1 ovvio.
  \end{enumerate}
}

\thm{Probabilita' unione non disgiunta}{
  siano $ A $ e $ B $ eventi:
  \[
    \mathbb{P}(A \cup B) = \mathbb{P}(A) + \mathbb{P}(B) - \mathbb{P}(A \cap B)
  \]
}

\pf{}{
  $ A \cup B = (A \setminus B) \cup (B\setminus A) \cup (B \cap A) $, quindi per addittivita':
  \[
    \mathbb{P}(A \cup B) = \mathbb{P}(A \setminus B) + \mathbb{P}(B \setminus A) + \mathbb{P}(A \cap B)
  \]
  $ \mathbb{P}(A \setminus B) + \mathbb{P}(A \cap B) = \mathbb{P}(A) $ e $ \mathbb{P}(B \setminus A) + \mathbb{P}(A \cap B) = \mathbb{P}(B)  $ dato che $ (X \setminus Y) \cup (X \cap Y) $, ovvio.
}

\nt{
  La formula si complica con un numero di eventi maggiore di 2, infatti per $ n=3 $:
  \[
    \mathbb{P}(A \cup B \cup C) = \mathbb{P}(A) + \mathbb{P}(b) + \mathbb{P}(C) - \mathbb{P}(A \cap B) - \mathbb{P}(B \cap C) - \mathbb{P}(A \cap C) + \mathbb{P}(A \cap B \cap C)
  \]
}

Provare esercizi (da soli!)

% \end{document}
