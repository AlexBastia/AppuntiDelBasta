% \begin{document}
\chapter{Vettori Aleatori}

\section{Introduzione}

Si introduce tale concetto di \textit{vettori aleatori} per trattare variabili aleatorie che non sono più scalari, ma vettori. Infatti ciò che è stato presentato fino ad ora, dato uno spazio di probabilità $(\Omega, \mathbb{P})$, sono esempi con una sola variabile aleatoria, che assume valori in $\mathbb{R}$, tuttavia, è possibile definire più variabili aleatoria in un qualsiasi spazio di probabilità, questo ci induce, così, a considerare la tupla $(X,X_1, \dots, X_n)$ definita su $\Omega$ e che assume valori in $\mathbb{R}^n$ (in base alla quantità di variabili aleatorie). Questa tupla è detta \textit{vettore aleatorio}

\dfn{Vettore aleatorio}{
    Sia $(\Omega, \mathbb{P})$ uno spazio di probabilità, una qualunque funzione
    \begin{equation}
        (X_1, \dots, X_n): \Omega \to \mathbb{R}^n
    \end{equation}
    è definita \textbf{vettore aleatorio}
}

\subsection{Legge di un vettore aleatorio}
Come per le variabili aleatorie, possiamo associare ad ogni vettore aleatorio la sua distribuzione o legge. Diamo la definizione solo per il caso bidimensionale.

\dfn{Legge di un vettore aleatorio}{
    Sia $(\Omega, \mathbb{P})$ uno spazio di probabilità $(X,Y):\Omega\to\mathbb{R}^2$. Si definisce \textbf{legge di un vettore aleatorio} la probabilità
    \begin{equation}
        \mathbb{P}_{(X,Y)} = \mathcal{P}(\mathbb{R^2})\to [0,1]
    \end{equation}
    definita da
    \begin{equation}
        \mathbb{P}_{(X,Y)}(B) = \mathbb{P}((X,Y)\in B) \quad \forall B\subset\mathbb{R}^2
    \end{equation}
    Per dire che $(X, Y )$ ha distribuzione o legge $\mathbb{P}_{(X,Y)}$ si screive
    \[
        (X,Y)\sim\mathbb{P}_{(X,Y)}
    \]
}

\nt{
    Se $B\subset\mathbb{R}^2$è il prodotto cartesiano di due insiemi $B_1$ e $B_2$ in $\mathbb{R}$, allora
    \[
        \mathbb{P}(\xy\in B_1\times B_2) = \P(\{X\in B_1\}\cap\P\{Y\in B_2\})
    \]
    Infatti l'evento $\{(X,Y)\in B_1\times B_2\}$ è dato da
    \[
        \{\xy\in B_1\times B_2\} = \{X\in B_1\}\cap\{Y\in B_2\}
    \]
}

$\mathbb{P}(\{X \in B_1\} \cap \{Y \in B_2\})$ verrà scritto nella forma nella forma $\mathbb{P}(X \in B_1, Y \in B_2)$.



Introduciamo il concetto di dipendenza fra variabili aleatorie. 

definizione di vettori aleatori n-dimensionali

TODO: zio pera ho perso tutto il resto della lezione

Facciamo un passo in piu' 

\section{Vettori aleatori (bidimenzionali) discreti}

definizione (e' quella che ci si aspetta)

Quindi conosciamo i due supporti e le due densita' discrete (che ci identificano interamente la legge anche nel caso bidimensionale come vedremo)

defn di densita' su vettori bidimenzionali

possiamo lavoriamo col supporto (densita' strettamente positiva), ma possiamo notare che questo e' sicuramente un sottoinsieme del prodotto cartesiano dei supporti singoli delle due variabili

teorema analogo a quello uni-dimensionale con densita' supporto e rapporto fra legge e ste cose

la legge congiunta contiene piu' informazioni rispetto a quelle marginali, dato che ha anche informazioni sulla loro dipendenza. Ci aspettiamo quindi di riuscire a ricavarci le marginali ma non viceversa. Come facciamo?

Usiamo la formula delle probabilita' totali per scomporre le legge congiunta (che alla fine e' un'intersezione di eventi). La partizione la creiamo usando y su tutto il suo supporto (dato che gli eventi creati da y diverse saranno diversi e la somma delle probabilita' e' 1). Stessa roba con x.x

Se il supporto e' finito, possiamo fare mega tabellona dove sommando sulle righe o sulle colonne troviamo quello che ci aspettiamo. 

Per fare il contrario, ci servono informazioni sulla dipendenza delle variabili. Infatti, se sappiamo che le variabili sono indipendenti, allora possiamo farlo: teorema con se solo se! quindi e' una caratterizzazione piu' specifica di indipendenza

dfn:
ricordo che x ind y significa che la probabilita' della congiunzione si fattorizza

la dimostrazione e' facile da sx a dx, un po piu' complessa dalll'altra parte ma sono sicuro che il maestro del Pesce Giovanni Il Pescatore sara' in grado di farlo


% \end{document}
