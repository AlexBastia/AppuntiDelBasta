% \begin{document}
\chapter{Vettori Aleatori}

\section{Introduzione}

Si introduce tale concetto di \textit{vettori aleatori} per trattare variabili aleatorie che non sono più scalari, ma vettori. Infatti ciò che è stato presentato fino ad ora, dato uno spazio di probabilità $(\Omega, \mathbb{P})$, sono esempi con una sola variabile aleatoria, che assume valori in $\mathbb{R}$, tuttavia, è possibile definire più variabili aleatoria in un qualsiasi spazio di probabilità, questo ci induce, così, a considerare la tupla $(X,X_1, \dots, X_n)$ definita su $\Omega$ e che assume valori in $\mathbb{R}^n$ (in base alla quantità di variabili aleatorie). Questa tupla è detta \textit{vettore aleatorio}

\dfn{Vettore aleatorio}{
    Sia $(\Omega, \mathbb{P})$ uno spazio di probabilità, una qualunque funzione
    \begin{equation}
        (X_1, \dots, X_n): \Omega \to \mathbb{R}^n
    \end{equation}
    è definita \textbf{vettore aleatorio}
}

\subsection{Legge di un vettore aleatorio}
Come per le variabili aleatorie, possiamo associare ad ogni vettore aleatorio la sua distribuzione o legge. Diamo la definizione solo per il caso bidimensionale.

\dfn{Legge di un vettore aleatorio}{
    Sia $(\Omega, \mathbb{P})$ uno spazio di probabilità $(X,Y):\Omega\to\mathbb{R}^2$. Si definisce \textbf{legge di un vettore aleatorio} la probabilità
    \begin{equation}
        \mathbb{P}_{(X,Y)} = \mathcal{P}(\mathbb{R^2})\to [0,1]
    \end{equation}
    definita da
    \begin{equation}
        \mathbb{P}_{(X,Y)}(B) = \mathbb{P}((X,Y)\in B) \quad \forall B\subset\mathbb{R}^2
    \end{equation}
    Per dire che $(X, Y )$ ha distribuzione o legge $\mathbb{P}_{(X,Y)}$ si screive
    \[
        (X,Y)\sim\mathbb{P}_{(X,Y)}
    \]
}

\nt{
    Se $B\subset\mathbb{R}^2$è il prodotto cartesiano di due insiemi $B_1$ e $B_2$ in $\mathbb{R}$, allora
    \[
        \mathbb{P}(\xy\in B_1\times B_2) = \P(\{X\in B_1\}\cap\P\{Y\in B_2\})
    \]
    Infatti l'evento $\{(X,Y)\in B_1\times B_2\}$ è dato da
    \[
        \{\xy\in B_1\times B_2\} = \{X\in B_1\}\cap\{Y\in B_2\}
    \]
}

$\mathbb{P}(\{X \in B_1\} \cap \{Y \in B_2\})$ verrà scritto nella forma nella forma $\mathbb{P}(X \in B_1, Y \in B_2)$.
\subsection{Indipendenza di variabili aleatorie}
Si può estendere il concetto di indipendenza di eventi a quello di indipendenza di variabili aleatorie. Intuitivamente due variabili aleatorie $X$ e $Y$ sono indipendenti se la conoscenza del valore di $X$ non fornisce informazioni sul valore di $Y$ e viceversa. Matematicamente $n$ variabili aleatorie $X_1, \dots, X_n$ sono indipendenti se gli eventi da esse generati, ovvero
\[
    \{X_1\in B_1\} \dots \{X_n\in B_n\} \text{ al variare di tutti i sottoinsiemi } B_1, \dots, B_n \text{ in }\mathbb{R}
\]
sono indipendenti
\dfn{Varaibili aleatorie indipendenti}{
    Sia $(\Omega, \mathbb{P})$ uno spazio di probabilità e $X_1, \dots, X_n$ variabili aleatorie definite su $\Omega$. Si dice che $X_1, \dots, X_n$ sono \textbf{indipendenti} se
    \begin{equation}
        \mathbb{P}(X_1\in B_1, \dots, X_n\in B_n) = \prod_{i=1}^n \mathbb{P}(X_i\in B_i) \quad \forall B_1, \dots, B_n\subset\mathbb{R}
    \end{equation}
    Inoltre la notazione per indicare due indipendenti $X$ e $Y$ è
    \[
        X\dperp Y
    \] 
}
\nt{
    Si dice anche che due variabili aleatorie sono indipendenti se la distribuzione congiunta si fattorizza nel prodotto delle marginali
}
Concludiamo questa sezione con il seguente risultato, in cui si afferma che funzioni di variabili aleatorie indipendenti sono indipendenti

\mprop{indipendenza di funzioni di variabili aleatorie}{
    Siano $X$ e $Y$ due variabili aleatorie indipendenti e siano $f:\mathbb{R}\to\mathbb{R}$ e $g:\mathbb{R}\to\mathbb{R}$ due funzioni arbitrarie. Allora le varaibili aleatorie $f(X)$ e $g(Y)$ sono indipendenti
}
\pf{Dimostrazione}{ \label{prop:indep_var_aleat}
    Siano $B_1, B_2\subset\mathbb{R}$ due sottoinsiemi arbitrari. Si deve dimostrare che
    \[
        \P(f(X)\in B_1, g(Y)\in B_2) = \P(f(X)\in B_1)\cdot\P(g(Y)\in B_2)
    \]

    Siano 
    \begin{align*}
        f^{-1}(B_1) &= \{x\in\mathbb{R} : f(x)\in B_1\}\\
        g^{-1}(B_2) &= \{y\in\mathbb{R} : g(y)\in B_2\}
    \end{align*}
    Le controimmagini di $B_1$ e $B_2$ tramite $f$ e $g$, rispettivamente. Allora
    \[
        \{f(X)\in B_1\} = \{X\in f^{-1}(B_1)\} \quad \text{e} \quad \{g(Y)\in B_2\} = \{Y\in g^{-1}(B_2)\}
    \]
    Quindi
    \begin{align*}
        \P(f(X)\in B_1, g(Y)\in B_2) &= \P(\{X\in f^{-1}(B_1)\}\cap\{Y\in g^{-1}(B_2)\})\\
        &= \P(X\in f^{-1}(B_1))\cdot\P(Y\in g^{-1}(B_2))\\
        &= \P(f(X)\in B_1)\cdot\P(g(Y)\in B_2)
    \end{align*}
}
\nt{
    Per \ref{prop:indep_var_aleat} si deduce che se $X$ e $Y$ sono
    indipendenti allora non pu`o esistere alcuna dipendenza funzionale tra $X$ e $Y$, tranne nel caso in cui almeno una tra $X$ e $Y$ sia una costante.

    In altre parole, non può esistere aluna funzione $f:\mathbb{R}\to\mathbb{R}$ tale che
    \[
        Y = f(X)
    \]

}

\section{Vettori aleatori discreti}
In questa sezione studiamo una particolare classe di vettori aleatori, i vettori aleatori discreti (bidimensionali)

% \begin{quote}
%     (e' quella che ci si aspetta) \citep{il_basta}
% \end{quote}

\dfn{vettore aleatorio discreto}{
    Sia $(\Omega, \mathbb{P})$ uno spazio di probabilità e $(X,Y)$ un vettore aleatorio. Si dice che $(X,Y)$ è un \textbf{vettore aleatorio discreto} se sia $X$ che $Y$ sono variabili aleatorie discrete
}
Si noti che il vettore $\xy$ assume un numero finito o al più numerabile di valori, dati al più da tutte le coppie $\mathcal{S}_X\times \mathcal{S}_Y $. Quindi il supporto di $\xy$ è un sottoinsieme di $\mathcal{S}_X\times \mathcal{S}_Y$. Adesso prensento la la dentisità discreta
\dfn{Densitò discreta congiunta}{
    Sia $(\Omega, \mathbb{P})$ uno spazio di probabilità e $(X,Y)$ un vettore aleatorio discreto. La funzione $p_{(X,Y)}:\mathbb{R}^2\to[0,1]$ definita da
    \[
        p_{(X,Y)}(x,y) =\P(X=x, Y=y) = \mathbb{P}(\xy=(x,y)) \quad \forall (x,y)\in\mathcal{S}_X\times\mathcal{S}_Y
    \]
    è detta \textbf{densità discreta congiunta} di $X$ e $Y$
}

Si noti che $p_{(X,Y)}$ è la probabilità che il vettore aleatorio $\xy$ assuma il valore $(x,y)$, quindi $p_{(X,Y)}(x,y)$ verifica le seguenti disuguaglianze
\[
    0\leq p_{(X,Y)}(x,y) \leq 1\quad \forall (x,y)\in\mathcal{S}_X\times\mathcal{S}_Y
\]

\thm{
    sulla legge congiunta di un vettore aleatorio discreto
}{ \label{thm:legge_congiunta_discreta}
    Sia $(\Omega, \mathbb{P})$ uno spazio di probabilità e $(X,Y)$ un vettore aleatorio discreto. Sia $\mathcal{S}_X$ e $\mathcal{S}_Y$ i supporti di $X$ e $Y$, rispettivamente. Allora:
    \begin{itemize}
        \item $\p =0\,\forall(x,y)\notin\ss$
        \item $\sum_{x\in\mathcal{S}_X}\sum_{y\in\mathcal{S}_Y}p_{(X,Y)}(x,y) = 1$
        \item Vale la seguente formula
        \begin{equation} \label{eq:formula_probabilita_totali}
            \P(\xy\in B) = \sum_{(x,y)\in B}p_{(X,Y)}(x,y) \quad \forall B\subset\mathcal{S}_X\times\mathcal{S}_Y
        \end{equation}
    \end{itemize}
    
}
\subsection{Densità discreta congiunta e densità discrete marginali}
In questa sezione si indaga la relazione tra la densità congiunta $p_(X,Y)$ e le marginali $p_X$ e $p_Y$. Ci si ricordi che
    \begin{align*}
        p_X(x) &= \P(X=x) \\
        p_Y(y) &= \P(Y=y) \\
        p_{(X,Y)}(x,y) &= \P(X=x, Y=y)
    \end{align*}

Dalla formula delle probabilità totali otteniamo il seguente risultato
\thm{}{ \label{thm:formula_probabilita_totali}
    Sia $\xy$ un vettore aleatorio discreto. Allora
    \begin{align*}
        p_X(x) &= \sum_{y\in\mathcal{S}_Y}p_{(X,Y)}(x,y) \quad \forall x\in\mathcal{S}_X\\
        p_Y(y) &= \sum_{x\in\mathcal{S}_X}p_{(X,Y)}(x,y) \quad \forall y\in\mathcal{S}_Y
    \end{align*}
}
\pf{Dimostrazione}{
    Dimostriamo solo la prima formula nel caso in cui $S_Y$ è un insieme
    finito: $S_Y = \{y_1, \ldots, y_m\}$. Quindi dobbiamo dimostrare che, per ogni $x_i \in S_X$ fissato,
    vale
    \[
    p_X(x_i) = \sum_{j=i}^{m} p_{(X,Y)}(x_i, y_j).
    \]
    Riscritta in termini di $\mathbb{P}$ diventa
    \[
        \mathbb{P}(X = x_i) = \sum_{j=i}^{m} \mathbb{P}(X = x_i, Y = y_j)
    \]
    Poniamo
    \[
    A = \{X = x_i\}, \qquad B_j = \{Y = y_j\}, \quad \forall j = 1, \ldots, m.
    \]
    Gli eventi $B_1, \ldots, B_m$ sono una partizione di $\Omega$. Quindi, dalla formula delle probabilità
    totali abbiamo che
    \[
    \mathbb{P}(A) = \sum_{j=i}^{m} \mathbb{P}(A \cap B_j),
    \]
    che corrisponde all'uguaglianza.
}
\subsubsection{Tabella della densità discreta congiunta}
Nel caso in cui sia $\mathcal{S}_X$ che $\mathcal{S}_Y$ siano finiti, quindi
\[
    \mathcal{S}_X = \{x_1, \ldots, x_n\} \quad \text{e} \quad \mathcal{S}_Y = \{y_1, \ldots, y_m\}
\]
possiamo riportare i valori di $p_{(X,Y)} $in una tabella:

\begin{center}
    \begin{tabular}{|c|c|c|c|c|c|}
        \hline
        \diagbox{$X$}{$Y$} & $y_1$ & $y_2$ & $\cdots$ & $y_m$ & $p_X$ \\
        \hline
        $x_1$ & $p_{(X,Y)}(x_1, y_1)$ & $p_{(X,Y)}(x_1, y_2)$ & $\cdots$ & $p_{(X,Y)}(x_1, y_m)$ & $p_X(x_1)$ \\
        \hline
        $x_2$ & $p_{(X,Y)}(x_2, y_1)$ & $p_{(X,Y)}(x_2, y_2)$ & $\cdots$ & $p_{(X,Y)}(x_2, y_m)$ & $p_X(x_2)$ \\
        \hline
        $\vdots$ & $\vdots$ & $\vdots$ & $\ddots$ & $\vdots$ & $\vdots$ \\
        \hline
        $x_n$ & $p_{(X,Y)}(x_n, y_1)$ & $p_{(X,Y)}(x_n, y_2)$ & $\cdots$ & $p_{(X,Y)}(x_n, y_m)$ & $p_X(x_n)$ \\
        \hline
        $p_Y$ & $p_Y(y_1)$ & $p_Y(y_2)$ & $\cdots$ & $p_Y(y_m)$ & $1$ \\
        \hline
    \end{tabular}
\end{center}

Per il \ref{thm:formula_probabilita_totali} si ha che i valori di $p_X$ si ottengono sommando i valori di $p_{(X,Y)}$ lungo le righe, mentre i valori di $p_Y$ si ottengono sommando i valori di $p_{(X,Y)}$ lungo le colonne. Infine, sommando i valori dell'ultima colonna (quindi i valori di $p_X$) si ottiene $1$. Analogamente, sommando i valori dell'ultima riga (quindi i valori di $p_Y$) si ottiene ancora $1$. Questo spiega la presenza del numero $1$ nell'angolo in basso a destra della tabella.

\subsection{Indipendenza e Densità discreta congiunta}
Conoscendo la densità discreta congiunta $p_{(X,Y)}$ di un vettore aleatorio discreto e possibile ricostruire le marginali $p_X$ e $p_Y$. Non è possibile viceversa

\ex{
    Esercizio sulla densità discreta congiunta
}{
    Si lanciano due monete e si considerano le seguenti variabili aleatorie
    \begin{align*}
        X &= \text{“vale 1 se l'esito del lancio della prima moneta è testa, vale 0 altrimenti”}\\
        Y &= \text{“vale 1 se l'esito del lancio della seconda moneta è testa, vale 0 altrimenti”}
    \end{align*}

    Se si considera il vettore aleatorio $\xy$, allora
    \begin{center}
        \begin{tabular}{c|c|c|c}
            
            \diagbox{$X$}{$Y$} & $0$ & $1$ &$p_X$\\
            \hline
            $0$ & $\frac{1}{4}$ & $\frac{1}{4}$ & $\frac{1}{2}$\\
            \hline
            $1$ & $\frac{1}{4}$ & $\frac{1}{4}$ &$\frac{1}{2}$\\
            \hline
            $p_Y$ & $\frac{1}{2}$ & $\frac{1}{2}$ & $1$\\
        \end{tabular}
    \end{center}
    Se si considera il vettore aleatorio $(X,X)$, allora
    \begin{center}
        \begin{tabular}{c|c|c|c}
            \diagbox{$X$}{$X$} & $0$ & $1$ &$p_X$\\
            \hline
            $0$ & $\frac{1}{2}$ & $0$ & $\frac{1}{2}$\\
            \hline
            $1$ & $0$ & $\frac{1}{2}$ &$\frac{1}{2}$\\
            \hline
            $p_X$ & $\frac{1}{2}$ & $\frac{1}{2}$ & $1$\\
            
        \end{tabular}
    \end{center}
}

Adesso si osservi il seguente teorema
\thm{
    Indipendenza di variabili aleatorie discrete
}{
    Siano $X$ e $Y$ variabili aleatorie discrete. Allora $X$ e $Y$ sono indipendenti se e solo se
    \begin{equation} \label{eq:indep_var_aleat}
        p_{(X,Y)}(x,y) = p_X(x)p_Y(y) \quad \forall (x,y)\in\mathcal{S}_X\times\mathcal{S}_Y
    \end{equation}
}
\pf{Dimostrazione}{
    \begin{itemize}
        \item \textbf{($\Leftarrow$)} si deve dimostrare che se vale \ref{eq:indep_var_aleat} allora $X$ e $Y$ sono indipendenti. 
        
        Siano $B_1$ e $B_2$ due sottoinsiemi arbitrari di $\mathbb{R}$. Si deve dimostrare che
        \[
            \P(X\in B_1, Y\in B_2) = \P(X\in B_1)\cdot\P(Y\in B_2)
        \]
        Dato che $\P(X\in B_1, Y\in B_2) = \P(\xy\in B_1\times B_2 )$ per la \ref{eq:formula_probabilita_totali} si ha che
        \[
            \P(X\in B_1, Y\in B_2) = \sum_{(x,y)\in B_1\times B_2}p_{(X,Y)}(x,y) =\sum_{x\in B_1}\sum_{y\in B_2}p_{(X,Y)}(x,y)
        \]
        Per \label{eq:indep_var_aleat} si ha che
        \[
            \sum_{x\in B_1}\sum_{y\in B_2}p_{(X,Y)}(x,y) = \sum_{x\in B_1}\sum_{y\in B_2}p_X(x)p_Y(y) 
        \]
        Eseguendo ulteriori passaggi si ha che
        \[
            \sum_{x\in B_1}\sum_{y\in B_2}p_X(x)p_Y(y) = \sum_{x\in B_1}p_X(x)\cdot\sum_{y\in B_2}p_Y(y) = \P(X\in B_1)\cdot\P(Y\in B_2)
        \]
        Quindi 
        \[
            \P(X\in B_1, Y\in B_2) = \P(X\in B_1)\cdot\P(Y\in B_2)
        \]
        Dimostrato
        \item \textbf{($\Rightarrow$)} si deve dimostrare che se $X$ e $Y$ sono indipendenti allora vale \ref{eq:indep_var_aleat}. Siano $B_1=\{x\}$ e $B_2=\{y\}$ due sottoinsiemi arbitrari di $\mathbb{R}$ per la definizione di indipendenza si ottiene l'uguaglianza \ref{eq:indep_var_aleat}
        
        Dimostrato
    \end{itemize}
}

Esercizietto

\ex{}{
    Siano $X$ e $Y$ due variabili aleatorie discrete con densità discreta congiunta
    \begin{center}
        \begin{tabular}{|c|c|c|c|c|}
            \hline
            \diagbox{X}{Y} & -1 & 5 & 10 & $p_X$ \\
            \hline
            0 &  & 0.12 &  & 0.4 \\
            \hline
            5 &  &  &  &  \\
            \hline
            $p_Y$ & 0.3 &  &  & 1 \\
            \hline
            \end{tabular}
    \end{center}
    \begin{enumerate}
        \item Completare la tabella in modo che $X$ e $Y$ siano indipendenti.
        \item Calcolare $P(X < Y)$.
        \item Calcolare $P(\vert XY \vert \ge 5)$ e $P(X+Y > 5)$.
        \item Siano $U = \vert XY \vert$ e $V = X + Y$. Trovare la densità discreta congiunta di $U$ e $V$ e le densità marginali
    \end{enumerate}
}
\pf{Soluzione}{
    Sia data al bonzo
}
\subsection{Valore atteso e varianza di una funzione di $(X, Y)$}

Nel seguito capiter`a spesso di dover calcolare valore atteso e varianza di una funzione di $(X, Y )$:
\[
    h(X, Y ) 
\]
Risulta dunque particolarmente utile il seguente risultato
\thm{
    sul valore atteso e varianza di una funzione di un vettore aleatorio discreto
}{ \label{thm:valore_atteso_varianza}
    Siano $\xy$ un vettore aleatorio discreto e $h:\mathbb{R}^2\to\mathbb{R}$. Allora
    \[
        \mathbb{E}[h(X,Y)] = \sum_{(x,y)\in\mathcal{S}_X\times\mathcal{S}_Y}h(x,y)p_{(X,Y)}(x,y)
    \]
    e
    \[
        Var(h(X,Y)) = \mathbb{E}[(h\xy-\mathbb{E}[h\xy])^2] = \sum_{(x,y)\in\mathcal{S}_X\times\mathcal{S}_Y}(h(x,y)-\mathbb{E}[h(X,Y)])^2p_{(X,Y)}(x,y)
    \]
    Vale inoltre la formula
    \[
        Var(h(X,Y)) = \mathbb{E}[h(X,Y)^2] - (\mathbb{E}[h(X,Y)])^2 = \sum_{(x,y)\in\mathcal{S}_X\times\mathcal{S}_Y}h(x,y)^2p_{(X,Y)}(x,y) - (\mathbb{E}[h(X,Y)])^2
    \]
}

\cor{somma di $X$ e $Y$}{
    Siano $X$ e $Y$ due variabili aleatorie discrete. Siano inoltre $a$ e $b$ due costanti. Allora
    \[
        \mathbb{E}[aX + bY] = a\mathbb{E}[X] + b\mathbb{E}[Y]
    \]
}
\pf{Dimostrazione}{
    Per il teorema \ref{thm:valore_atteso_varianza} si ha che con $h(x,y)=ax+by$ si ottiene
    \[
        \begin{aligned}
            \mathbb{E}[aX + bY] &= \sum_{(x,y)\in\mathcal{S}_X\times\mathcal{S}_Y}(ax+by)p_{(X,Y)}(x,y)\\
            &= \sum_{(x,y)\in\mathcal{S}_X\times\mathcal{S}_Y}axp_{(X,Y)}(x,y) + \sum_{(x,y)\in\mathcal{S}_X\times\mathcal{S}_Y}byp_{(X,Y)}(x,y)\\
            &= a\sum_{(x,y)\in\mathcal{S}_X\times\mathcal{S}_Y}xp_{(X,Y)}(x,y) + b\sum_{(x,y)\in\mathcal{S}_X\times\mathcal{S}_Y}yp_{(X,Y)}(x,y)\\
            &= a\sum_{x\in\mathcal{S}_X}x\sum_{y\in\mathcal{S}_Y}p_{(X,Y)}(x,y) + b\sum_{y\in\mathcal{S}_Y}y\sum_{x\in \mathcal{S}_X} p_{(X,Y)}(x,y)\\
            &= a\sum_{x\in\mathcal{S}_X}xp_X(x) + b\sum_{y\in\mathcal{S}_Y}yp_Y(y)\\
            &= a\mathbb{E}[X] + b\mathbb{E}[Y]
        \end{aligned}
    \]
}

\cor{Prodotto $X$ e $Y$}{
    Siano $X$ e $Y$ due variabili aleatorie discrete. Se $X$ e $Y$ sono indipendenti, allora
    \[
        \mathbb{E}[XY] = \mathbb{E}[X]\mathbb{E}[Y]
    \]
}
\nt{
    NON VALE IL CONTRARIO: se $\mathbb{E}[XY] = \mathbb{E}[X]\mathbb{E}[Y]$ non è necessariamente vero che $X$ e $Y$ sono indipendenti
}
\pf{Dimostrazione}{
    Per il teorema \ref{thm:valore_atteso_varianza} si ha che con $h(x,y)=xy$ si ottiene
    \[
        \begin{aligned}
            \mathbb{E}[XY] &= \sum_{(x,y)\in\mathcal{S}_X\times\mathcal{S}_Y}xy p_{(X,Y)}(x,y)\\
            &\underset{\substack{\uparrow \\ \text{indipendenza}}}{=} \sum_{(x,y)\in\mathcal{S}_X\times\mathcal{S}_Y}xy p_X(x)p_Y(y)\\
            &= \sum_{x\in\mathcal{S}_X}xp_X(x)\sum_{y\in\mathcal{S}_Y}yp_Y(y)\\
            &= \mathbb{E}[X]\cdot\mathbb{E}[Y]
        \end{aligned}
    \]
}
\subsection{Indici di sintesi della distribuzione di un vettore aleatorio discreto}

In termini intuitivi, la covarianza è una misura di come due variabili aleatorie (chiamiamole X e Y) tendono a variare insieme. Ti dice se, in media, quando una variabile assume valori sopra la sua media, l'altra tende a fare lo stesso, oppure se tende a fare il contrario.

In altre parole, la covarianza misura la direzione della relazione lineare tra due variabili aleatorie. Se la covarianza è positiva, significa che quando una variabile aumenta, l'altra tende ad aumentare. Se è negativa, significa che quando una variabile aumenta, l'altra tende a diminuire. Se è zero, non c'è una relazione lineare tra le due variabili.

\dfn{Covarianza}{ \label{def:covarianza}
    Siano $X$ e $Y$ due variabili aleatorie discrete. La \textbf{covarianza} di $X$ e $Y$ è definita come
    \[
        Cov(X,Y) = \mathbb{E}[(X-\mathbb{E}[X])(Y-\mathbb{E}[Y])] = \sum_{(x,y)\in\mathcal{S}_X\times\mathcal{S}_Y}(x-\mathbb{E}[X])(y-\mathbb{E}[Y])p_{(X,Y)}(x,y)
    \]
    Se $Cov(X,Y) = 0$ si dice che $X$ e $Y$ sono \textbf{scorrelate}
}

\nt{
    Si noti che la covarianza di $X$ e $Y$ è definita come il valore atteso della variabile aleatoria $h(X,Y)$, dove
    \[
        h(X,Y) = (X-\mathbb{E}[X])(Y-\mathbb{E}[Y])
    \]
    Quindi la seconda uguaglianza in \ref{def:covarianza} è una conseguenza del teorema \ref{thm:valore_atteso_varianza}W
}
\nt{
    Si noti che
    \[
        Cov(X,X) = Var(X)
    \]
}
\nt{
    La covarianza è simmetrica, ovvero
    \[
        Cov(X,Y) = Cov(Y,X)
    \]
}

Per calcolare la covarianza di $X$ e $Y$ è utile la seguente formula
\thm{
    Formula per la covarianza
}{ \label{thm:formula_cov}
    Siano $X$ e $Y$ due variabili aleatorie discrete. Allora
    \[
        Cov(X,Y) = \mathbb{E}[XY] - \mathbb{E}[X]\mathbb{E}[Y] = \sum_{(x,y)\in\mathcal{S}_X\times\mathcal{S}_Y}xy p_{(X,Y)}(x,y) - \mathbb{E}[X]\mathbb{E}[Y]
    \]
}
\nt{
    Se le variabili aleatorie $X$ e $Y$ sono indipendenti, allora sono scorrelate
}
\pf{Dimostrazione}{
    Per definizione, si ha che
    \[
        Cov(X,Y) = \mathbb{E}[(X-\mathbb{E}[X])(Y-\mathbb{E}[Y])] = \mathbb{E}[XY] - \mathbb{E}[X]\mathbb{E}[Y]
    \]
}