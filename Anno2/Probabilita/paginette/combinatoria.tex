\begin{document}
\chapter{Calcolo Combinatorio}  

Quando abbiamo numero finito di elementi, e' possibile contare gli elementi dell'evento per ridurre il problema di probabilita' a un problema di conteggio -> dobbiamo imparare a contare

Dato un certo insieme, dobbiamo calcolarne la cardinalita' utilizzando i giusti stumenti matematici

\thm{Calcolo probabilita' uniforme (Laplace)}{
  Se $ \Omega $ e' finito, $ \Omega = \{w_1,...,w_N\} $ e $ \forall i = 1,...,N.\ \mathbb{P}(\{w_i\}) = \frac{1}{N} $ (probabilita' uniforme), allora $ \forall A \subseteq \Omega $ (evento):
  \[
    \mathbb{P}(A) = \frac{|A|}{N}
  \]
}

Dobbiamo introdurre:
\begin{itemize}
\item \textbf{Fattoriali}:
  \begin{itemize}
  \item $ 0! \coloneq 1 $
  \item $ (n+1)! \coloneq (n+1) \cdot n! $
  \end{itemize}
  \item \textbf{Coefficenti Binomiali}:
    \[
      (n,k) = \frac{n!}{k! (n-k)!} \quad \forall n,k\in \mathbb{N}. n \geq k
    \]
    In generale:
    \[
      (n,k) = (n,n-k)
    \]
    Dal triangolo di Tartaglia, possiamo visualizzare altre proprieta' (oltre alla simmetria)
\end{itemize}

Vedremo che $ (n,k) $ sono il numero di modi diversi che abbiamo per selezionare $ k $ sottoinsiemi diversi da un insieme di cardinalita' $ n $.

\thm{Formula di Newton}{
  \[
    (a+b)^n = \sum_{k=0}^{n} (n,k) a^k b^{n-k}
  \]
}

Quindi anche il coefficente deriva da un problema di conteggio.

\section{Metodo (Principio) delle Scelte Successive}
E' un algoritmo per determinare la cardinalita' di un insieme. Vediamo un esempio:
\ex{}{
  Alfabeto di 36 caratteri dove ognuno dei numeri corrisponde a un carattere alfanumerico.

  \underline{Domanda}: "Quante possibili password distinte di 8 caratteri esistono in questo alfabeto?"

  $ \Omega = \text{Alfanumerici} \times ... \times \text{Alfanumerici} $ 8 volte.

  \begin{itemize}
  \item Scelte:
    \begin{enumerate}
    \item un carattere dei 36 totali
    \item \vdots (e' cosi 8 volte)
    \end{enumerate}
    Quindi $ |\Omega| = 36 \times ... \times 36 = 36^8 $
  \end{itemize}

  E se vogliamo evitare di ripetere i caratteri? Vediamo le scelte:
  \begin{enumerate}
  \item un carattere dei 36 totali
  \item un carattere dei \textbf{35 possibili}
  \item \vdots
  \item un carattere dei 29 possibili
  \end{enumerate}
  Quindi $ |\Omega| = 36 \times 35 \times ... \times 29 = \frac{36!}{28!} $
}

Definiamo il principio generale:

\thm{Non proprio un teorema}{
  Ciascun elemetno di un insieme $ A $ puo' essere determinato tramite sola sequenza di $ k $ scelte, dove per ogni scelta ci sono $ n_1,....,n_k $ possibilita', allora:
  \[
  |A| = n_1 \times ... \times n_k
  \]
}

\nt{
  Puo' essere riscritto come teorema formale ma \textbf{E' TROPPO DIFFICILE PER NOI INFORMATICI} quindi non lo facciamo!!!!
}

\ex{}{
  52 carte (13 tipi 4 semi)
  \begin{enumerate}
    \item $ A = \text{iniseme dei full (un tris e una coppia)} $, $ |A| = ? $

      \underline{4 scelte:}
      \begin{itemize}
        \item tipo di carta nel tris (13)
        \item tipo di carta nella coppia (12)
        \item semi nel tris (4)
        \item semi nella coppia (6)
      \end{itemize}
      $ |A| = 13 12 4 6 = \text{casi favorevoli} $

    \item $ A = \text{doppie coppie (due coppie di tipo diverso e una carta libera)} $, $ |A|=? $

      \underline{Scelte}:
      \begin{itemize}
        \item tipo nella prima coppia (13)
        \item tipo nella seconda coppia (12)
        \item semi nella prima coppia (6)
        \item semi seconda (6)
        \item tipo singolo (11)
        \item seme singolo (4)
      \end{itemize}
      $ |A| = 13 12 6 6 11 4 $ \textbf{SBAGLIATO!!}

      Perche' non ci interessa dell'ordine delle due coppie (bisogna dedurlo dalla definizione di A), quindi NON c'e' una prima e seconda coppia (anche sopra era sbagliato vederlo cosi') dato che non c'e' l'ordine.

      Combinazioni dei tipi che compongono 2 coppie ($ 13 \times 12 / 2 $)
  \end{enumerate}
}

\section{Disposizioni e Combinazioni}

\def{Disposizioni con ripetizione}{
  Dato un insieme $ E $ con $ n $ elementi, indichiamo con $ DR_{n,k} $ le sequenze ordinate di $ k $ elementi di $ E $. 
}
Sostanzialmente, $ DR_{n,k} = E \times E ... \times E $ $ k $ volte, ovvero:
\[
DR_{n,k} = E^k
\]
Quindi usando il principio delle scelte successive:
\[
|DR_{n,k}| = n^k
\]
dato che per ogni $ E $ abbiamo una scelta fra $ n $ elementi.

\nt{
  Indicando tale insieme con $ DR_{n,k} $, l'insieme $ E $ sparisce, dato che ci interessa solo la \textbf{cardinalita'} di tale insieme e non ce ne frega un cavolo dei sui elementi.
}

\ex{Iniziamo a calcolare le probabilita'!}{
  Presa un urna con $ n $ palline numerate ($ E = \{1,...,n\} $) e si estraggono $ k $ palline con \textit{reinbussolamento}.
  \[
  \Omega = E \times ... \times E = DR_{n,k}
  \]
  Quindi $ |\Omega| = n^k $, e la probabilita' uniforme e' data da:
  \[
    \mathbb{P}(\{\omega\}) = \frac{1}{|\Omega|} = n^{-k}
  \]
}
\ex{}{
  Determinare spazi campionari per i seguenti esperimenti:
  \begin{itemize}
  \item Scelta casuale di una parola di 8 lettere
  \item Scelta di colonne deltotocalcio (risultato di 13 partite)
  \end{itemize}
}

Quindi lo usiamo nei casi di estrazione con reinbussolamento quando ci interessa l'ordine.

\def{Disposizioni}{
  Dato un insieme $ E  $ di $ n $ elementi, l'insieme delle disposizioni (senza ripetizione) di $ k $ elementi dell'insieme $ E $ e' l'insieme delle sequenze ordinate di $ k $ elementi \textit{distinti}, ovvero:
  \[
    D_{n,k} \coloneq \{(x_1,...,x_k)|x_1,...,x_k \in E \land x_i \neq x_j \text{se} i \neq j \}
  \]
}

\nt{
  $ D_{n,k} $ e' un sottoinsieme \textbf{stretto} di $ DR_{n,k} $.
}

Usando le scelte successive:
\[
  |D_{n,k}| = n\times (n-1) \times ... \times (n-k+1) = \frac{n!}{(n-k)!}
\]
L'esperimento aleatorio di riferimento e' l'estrazione \textbf{senze} reimissione.

\dfn{Permutazioni}{
  $ P_n = D_{n,n} $, quindi:
  $ |P_n| = n! $
}

\dfn{}{
  Dato un insieme $ E $ di $ n $ elementi, indichiamo con $ C_{n,k} $ la classe dei sottoinsiemi di $ E $ contenenti $ k $ elementi, ovvero:
  \[
  C_{n,k} = \{A \subseteq E | |A|= k\}
  \]
}
Siamo passati da sequenze a sottoinsiemi, ovvero non ci interessa piu' dell'ordine:
\[
\text{sottoinsieme} = \text{sequenza non ordinata}
\]

\end{document}
