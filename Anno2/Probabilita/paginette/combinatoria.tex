% \begin{document}
\chapter{Calcolo Combinatorio}  

Quando abbiamo numero finito di elementi con probabilita' uniforme, e' possibile utilizzare l'addittivita' delle probabilita' per ridurre il problema di probabilita' a un problema di conteggio (la probabilita' di un evento e' data dal numero di casi favorevoli) $ \implies $ dobbiamo imparare a contare.

Dato un certo insieme, dobbiamo calcolarne la cardinalita' utilizzando i giusti stumenti matematici:

\thm{Calcolo probabilita' uniforme (Laplace)}{
  Se $ \Omega $ e' finito, $ \Omega = \{w_1,...,w_N\} $ e $ \forall i = 1,...,N.\ \mathbb{P}(\{w_i\}) = \frac{1}{N} $ (probabilita' uniforme), allora $ \forall A \subseteq \Omega $ (evento):
  \[
    \mathbb{P}(A) = \frac{|A|}{N}
  \]
}

Dobbiamo introdurre:
\begin{itemize}
\item \textbf{Fattoriali}:
  \begin{itemize}
  \item $ 0! \coloneq 1 $
  \item $ (n+1)! \coloneq (n+1) \cdot n! $
  \end{itemize}
  \item \textbf{Coefficenti Binomiali}:
    \[
      \binom{n}{k} = \frac{n!}{k! (n-k)!} \quad \forall n,k\in \mathbb{N}. n \geq k
    \]
    In generale:
    \[
      \binom{n}{k} = \binom{n}{n-k}
    \]
    Dal triangolo di Tartaglia, possiamo visualizzare altre proprieta' (oltre alla simmetria)
\end{itemize}

Vedremo che $ \binom{n}{k} $ sono il numero di modi diversi che abbiamo per selezionare $ k $ sottoinsiemi diversi da un insieme di cardinalita' $ n $.

\thm{Formula di Newton}{
  \[
    (a+b)^n = \sum_{k=0}^{n} \binom{n}{k} a^k b^{n-k}
  \]
}

Quindi anche il coefficente deriva da un problema di conteggio.

\section{Metodo (Principio) delle Scelte Successive}
E' un algoritmo per determinare la cardinalita' di un insieme. Vediamo un esempio:
\ex{}{
  Alfabeto di 36 caratteri dove ognuno dei numeri corrisponde a un carattere alfanumerico.

  \underline{Domanda}: "Quante possibili password distinte di 8 caratteri esistono in questo alfabeto?"

  $ \Omega = \text{Alfanumerici} \times ... \times \text{Alfanumerici} $ 8 volte.

  \begin{itemize}
  \item Scelte:
    \begin{enumerate}
    \item un carattere dei 36 totali
    \item \vdots (e' cosi 8 volte)
    \end{enumerate}
    Quindi $ |\Omega| = 36 \times ... \times 36 = 36^8 $
  \end{itemize}

  E se vogliamo evitare di ripetere i caratteri? Vediamo le scelte:
  \begin{enumerate}
  \item un carattere dei 36 totali
  \item un carattere dei \textbf{35 possibili}
  \item \vdots
  \item un carattere dei 29 possibili
  \end{enumerate}
  Quindi $ |\Omega| = 36 \times 35 \times \dots \times 29 = \frac{36!}{28!} $
}

Definiamo il principio generale:

\thm{Non proprio un teorema}{
  Ciascun elemetno di un insieme $ A $ puo' essere determinato tramite sola sequenza di $ k $ scelte, dove per ogni scelta ci sono $ n_1,\dots,n_k $ possibilità, allora:
  \[
  |A| = n_1 \times \dots \times n_k
  \]
}

\nt{
  Puo' essere riscritto come teorema formale ma \textbf{E' TROPPO DIFFICILE PER NOI INFORMATICI} quindi non lo facciamo!!!!
}

\ex{}{
  52 carte (13 tipi 4 semi)
  \begin{enumerate}
    \item $ A = \text{iniseme dei full (un tris e una coppia)} $, $ |A| = ? $

      \underline{4 scelte:}
      \begin{itemize}
        \item tipo di carta nel tris (13)
        \item tipo di carta nella coppia (12)
        \item semi nel tris (4)
        \item semi nella coppia (6)
      \end{itemize}
      $ |A| = 13 12 4 6 = \text{casi favorevoli} $

    \item $ A = \text{doppie coppie (due coppie di tipo diverso e una carta libera)} $, $ |A|=? $

      \underline{Scelte}:
      \begin{itemize}
        \item tipo nella prima coppia (13)
        \item tipo nella seconda coppia (12)
        \item semi nella prima coppia (6)
        \item semi seconda (6)
        \item tipo singolo (11)
        \item seme singolo (4)
      \end{itemize}
      $ |A| = 13 12 6 6 11 4 $ \textbf{SBAGLIATO!!}

      Perche' non ci interessa dell'ordine delle due coppie (bisogna dedurlo dalla definizione di A), quindi NON c'e' una prima e seconda coppia (anche sopra era sbagliato vederlo cosi') dato che non c'e' l'ordine.

      Combinazioni dei tipi che compongono 2 coppie ($ 13 \times 12 / 2 $)
  \end{enumerate}
}

\section{Disposizioni}

\dfn{Disposizioni con ripetizione}{
  Dato un insieme $ E $ con $ n $ elementi, indichiamo con $ DR_{n,k} $ le sequenze ordinate di $ k $ elementi di $ E $. }
Sostanzialmente, $ DR_{n,k} = E \times E ... \times E $ $ k $ volte, ovvero:
\[
DR_{n,k} = E^k
\]
Quindi usando il principio delle scelte successive:
\[
|DR_{n,k}| = n^k
\]
dato che per ogni $ E $ abbiamo una scelta fra $ n $ elementi.

\nt{
  Indicando tale insieme con $ DR_{n,k} $, l'insieme $ E $ sparisce, dato che ci interessa solo la \textbf{cardinalita'} di tale insieme e non ce ne frega un cavolo dei sui elementi.
}

\ex{Iniziamo a calcolare le probabilita'!}{
  Presa un urna con $ n $ palline numerate ($ E = \{1,...,n\} $) e si estraggono $ k $ palline con \textit{reinbussolamento}.
  \[
  \Omega = E \times ... \times E = DR_{n,k}
  \]
  Quindi $ |\Omega| = n^k $, e la probabilita' uniforme e' data da:
  \[
    \mathbb{P}(\{\omega\}) = \frac{1}{|\Omega|} = n^{-k}
  \]
}
\ex{}{
  Determinare spazi campionari per i seguenti esperimenti:
  \begin{itemize}
  \item Scelta casuale di una parola di 8 lettere
  \item Scelta di colonne deltotocalcio (risultato di 13 partite)
  \end{itemize}
}

Quindi lo usiamo nei casi di estrazione con reinbussolamento quando ci interessa l'ordine.

\dfn{Disposizioni}{
  Dato un insieme $ E  $ di $ n $ elementi, l'insieme delle disposizioni (senza ripetizione) di $ k $ elementi dell'insieme $ E $ e' l'insieme delle sequenze ordinate di $ k $ elementi \textit{distinti}, ovvero:
  \[
    D_{n,k} \coloneq \{(x_1,...,x_k)|x_1,...,x_k \in E \land x_i \neq x_j \text{se} i \neq j \}
  \]
}

\nt{
  $ D_{n,k} $ e' un sottoinsieme \textbf{stretto} di $ DR_{n,k} $.
}

Usando le scelte successive:
\[
  |D_{n,k}| = n\times (n-1) \times ... \times (n-k+1) = \frac{n!}{(n-k)!}
\]
L'esperimento aleatorio di riferimento e' l'estrazione \textbf{senze} reimissione.

\dfn{Permutazioni}{
  $ P_n = D_{n,n} $, quindi:
  $ |P_n| = n! $
}

\section{Combinazioni}
\dfn{Combinazioni}{
  Dato un insieme $ E $ di $ n $ elementi, indichiamo con $ C_{n,k} $ la classe dei sottoinsiemi di $ E $ contenenti $ k $ elementi, ovvero:
  \[
  C_{n,k} = \{A \subseteq E \mid |A|= k\}
  \]
}
Siamo passati da sequenze a sottoinsiemi, ovvero non ci interessa piu' dell'ordine:
\[
\text{sottoinsieme} = \text{sequenza non ordinata}
\]



\mprop{sulla combinazione come coefficiente biniomiale}{
  Sia la combinazione $C_{n,k}$, allora vale:
  \begin{equation}
    |C_{n,k}| = \binom{n}{k}
  \end{equation}
}
\pf{Dimostrazione}{
  Mi riduco a dimostrare che:
  \[
    \frac{n!}{(n-k)!} = |C_{n,k}|k!
  \]
  Ovvero
  \[
    |D_{n,k}| = |C_{n,k}|\cdot |P_k|
  \]



  Ma $|D_{|n,k|} = |C_{n,k}|\cdot |P_k|$ vale per il principio delle scelte successive, infatti:
  \begin{itemize}
    \item scelta per i $k$ elementi selezionati: $|C_{n,k}|$
    \item scelta dell'ordinamento $|P_k|$
  \end{itemize}
}

Si noti questo esempio

\ex{}{
  Consideriamo un'unrna con $n$ palline distinte delle quali ne vengono estratte $k$. Determinare lo spazio di probabilità
  
  \hrule
  
  Si ha che: $\Omega = C_{n,k}$

  Mentre per il principio di probabilià uniforme si ha:
  \[
    \mathbb{P}(\{w\}) = \frac{1}{\binom{n}{k}} = \frac{k!(n-k)!}{n!}
  \]
}

Interpretazione combinatoria della formula di Stifel
\nt{
  Si consideri la formula di Stifel:
  \begin{equation}
    \binom{n}{k} = \binom{n-1}{k-1} + \binom{n-1}{k}
  \end{equation}
  Si può interpretare questa formula in termini di combinazioni come segue:

  \[
    \underbrace{\binom{n}{k}}_{\parbox{4cm}{n° di combinazioni di $k$ elementi di $E$}} = 
    \underbrace{\binom{n-1}{k-1}}_{\parbox{4cm}{n° di combinazioni di $k$ elementi di $E$ in cui è presente l'elemento $\bar{e}$}} + 
    \underbrace{\binom{n-1}{k}}_{\parbox{4cm}{n° di combinazioni di $k$ elementi di $E$ senza l'elemento $\bar{e}$}}
  \]
  dove $\bar{e}$ è un elemento fissato
}

Interpretazione combinatoria del binomio di Newton
\nt{
  Si consideri la formula del binomio di Newton:
  \begin{equation}
    (a+b)^n=\sum_{k=0}^n\binom{n}{k}a^kb^{n-1}\quad a,b\in \mathbb{R}
  \end{equation}

  Il prodotto $(a + b)(a + b)\dots(a + b)$ di $n$ fattori si sviluppa in una somma di monomi di grado $n$ del tipo $a^{n-k}b^k$ dive $k$ varia da 0 a $n$. Dobbiamo quindi determinare il coefficente di ogni monomio  $a^{n-k}b^k$, ossia calcolare quante volte compare facendo il prodotto $(a+b)(a+b)\dots (a+b)$ ossia calcolare quante volte compare facendo il prodotto $(a+b)(a+b)\dots (a+b)$. Questo monomio però si ottiene sciegliendo il valore $b$ da $k$ degli $n$ fattori e $a$ dai rimanenti $n-k$, ovvero in $\binom{n}{k}$ modi
}

\section{Tipologie di esperimenti}
Riassumendo, si possono verificare 3 casi notevoli durante un esperimento aleatorio inseriti in questa tabella
TODO:  NON FUNZIONA backslashbox, controllare
  
\begin{table}[h]
  \centering
    \begin{tabular}{|m{4cm}|m{5cm}|m{5cm}|}
    \hline
    \backslashbox{\textbf{ORDINE}}{\textbf{RIPETIZIONE}} & \textbf{Senza ripetizione} & \textbf{Con ripetizione} \\ 
    \hline
    Si tiene conto dell'ordine & 
    \begin{tabular}{@{}l@{}}
    Estrazione senza reimmissione \\
    $\Omega = \mathsf{D}_{n,k}$ \\
    $|\Omega| = \dfrac{n!}{(n-k)!}$
    \end{tabular} &
    \begin{tabular}{@{}l@{}}
    Estrazione con reimmissione \\
    $\Omega = \mathsf{DR}_{n,k}$ \\
    $|\Omega| = n^k$
    \end{tabular} \\ 
    \hline
    Non si tiene conto dell'ordine & 
    \begin{tabular}{@{}l@{}}
    Estrazione simultanea \\
    $\Omega = \mathsf{C}_{n,k}$ \\
    $|\Omega| = \dfrac{|\mathsf{D}_{n,k}|}{k!} = \binom{n}{k}$
    \end{tabular} & --- \\ 
    \hline
    \end{tabular}
    \caption{Gli spazi campionari e le loro cardinalità.}
  \label{tab:spazi_campionari}
\end{table}


\nt{
  La casella vuota nella tabella sopra riportata corrisponde all’insieme delle cosiddette \textit{combinazioni con ripetizione}, un raggruppamento di $k$ elementi estratti da $n$ elementi distinti, nel quale uno stesso elemento può ripetersi fino a $k$ volte, e in cui l'ordine di estrazione non è rilevante
}

Si noti questo esempio:
\ex{Esempio sulla casella mancante}{
  si lancia tre volte una moneta. Qual è la probabilità che esca due volte testa?

  \hrule

  Si hanno due approcci per scegliere lo spazio campionario:
  \begin{itemize}
    \item utilizzare le $DR_{2,3}$ con $\mathbb{P}$ uniforme.
    
    In questo caso:
  
  \[
    \Omega =DR_{2,k} = \{(T, T, T), (T, T, C), (T, C, T), (C, T, T), (T, C, C), (C, T, C), (C, C, T), (C, C, C)\}
  \]

  Ci sono 8 possibili sequenze perché ogni lancio ha 2 possibili risultati:
  \[
    2^3 = 8
  \]
  Siccome ogni sequenza ha la stessa probabilità di verificarsi, possiamo assegnare la probabilità uniforme, si ha

  \[
    \mathbb{P}(x_1,x_2,x_3) = \frac{1}{8} \quad \forall x 
  \]

  Sarà quindi facile calcolare la probabilità di $A$ (lascio al basta come esercizio)
  \item Lo spazio campionario con combinazioni con ripetizione $CR_{n,k}$.
  
  In questo caso, dato che non ci interessa l'ordine dei lanci, dobbiamo raggruppare le sequenze "uguali":

  \[
    CR_{2,3} = \{[T,T,T],[T,T,C],[T,C,C],[C,C,C]\}
  \]

  Tuttavia è facile verificare che in questo caso la probabilità uniforma \textbf{non sussiste}, perché alcuni gruppi contengono più sequenze di altri

  infatti:

  \begin{itemize}
    \item $\mathbb{P}([T, T, T]) = \P((T, T, T)) =\frac{1}{8}$
    \item  $\P([T, T, C]) = \P((T, T, C)) + \P((T, C, T)) + \P((C, T, T))=\frac{3}{8}$
    \item $\P([T, C, C]) = \P((T, C, C)) + \P((C, T, C)) + \P((C, C, T)) =\frac{3}{8}$
    \item $\mathbb{P}([C,C,C]) = \P((C,C,C)) =\frac{1}{8}$
  \end{itemize}
\end{itemize}
}

\nt{
  Se scegliamo lo spazio campionario $CR_{n,k}$, dobbiamo lavorare con probabilità non uniformi, il che complica i calcoli.Invece, se lavoriamo con , tutte le sequenze hanno la stessa probabilità e possiamo usare il calcolo combinatorio standard per trovare i risultati.

  Per questo motivo, nella maggior parte dei problemi in cui c’è ripetizione, conviene lavorare con disposizioni con ripetizione , perché permette di usare la probabilità uniforme.
}

Bho adesso si fa un altro esempio, e va benissimo così

\ex{Probabilità binomiale}{
  Si consideri un’urna che contiene $b$ palline bianche ed $r$ palline rosse. Si effettuano $k$ estrazioni con reimmissione. Calcolare la probabilità dell’evento

  \[
    A_\ell = \text{"si estraggono } \ell \text{ palline bianche ed } k - \ell \text{ palline rosse''}
  \]
}

\pf{soluzione}{
  Etichettiamo le $b$ palline bianche con $b_1,b_2, \dots, b_b$; analogamente, le $r$ palline rosse con $r_1,\dots , r_r$. Sia dunque:
  \[
    E = \{b_1,\dots, b_b,r_1,\dots , r_r\}
  \]

  E si noti che $|E| = b+r$.

  Come spazio di probabilità $(\Omega,\P)$ è naturale considerare $\Omega = DR_{b+r, k}$ (insieme delle disposizioni con ripetizione di $k$ elementi di $E$) e $\P$ probabilità uniforme, quindi $\P(\{w\})=\frac{1}{(b+r)^n}$

  Per calcolare $\P(A_\ell)$ dobbiamo calcolare $|A_\ell|$ che si rivela possibile usando il principio delle scelte successive:
  \begin{itemize}
    \item scelta della sequenza di $\ell$ palline bianche estratte, ovvero $|DR_{b,\ell}|$ possibilità
    \item scelta della sequenza di $k-\ell$ palline bianche estratte, ovvero $|DR_{b,k-\ell }|$ possibilità
    \item scelta delle $\ell$ estrazioni in cui sono uscite le palline bianche: $|C_{k,\ell}|$ possibilità
  \end{itemize}

  In defintiva:
  \[
    \P(A_\ell) = \frac{|DR_{b,\ell}||DR_{b,k-\ell }||C_{k,\ell}|}{| DR_{b+r, k}|} = \binom{k}{\ell}\frac{b^\ell r^{k-\ell}}{(b+r)^k}
  \]
  O, equivalentemente, dato $p=\frac{b}{b+r}$ si ha:
  \[
    \P(A_\ell) = \binom{k}{\ell}p^\ell (1-p)^{k-\ell}, \quad \ell = 0,\dots, k
  \]
}

\nt{
  Consideriamo lo spazio di probabilità $(\Omega, \mathbb{P})$ con $\Omega = \{0, 1, 2, ..., k\}$ e $\mathbb{P}$ data da
\[
\mathbb{P}(\{\ell\}) = \binom{k}{\ell} p^\ell (1-p)^{k-\ell}, \qquad \ell = 0, 1, ..., k.
\]

Si noti che $\mathbb{P}$ è effettivamente una probabilità. Infatti, per la formula del binomio di Newton vale che
\[
\sum_{\ell=0}^k \mathbb{P}(\{\ell\}) = \sum_{\ell=0}^k \binom{k}{\ell} p^\ell (1-p)^{k-\ell} = (p + (1-p))^k = 1.
\]

$\mathbb{P}$ non è una probabilità uniforme. $\mathbb{P}$ si chiama \textbf{probabilità binomiale}.





}