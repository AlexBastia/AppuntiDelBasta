\documentclass{report}
\usepackage{amssymb}
\usepackage{amsmath} % Per le formule matematiche
\usepackage{tikz}    % Per i diagrammi

\input{../LatexTemp/preamble}
\input{../LatexTemp/macros}
\input{../LatexTemp/letterfonts}

\newcommand{\dperp}{\mathrel{\bot\!\!\!\bot}} % Independent events

\title{\Huge{Probabilità}\\Appunti}
\author{\huge{Giovanni Palma e Alex Basta}}
\date{}
\pagenumbering{gobble}

\graphicspath{{paginette/img/}}

\begin{document}

\maketitle
\newpage% or \cleardoublepage
% \pdfbookmark[<level>]{<title>}{<dest>}
\pdfbookmark[section]{\contentsname}{toc}




\tableofcontents

\pagebreak
\section{Es 7}
Si consideri una popolazione in cui una persona su 100 abbia una certa malattia. Un test
`e disponibile per diagnosticare tale malattia. Si supponga che il test non sia perfetto, in quanto esso
risulta positivo (ovvero indica la presenza della malattia) nel 5\% dei casi quando `e effettuato su persone
sane, mentre risulta negativo (indicando l’assenza della malattia) nel 2\% dei casi quando `e effettuato su
persone malate. Si calcolino le probabilit`a che
(a) il test risulti positivo quando effettuato su una persona malata,
(b) il test risulti positivo,
(c) una persona sia malata se il test risulta positivo.

\begin{itemize}
  \item b) $ P(+) = P(+ \cap M) + P(+ \cap S) = P(+|M)P(M) + P(+|S)P(S) $ (Probabilita' totali e regola catena)
  \item c) $ P(M|+) = \frac{P(+|M)P(M)}{P(+)} $ (Bayes)
\end{itemize}

\section{Es 5}
Ci sono quattro dadi: due non truccati, i rimanenti invece sono truccati in quanto hanno
tre facce che indicano il numero 6 e le altre tre il numero 5. Si lancia una moneta (non truccata). Se
viene testa si lanciano i primi due dadi, mentre se viene croce si lanciano i dadi truccati.
(a) Calcolare la probabilit`a che la somma dei due dadi sia 11.
(b) Sapendo di aver ottenuto un 11 lanciando i due dadi, calcolare la probabilit`a di aver ottenuto croce
lanciando la moneta.

\begin{itemize}
  \item a) $ P(S_{11}) = P(S_{11}|T)P(T) + P(S_{11}|C)P(C) = 0.5P(\{(5,6), (6,5)|T\}) + 0.5P(\{(5,6), (6,5)\}) $ 
  \item b) $ P(C|S_{11}) = \frac{P(S_{11}|C)P(C)}{P(S_{11})} $
\end{itemize}

\section{Es 2}
Si consideri l’esperimento di lanciare due volte un dado.
(a) Determinare uno spazio di probabilit`a che descriva l’esperimento aleatorio.
(b) Si considerino i seguenti eventi:
• A = “numero dispari sul primo dado”,
• B = “numero dispari sul secondo dado”,
• C = “la somma dei due risultati `e dispari”.
Gli eventi A, B e C sono indipendenti?
(c) Si considerino ora gli eventi:
• E = “il risultato del secondo lancio `e 1, 2 o 5”,
• F = “il risultato del secondo lancio `e 4, 5 o 6”,
• G = “la somma dei due risultati `e 9”.
Gli eventi E, F e G sono indipendenti?

\begin{itemize}
  \item a)  $ (\Omega, P) =  $
  \item b) Controllo che i tre siano indipendenti guardando i prodotti delle probabilita' e vedendo se sono uguali alla probabilita' dell'intersezione.
  \item c) 
\end{itemize}

\section{Es 3}
I componenti prodotti da una ditta possono avere due tipi di difetti con percentuali del 3\% e
del 7\% rispettivamente e in modo indipendente l’uno dall’altro. Qual `e la probabilit`a che un componente
scelto a caso
(a) presenti entrambi i difetti?
(b) sia difettoso?
(c) presenti il primo difetto, sapendo che `e difettoso?
(d) presenti uno solo dei difetti, sapendo che `e difettoso?

\begin{itemize}
  \item a) $ P(D_1 \cap D_2) = P(D_1)P(D_2) $ (eventi indipendenti)
  \item b) $ P(D_1 \cup D_2) = 1 - P(D_1^{c} \cap D_2^{c}) = 1 - P(D_1^{c})P(D_2^{c}) $ (complementari di eventi indipendenti sono indipendenti)
  \item c) $ P(D_1 | D_1 \cup D_2) =  $ (Bayes)
  \item d) $ P((D_1 \cap D_2^{c}) \cup (D_2 \cap D_1^{c}) | D_1 \cup D_2) $
\end{itemize}

\section{}
$ A = \text{clienti insolventi verso meccanico}, B = \text{l'aereo del cliente viene distrutto dal meccanico}, C = \text{l'aereo viene distrutto non dal meccanico} $ e quindi $ P(B|A) = 1/1000 $.

Quali fra A,B e C sono indipendenti? 

$ D = \text{l'aereo viene distrutto} = B \cup C $ 

$ P(B|A \cap D) = \frac{P(A \cap B \cap D)}{P(A \cap D)} = \frac{P(B \cap A \cap (B \cup C))}{P(A \cap (B \cup C))} $

$ P(D) = 1/10000 $, dai un limite inferiore (controllo dal basso??) 

$ C \subseteq D = B \cup C $, quindi per monotonia $ P(C) \leq P(D) $ quindi
\[
  P(B|A) + P(C) \leq P(B|A) + P(D) \implies \frac{P(B|A)}{P(B|A)+P(C)} \geq \frac{P(B|A)}{P(B|A) + P(D)}
\]

\end{document}
