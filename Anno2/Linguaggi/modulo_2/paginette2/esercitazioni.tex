\chapter{Esercitazioni}
\section{Scope}
\subsection{Esercizio 1}
\subsubsection{Testo}
\begin{lstlisting}[language=C]
    {
        int x =2;
        int func ( int y){
            x = x+y;
            write(x);
        }
        {
            int x = 5;
            func(x);
            write(x);
        }
        write(x);
    }
\end{lstlisting}

\begin{itemize}
    \item Si descriva il comportamento del programma assumendo uno scope statico.
    \item Si descriva il comportamento del programma assumendo uno scope dinamico.
\end{itemize}

\subsubsection{Soluzione}
\begin{itemize}
    \item \textbf{Scope statico}: il programma stampa 7, 5, 7
    \item \textbf{Scope dinamico}: il programma stampa 10, 10, 2
\end{itemize}
\subsection{Esercizio 2}
\subsubsection{Testo}
\begin{lstlisting}[language=C]
    {
        int x = 0;
        int next () {
            x = x+1;
            write(x);
        }
        int exec(){
            int x = 3;
            next();
            write(x);
        }
        exec();
        write(x);
    }
\end{lstlisting}
\begin{itemize}
    \item Si descriva il comportamento del programma assumendo uno scope statico
    \item Si descriva il comportamento del programma assumendo uno scope dinamico
\end{itemize}
\subsubsection{Soluzione}
\begin{itemize}
    \item \textbf{Scope statico}: il programma stampa 1, 3, 1
    \item \textbf{Scope dinamico}: il programma stampa 4 infatti \texttt{next} prenderà l'istanza di x dell'unltimo ambiente non disattivato, ovvero exec, modifico poi il valore di \texttt{x} nell'ambiente exec, 4 dato che l'amiente di exec è stato modificato, 0 dato che \texttt{exec} è stato disattivato si ha che \texttt{x = 0}
\end{itemize}

\subsection{Esercizio 3}
\subsubsection{Testo}
\begin{lstlisting}[language=C]
    {
        int x = 0;
        void pippo(value int y, rif int z){
            z= x+y+z;
        }
        {
            int x = 1;
            int y = 100;
            int z = 30;
            pippo(x++,x); // ricordarsi che il x++ prima passa il valore di x nudo e crudo poi come side-effect modica x staticamente in questo caso, pertanto il secondo parametro sarà 2: pippo(1, 2) -> 3 = 0+ 1 +2
            pippo(x++, x); // pippo(3, 4) -> 7 = 0 + 3 + 4
            write(x); // 7
        }
        write(x); // 0 perché il blocco è finito, terminato e riprende la variabile x dichiarata all'inizio
    }
\end{lstlisting}
Si descriva il comportamento del programma assumendo uno scope statico

\subsubsection{Soluzione}
Il programma stampa 7, 0

\subsection{Esercizio 4}
\subsubsection{Testo}
\begin{lstlisting}[language=C]
    {
        void pippo(value int y, value int z){
            x=y+z;
        }
        int x = 100;
        pippo(x++, x); // pippo(100, 101) -> 201, quindi x = 201
        pippo(x++, x); // pippo(201, 202) -> 403, quindi x = 403
        write(x); // 403
    }
\end{lstlisting}
Si descriva il comportamento del programma assumendo uno scope dinamico

\subsubsection{Soluzione}
Il programma stampa 403

\subsection{Esercizio 5}
\subsubsection{Testo}
\begin{lstlisting}[language=C]
{  
    int f(value int k){ //1. 2, 1. 1
        int g (value int n){
            return n+y //
        };
        int x=10;
        int y=10;
        if k=1 return g(x) + g(y); // 40
        else {
            int x = 30; 
            int y= 30;
            return f(k-1);
        }
    }
    int x =50;
    int y=50;
    x= f(2); // x = 40
    write(x); // 40
}
\end{lstlisting}
si risolva utilizzando uno scope dinamico

\subsubsection{Soluzione}
Il programma stampa 40

\subsection{Esercizio 6}
\subsubsection{Testo}
\begin{lstlisting}[language=C]
{
    int x =10;
    int v =5;
    void B(){
        write(x);
    }
    void A(z){
        int x = z * v;
        B();
    }
    A(??) // si ha che qualsiasi valore di al posto di ?? si scriverà 10
}
\end{lstlisting}
Fornire una chiamata alla funzione A di modo che il prgramma, usando scope statico stami il valore 10
\begin{lstlisting}[language=C]
    {
        int x =10;
        int v =5;
        void B(){
            write(x);
        }
        void A(z){
            int x = z * v;
            B();
        }
        A(??) // si scrive 2 cazzo
    }
\end{lstlisting}

fornire una chiamata per stampare il valore 10, ma usando uno scope dinamico

\subsubsection{Soluzione}
PEr la prima parte si stampera sempre 10, per la seconda parte si passa il parametro 2

\subsection{Esercizio 7}
\subsubsection{Testo}
\begin{lstlisting}[language=C]
{
    int x =1;
    int y=2;
    void A(){
        int x = 2;
        int k = 3;
    }
    void B(){
        int x = 5; // 2
        A();
        // ** 1**
        x=2;
        C():
    }
    void C(){
        int z =5;
        // ** 2 **
    }
    B();
}
\end{lstlisting}
SI descriva qual è il contenuto delle variabili attive al punto **1** e **2**, utilizzando scope statico e dinamico

\subsubsection{Soluzione}
\begin{itemize}
    \item \textbf{Scope statico}:
    \begin{itemize}
        \item \textbf{Punto 1}: \texttt{x = 5}, \texttt{y = 2}
        \item \textbf{Punto 2}: \texttt{x = 1}, \texttt{y = 2}, \texttt{z = 5}
    \end{itemize}
    \item \textbf{Scope dinamico}:
    \begin{itemize}
        \item \textbf{Punto 1}: \texttt{x = 5}: nel momento in cui si attiva \texttt{A()} lo shadowing impone che che x in memoria è 2, ma appena A termina si "disattiva" e si riprende x=5 definito nel punto B, \texttt{y = 2}
        \item \textbf{Punto 2}: \texttt{x = 2}, \texttt{y = 2}, \texttt{z = 5}
    \end{itemize}
\end{itemize}