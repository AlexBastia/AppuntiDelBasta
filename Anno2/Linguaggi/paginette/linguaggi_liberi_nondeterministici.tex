\chapter{Linguaggi liberi nondeterministici}
Fino ad ora abbiamo studiato due modi equivalenti per descrivere linguaggi: gli \textbf{automi finiti} e le \textbf{espressioni regolari}. Abbiamo anche visto le limitazioni di questi metodi, che non riescono ad accettare linguaggi, anche alcuni semplici come $ \{0^n 1^n \mid n \geq 0\} $. 

Per espandere la quantita' di linguaggi che possiamo studiare, introduciamo le cosidette \textbf{grammatiche libere (dal contesto)}, un metodo piu' potente che permette di descrivere certe proprieta' di linguaggi che hanno una struttura ricorsiva. Inizialmente queste grammatiche sono state studiate per analizzare linguaggi umani, ma piu' recentemente hanno visto svariate applicazioni, come nel caso dei \textbf{parser}, utilizzati dalla maggior parte di compilatori e interpreti per estraggono il significato dal codice. 

L'insieme dei linguaggi che possono essere riconosciute da grammatiche libere sono i \textbf{linguaggi liberi (dal contesto)}, che includono i linguaggi regolari e 
