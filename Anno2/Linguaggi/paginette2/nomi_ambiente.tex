% \begin{document}
\chapter{Nomi e Ambiente}

Nell'evoluzione dei linguaggi di programmazione, i \textit{nomi} hanno avuto un ruolo fondamentale nella sempre maggiore astrazione rispetto al linguaggio macchina. 

\dfn{Nome}{
  I nomi sono solo una sequenza (significativa o meno) di caratteri che sono usati per rappresentare un oggetto, che puo' essere uno spazio di memoria se vogliamo etichettare dei dati, o un insieme di comandi nel caso di una funzione. 
}

\section{Nomi e Oggetti denotabili}

Spesso, i nomi sono \textit{identificatori}, ovvero token alfanumerici, ma possono essere usati anche simboli (+,-,...). E' importante ricordare che il nome e l'oggetto denotato non sono la stessa cosa, infatti un oggetto puo' avere diversi nomi (\textit{aliasing}) e lo stesso nome puo' essere attribuito a diversi oggetti in momenti diversi (\textit{attivazione} e \textit{deattivazione}).  

\dfn{Oggetti denotabili}{
  Sono gli oggetti a cui e' possibile attribuire un nome.
}
\nt{Non centra con la programmazione ad oggetti}

Possono essere:
\begin{itemize}
\item Predefiniti: tipi e operazioni primitivi, ...
  \item Definibili dall'utente: variabili, procedure, ...
\end{itemize}

Quindi il legame fra nome e oggetto (chiamato \textbf{binding}) puo' avvenire in momenti diversi:
\begin{itemize}
\item Statico: prima dell'esecuzione del programma
\item Dinamico: durante l'esecuzoine del programma
\end{itemize}

\section{Ambienti e Blocchi}

Non tutti i legami fra nomi e oggetti vengono creati all'inizio del programma restando immutati fino alla fine. Per capire come i binding si comportano, occorre introdurre il concetto di \textit{ambiente}:

\dfn{Ambiente}{
  Insieme di associazioni nome/oggetto denotabile che esistono a runtime in un punto specifico del programma ad un momento specifico durante l'esecuzione.
}

Solitamente nell'ambiente non vengono considerati i legami predefiniti dal linguaggio, ma solo quelli creati dal programmatore utilizzando le \textit{dichiarazioni}, costrutti che permettono di aggiungere un nuovo binding nell'ambiente corrente.

Notare che e' possibile che nomi diversi possano denotare lo stesso oggetto. Questo fenomeno e' detto \textit{aliasing} e succede spesso quando si lavora con puntatori.

\subsection{Blocchi}

Tutti i linguaggi di programmazione importanti al giorno d'oggi utilizzano i \textit{blocchi}, strutture introdotte da ALGOL 60 che servono per strutturare e organizzare l'ambiente:

\dfn{Blocco}{
  Pezzo contiguo del programma delimitato da un inizio e una fine che puo' contenere dichiarazioni \textbf{locali} a quella regione.
}

Puo' essere:
\begin{itemize}
  \item In-line (o anonimo): puo' apparire in generale in qualunque punto nel programma e non corrisponde a una procedura. 
\item Associato a una procedura 
\end{itemize}

Permettono di strutturare e riutilizzare il codice, oltre a ottimizzare l'occupazione di memoria e rendere possibile la ricorsione. 

\subsection{Tipi di Ambiente}

Un'altro meccanismo importante che forniscono i blocchi e' il loro \textit{annidameno}, ovvero l'inclusione di un blocco all'interno di un altro (non la sovrapposizione parziale). In questo caso, se i nomi locali del blocco esterno sono presenti nell'ambiente del blocco interno, si dice che i nomi sono \textit{visibili}. Le regole che determinano se un nome e' visibile o meno a un blocco si chiamano \textit{regole di visibilita'} e sono in generale:

\begin{itemize}
\item Un nome locale di un blocco e' visibile a esso e a tutti i blocchi annidati.
  \item Se in un blocco annidato viene creata una nuova dichiarazione con lo stesso nome, questa ridefinizione \textit{nasconde} quella precedente.
\end{itemize}

\dfn{Ambiente associato a un blocco}{
L'ambiente di un blocco e' diviso in:
\begin{itemize}
\item locale: associazioni create all'ingresso nel blocco:
  \begin{itemize}
  \item variabili locali
  \item parametri formali (nel caso di un blocco associato a una procedura)
  \end{itemize}
\item non locale: associazioni ereditate da altri blocchi (senza considerare il blocco globale), che quindi non sono state dichiarate nel blocco corrente
\item globale: associazioni definite nel blocco globale (visibile a tutti gli altri blocchi)
\end{itemize}
}

\subsection{Operazioni sull' ambiente}
\begin{itemize}
\item Creazione: dichiarazione locale
\item Riferimento: uso di un nome di un oggetto denotato
\item Disattivazione/Riattivazione: quando viene ridefinito un certo nome, all'interno del blocco viene disattivato. Quando esco dal blocco riattivo la deinizione originale
\item Distruzione: le associazioni locali del blocco da qui si esce vengono distrutte
\end{itemize}

\nt{
  Creazione e distruzione di un \textit{oggetto denotato} non coincide necessariamente con la creazione o distruzione dei legami per esso.
}

\section{Regole di scope}

TODO: esempio scope statico e dinamico

statico:
Il nome non locale e' risolto (dal punto di vista del riferimento) e' la prima dichiarazione di x che trovo andando verso l'esterno.

dinamico: 
vado indietro \textit{nell'escecuzione} per cercare l'occerrenza ch ci interressa (e' l'ultima che e' stata introdotta) blocco attivato per ultimo (che deve essere ancora attivo)

notare che se cambio il nome di una variabile locale, con scope statico la semantica non cambia, ma con scope dinamico puo' cambiare (esempio p.24) -> collegamento con lo shadowing logica

L'ambiente e' quindi determinato da:
\begin{itemize}
\item Regole di scope
\item Regole di visibilita'
\item Regole di binding (solo quando posso passare funzioni come parametri)
\item Regole per il passaggio di parametri
\end{itemize}

% \end{document}
