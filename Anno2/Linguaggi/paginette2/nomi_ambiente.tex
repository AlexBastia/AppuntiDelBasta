\begin{document}
\chapter{Nomi e Ambiente}

\section{Nomi e Oggetti denotabili}

x e' solo il nome della variabile, non la variabile stessa. 

\dfn{Oggetti denotabili}{
  Sono gli elementi a cui e' possibile assegnare un nome
}
\nt{Non centra con la programmazione ad oggetti}

Possono essere:
\begin{itemize}
\item Predefiniti: int,...
  \item Definibili: variabili, ...
\end{itemize}

chiamiamo il legame fra nome e oggetto \textbf{binding}, il momento in cui viene creato questo legame puo' essere:
\begin{itemize}
\item Statico: prima dell'esecuzione del programma
\item Dinamico: durante l'esecuzoine del programma
\end{itemize}

\section{Ambienti e Blocchi}

\dfn{Ambiente}{
  Insieme di associazioni nome/oggetto denotabile che esistono durante l'esecuzione del programma
}

\dfn{Blocco}{
  Pezzo contiguo del programma delimitato da un inizio e una fine che puo' contenere dichiarazioni \textbf{locali} a quella regione.
}

Puo' essere:
\begin{itemize}
\item Anonimo: come parentesi graffe in C 
\item Con nome: ad esempio nelle funzioni, dove il nome della funzione corrisponde al nome del blocco
\end{itemize}

Permettono di strutturare e riutilizzare il codice, oltre a ottimizzare l'occupazione di memoria e rendere possibile la ricorsione. 

L'ambiente di un blocco e' diviso in:
\begin{itemize}
\item locale: associazioni create all'ingresso nel blocco:
  \begin{itemize}
  \item variabili locali
  \item parametri formali
  \end{itemize}
\item non locale: associazioni ereditate da altri blocchi (senza considerare il blocco globale), che quindi non sono state dichiarate nel blocco corrente
\item globale: associazioni definite nel blocco globale (visibile a tutti gli altri blocchi)
\end{itemize}

\subsection{Operazioni sull' ambiente}
\begin{itemize}
\item Creazione: dichiarazione locale
\item Riferimento: uso di un nome di un oggetto denotato
\item Disattivazione/Riattivazione: quando viene ridefinito un certo nome, all'interno del blocco viene disattivato. Quando esco dal blocco riattivo la deinizione originale
\item Distruzione: le associazioni locali del blocco da qui si esce vengono distrutte
\end{itemize}

\nt{
  Creazione e distruzione di un \textbf{oggetto denotato} non coincide necessariamente con la creazione o distruzione dei legami per esso.
}

\end{document}
