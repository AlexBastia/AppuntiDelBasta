\documentclass{report}

\input{../LatexTemp/preamble.tex}
\input{../LatexTemp/macros}
\input{../LatexTemp/letterfonts}

\usepackage[utf8]{inputenc}

\setlength{\parindent}{0pt}

\title{\Huge{Ottimizzazione Combinatoria}\\Appunti}
\author{\huge{Alex Bastianini}}
\date{}
\pagenumbering{gobble}

\begin{document}

\maketitle
\newpage% or \cleardoublepage
% \pdfbookmark[<level>]{<title>}{<dest>}
\pdfbookmark[section]{\contentsname}{toc}
\tableofcontents

\pagebreak

\chapter{Introduzione}
Prova scritta e orale. Si studiano metodi algoritmici per ottimizzare problemi di flusso su reti e di programmazione lineare. In poche parole, impariamo come prendere decisioni.

\chapter{Problemi e Modelli}
\section{Problemi di ottimizzazione}

Processo decisionale:
\begin{itemize}
\item Individuazione problema
  \item Raccolta dati
    \item Costruzione modello
      \item Determinazione di piu' soluzioni
        \item Analisi dei risultati
\end{itemize}

La costruzione del modello e' la fase piu' iportante, che ci permette di utilizzare tecniche note per risolvere problemi che possono apparire piu' complessi. Non ci interessa la raccolta dei dati. 

I modelli ci permettono di inquadrare i problemi in una determinata "cornice" che ci permette di determinare quale tipo di algoritmo risolutivo usare Esistono tre tipi di modelli:
\begin{itemize}
\item Giochi: ricerca di un equilibrio fra le componenti coinvolte, spesso con obbiettivi contrastanti. (non ce ne occupiamo)
\item Simulazione: il problema viene studiato simulando la situazione senza studiarne la natura in modo analitico tramite generazione di istanze casuali. (anche questi modelli non ci interessano)
\item Analitici: costruiamo un modello matematico (senza perdere informazione sul problema reale) da risolvere in modo analitico, ovvero senza nessun tipo di simulaizone. E' la natura stessa dello spaizo matematico a dare la soluzione ottima. E' vantaggiosa in quanto la soluzione e' sicuramente quella ottima. E' essenziale l'uso della creativita'
\end{itemize}

Cos'e' un problema? E' una domanda la cui rispostadipende da parametri e variabili, sopratutto nei problemi analitici. Un problema $ \mathcal{P} $ e' descritto tramite:
\begin{itemize}
\item I suoi parametri e variabili
  \item Le caratteristiche che una soluzione deve avere
\end{itemize}

Quado fissiamo un'istanza di un problema, vengono fissati i parametri ma non le variabili, che sono le incognite che devono essere definite. Distinguiamo un problema dalla sua istanza per generalizzarlo. Attenzione che molti si confondono. 

Un modo comune per descrivere un problema e' dare l'insieme di soluzioni ammissibili $ F_{\mathcal{P}} \subseteq G $ definito dando dei vincoli che un generico $ g \in G $ deve soddisfare. Dove G e' un sovrainsieme.

\end{document}
