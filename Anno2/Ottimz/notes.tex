\documentclass{report}

\input{../LatexTemp/preamble.tex}
\input{../LatexTemp/macros}
\input{../LatexTemp/letterfonts}

\usepackage[utf8]{inputenc}

\setlength{\parindent}{0pt}

\title{\Huge{Ottimizzazione Combinatoria}\\Appunti}
\author{\huge{Alex Bastianini}}
\date{}
\pagenumbering{gobble}

\begin{document}

\maketitle
\newpage% or \cleardoublepage
% \pdfbookmark[<level>]{<title>}{<dest>}
\pdfbookmark[section]{\contentsname}{toc}
\tableofcontents

\pagebreak

\chapter{Introduzione}
Prova scritta e orale. Si studiano metodi algoritmici per ottimizzare problemi di flusso su reti e di programmazione lineare. In poche parole, impariamo come prendere decisioni.

\chapter{Problemi e Modelli}

\section{Problemi di ottimizzazione}

\dfn{Ricerca operativa}{
  La \textbf{ricerca operativa} è un ramo della matematica applicata che si occupa dello studio, della modellizzazione e della risoluzione dei cosiddetti \textit{problemi decisionali} complessi mediante strumenti matematici, algoritmici e computazionali, con l'obiettivo di ottimizzare processi e risorse
}

I problemi di cui si occupa la ricerca operativa, quindi, riguardano situazioni in cui occorra massimizzare i ricavi o minimizzare i costi, in presenza di risorse limitate. Detto in termini più matematici, data una funzione \textbf{vincolata} l'obiettivo è trovare una soluzione ottimale che massimizzi o minimizzi tale funzione.

È pertanto vero, quindi, che questa disciplina ha forte contenuto economico

La ricerca operativa si inserisce all'interno del processo decisionale, il quale può essere suddiviso in diverse fasi
\begin{itemize}
\item \textbf{Individuazione problema}
  \item \textbf{Raccolta dati}
    \item \textbf{Costruzione modello}, ovvero la Traduzione del problema in un modello matematico che descriva il sistema e i vincoli in modo formale
      \item \textbf{Determinazione di piu' soluzioni}: applicazione di algoritmi e tecniche di ottimizzazione per individuare la soluzione migliore 
  \item \textbf{Analisi dei risultati}
\end{itemize}

La ricerca operativa, quindi, si occupa delle fasi 3 e 4 del processo, dato che sono le fasi che richiedono l’impiego di modelli matematici, algoritmi di ottimizzazione e strumenti computazionali. Adesso andiamo a definire per benino che cosa intendiamo per "modello" 

\dfn{modello}{
  un \textbf{modello} è una descrizione astratta e scritta in linguaggio matematico, della parte di realtà utile al processo decisionale
}
I modelli ci permettono di inquadrare i problemi in una determinata "cornice" che ci permette di determinare quale tipo di algoritmo risolutivo usare.

Esistono tre tipi di modelli:
\begin{itemize}
\item \textbf{Teoria dei giochi}: ricerca di un equilibrio fra le componenti coinvolte in un'interazione reciproca, spesso con obbiettivi contrastanti. (non ce ne occupiamo)
\item \textbf{Simulazione}: il problema viene studiato simulando la situazione senza studiarne la natura in modo analitico tramite generazione di istanze casuali. (anche questi modelli non ci interessano)
\item \textbf{Analitici}: dal problema si costruisce un modello matematico rigoroso (senza perdere informazione sul problema reale) e risolto mediante tecniche analitiche, senza ricorrere a simulazioni. La natura stessa dello spazio matematico in cui è inserito il problema è in grado di garantire la soluzione ottima. Questo tipo approccio è particolarmente vantaggioso in quanto assicura l’esattezza della soluzione supponendo che il modello sia formulato correttamente. 

È tuttavia richiesto un discreto livello di creatività
\end{itemize}

Definiamo, adesso, i problemi che andiamo a trattare

\dfn{Problema}{
  Definiamo \textbf{problema} una domanda, espressa in termini generali, la cui risposta dipende da \textit{parametri} e \textit{variabili}, sopratutto nei problemi analitici
}
Un problema $ \mathcal{P} $ è descritto tramite:
\begin{itemize}
  \item I suoi parametri e variabili
  \item Le caratteristiche che una soluzione deve avere
\end{itemize}

Quando fissiamo un'istanza di un problema, vengono fissati i parametri ma non le variabili, che sono le incognite che devono essere definite. Distinguiamo un problema dalla sua istanza per generalizzarlo. Si presti attenzione alla differenza tra parametri e variabili che molti si confondono

\ex{Problema con paramteri e variabili}{
  Sia $ \mathcal{P} $ il seguente problema
  \[
    ax^2+bx+c =0
  \]
  Dove $a,b$ e $c$ sono i suoi parametri e $x$ rappresenta le variabili, una possibile istanza di tale problema è:
  \[
    5x^2-6x+1=0
  \] 
}
Un modo comune per descrivere un problema è dare l'insieme di soluzioni ammissibili $ \mathbb{F}_{\mathcal{P}} \subseteq G $, dove $G$ è un sovrainsieme generico noto, di solito contenente la collezione di tutte le possibili configurazioni o decisioni che si possono prendere, dando dei vincoli che un generico $ g \in G $ deve soddisfare per far parte di $\mathbb{F}_{\mathcal{P}}$, avremo così che $G - \mathbb{F}_{\mathcal{P}}$ è l'insieme delle soluzioni non ammissibili 
\ex{}{
  Sia l'instanza di $\mathcal{P}$ definita precedentemente
  \[
    5x^s - 6x+1= 0
  \]
  si ha che 
  \[
    \begin{array}{l}
      \mathbb{G}= \mathbb{R}\\
      \mathbb{F}_{\mathcal{P}} = \{x\in \mathbb{R} | 5x^2-6x+1=0\}
    \end{array}
  \]
}

\end{document}
