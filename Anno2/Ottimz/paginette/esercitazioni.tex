\chapter{Esercitazioni}

\section{Problemi e modelli}
\subsection{Problema dei data center}

\subsubsection{Testo}
Abbiamo $n$ server $1,\dots,n$, ogni server può lavorare in $m_{i\in\mathbb{N}}$ modalità operative diverse. Nella modalità $j\in\{1,\dots,m\}$, il server 1 riesce ad eseguire un numero di istruzioni..

\hspace{1.5cm}

\subsubsection{Svolgimento}

\textit{Variabili}:
\[
    \begin{aligned}
        x_{ij} &= \begin{cases}
            1 & \text{se il server $i$ è utilizzato in modalità $j\in\{1,\dots,m_i\}$} \\
            0 & \text{altrimenti}
        \end{cases}\\
        x_{ij}&\in \mathbb{N}, \quad \forall i\,\forall j
    \end{aligned}
\]
\textit{Vincoli}:
$\forall i\in \{1,\dots, n\},\quad\forall j\in\{1,\dots,m_{i}\}$ con $0\leq x_{ij}\leq 1$ si ha:
\[
    \begin{aligned}
        \forall i\in\{1,\dots,n\}\quad \sum_{j=1}^{m_i}x_{ij} &= 1\\
        \sum^n_{i=1}\sum^{m_i}_{j=1}x_{ij}s_{ij} &\geq k
    \end{aligned}
\]

\textit{funzione obbiettivo}

\[
    \min \sum^n_{i=1}\sum^{m_i}_{j=1}x_{ij}w_{ij}
\]

\subsection{Problema del docente di informatica}
\subsubsection{Testo}
Si hanno dei progetti ($t_1, t_2, \dots, t_n$), si hanno $m$ PC, dove ogni pc può compilare qualunque progetto in modalità sequenziale. Le prestazioni del PC sono identiche 

Vogliamo assegnare i progetti ai pc in modo da minimizzare il tempo complessivo parallelo di compilazione
\subsubsection{Svolgimento}
\textit{Variabili}:
\[
    \begin{aligned}
        x_{ij} &= \begin{cases}
            1 & \text{se il progetto $i$ è compilato al pc $j$} \\
            0 & \text{altrimenti}
        \end{cases}\\
        x_{ij}&\in \mathbb{N}, \quad \forall i\in \{1,\dots, n\}\,\forall j\in \{1,\dots, m\} \
    \end{aligned}
\]
\textit{Vincoli}:
\[
    \begin{aligned}
        \forall i\in\{1,\dots,n\}\quad \sum_{j=1}^{m}x_{ij} &= 1 \text{ (normale vincolo di semi-assegnamento!!)}\\
        \forall j\in\{1,\dots,m\}\quad y&\geq \sum^n_{i=1}x_{ij}t_i 
    \end{aligned}
\]
\textit{Funzione obbiettivo}:
\[
    \min (\underbrace{{\max_j \sum^n_{i=1}\underbrace{x_{ij}t_i}_{\text{tempo di compilazione del progetto $i$ al pc $j$}}}}_{\text{nonostante questa espressione matematica ha perfettamente senso, non è lineare!}})
\]

Funzione obbiettivo sbagliata! Pertanto prenderemo $y$ per "simulare" il massimo, di modo da rendere la funzione obbiettivo lineare:
\[
    \min y
\]

\subsection{Problema delle commesse}
\subsubsection{Testo}
Un'azienda deve decidere come impiegare i suoi $n$ dipendenti $1,\dots, n$.

L'azienda, nell'intervallo di tempo desiderato deve evadere $m$ commesse $1,\dots, m$

Ciascuna commessa $j$ deve essere svolta dal sottoinsieme $D_j\subseteq\{1,\dots, n\}$ dei dipendenti dall'azienda. 

Ogni commessa, se evasa darebbe luogo ad un ricavo pari a $r_j$ euro.

Ogni dipendente può lavorare ad una singola commessa nell'unità di tempo 

\subsubsection{Svolgimento}
Si consideri che questo è un Problema di selezione di sottoinsiemi, infatti:
$\quad N=\{1,\dots,n\}$ elementi/dipendenti

$F = \{F_1, \dots, F_m\} \quad \text{dove } F_j$ è la $j$-esima commessa

\[
    a_{ij}= \begin{cases}
        1 & \text{se} i\in F_j\\
        0 & \text{altrimenti}
    \end{cases}
\]

\textit{Variabili}:
\[
    \begin{aligned}
        x_j= \begin{cases}
            1 & \text{se la commessa $j$ è evasa}\\
            0 & \text{altrimenti}
        \end{cases}\\
        x_j\in\mathbb{N} \quad \red{y_j\in\mathbb{N}}
    \end{aligned}
\]

\textit{Vincoli}:
\[
    \begin{aligned}
        0\leq x_j\leq 1 \quad \forall j\in\{1,\dots,m\} \quad \red{0\leq y_j\leq 1}\\
        \forall i\in\{1,\dots,n\}\quad \sum^m_{j=1}a_{ij}x_j &\leq 1 \quad \red{y_j=1-x_j}
    \end{aligned}
\]
\textit{Funzione obbiettivo}:
\[
    \max \sum^m_{j=1}r_jx_j \red{-\sum^m_{j=1}y_ip_j}
\]
