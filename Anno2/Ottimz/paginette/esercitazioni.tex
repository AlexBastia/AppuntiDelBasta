\chapter{Esercitazioni}

\section{Problemi e modelli}
\subsection{Problema dei data center}

\subsubsection{Testo}
Abbiamo $n$ server $1,\dots,n$, ogni server può lavorare in $m_{i\in\mathbb{N}}$ modalità operative diverse. Nella modalità $j\in\{1,\dots,m\}$, il server 1 riesce ad eseguire un numero di istruzioni..

\hspace{1.5cm}

\subsubsection{Svolgimento}

\textit{Variabili}:
\[
    \begin{aligned}
        x_{ij} &= \begin{cases}
            1 & \text{se il server $i$ è utilizzato in modalità $j\in\{1,\dots,m_i\}$} \\
            0 & \text{altrimenti}
        \end{cases}\\
        x_{ij}&\in \mathbb{N}, \quad \forall i\,\forall j
    \end{aligned}
\]
\textit{Vincoli}:
$\forall i\in \{1,\dots, n\},\quad\forall j\in\{1,\dots,m_{i}\}$ con $0\leq x_{ij}\leq 1$ si ha:
\[
    \begin{aligned}
        \forall i\in\{1,\dots,n\}\quad \sum_{j=1}^{m_i}x_{ij} &= 1\\
        \sum^n_{i=1}\sum^{m_i}_{j=1}x_{ij}s_{ij} &\geq k
    \end{aligned}
\]

\textit{funzione obbiettivo}

\[
    \min \sum^n_{i=1}\sum^{m_i}_{j=1}x_{ij}w_{ij}
\]

\subsection{Problema del docente di informatica}
\subsubsection{Testo}
Si hanno dei progetti ($t_1, t_2, \dots, t_n$), si hanno $m$ PC, dove ogni pc può compilare qualunque progetto in modalità sequenziale. Le prestazioni del PC sono identiche 

Vogliamo assegnare i progetti ai pc in modo da minimizzare il tempo complessivo parallelo di compilazione
\subsubsection{Svolgimento}
\textit{Variabili}:
\[
    \begin{aligned}
        x_{ij} &= \begin{cases}
            1 & \text{se il progetto $i$ è compilato al pc $j$} \\
            0 & \text{altrimenti}
        \end{cases}\\
        x_{ij}&\in \mathbb{N}, \quad \forall i\in \{1,\dots, n\}\,\forall j\in \{1,\dots, m\} \
    \end{aligned}
\]
\textit{Vincoli}:
\[
    \begin{aligned}
        \forall i\in\{1,\dots,n\}\quad \sum_{j=1}^{m}x_{ij} &= 1 \text{ (normale vincolo di semi-assegnamento!!)}\\
        \forall j\in\{1,\dots,m\}\quad y&\geq \sum^n_{i=1}x_{ij}t_i 
    \end{aligned}
\]
\textit{Funzione obbiettivo}:
\[
    \min (\underbrace{{\max_j \sum^n_{i=1}\underbrace{x_{ij}t_i}_{\text{tempo di compilazione del progetto $i$ al pc $j$}}}}_{\text{nonostante questa espressione matematica ha perfettamente senso, non è lineare!}})
\]

Funzione obbiettivo sbagliata! Pertanto prenderemo $y$ per "simulare" il massimo, di modo da rendere la funzione obbiettivo lineare:
\[
    \min y
\]

\subsection{Problema delle commesse}
\subsubsection{Testo}
Un'azienda deve decidere come impiegare i suoi $n$ dipendenti $1,\dots, n$.

L'azienda, nell'intervallo di tempo desiderato deve evadere $m$ commesse $1,\dots, m$

Ciascuna commessa $j$ deve essere svolta dal sottoinsieme $D_j\subseteq\{1,\dots, n\}$ dei dipendenti dall'azienda. 

Ogni commessa, se evasa darebbe luogo ad un ricavo pari a $r_j$ euro.

Ogni dipendente può lavorare ad una singola commessa nell'unità di tempo 

\subsubsection{Svolgimento}
Si consideri che questo è un Problema di selezione di sottoinsiemi, infatti:
$\quad N=\{1,\dots,n\}$ elementi/dipendenti

$F = \{F_1, \dots, F_m\} \quad \text{dove } F_j$ è la $j$-esima commessa

\[
    a_{ij}= \begin{cases}
        1 & \text{se} i\in F_j\\
        0 & \text{altrimenti}
    \end{cases}
\]

\textit{Variabili}:
\[
    \begin{aligned}
        x_j= \begin{cases}
            1 & \text{se la commessa $j$ è evasa}\\
            0 & \text{altrimenti}
        \end{cases}\\
        x_j\in\mathbb{N} \quad \red{y_j\in\mathbb{N}}
    \end{aligned}
\]

\textit{Vincoli}:
\[
    \begin{aligned}
        0\leq x_j\leq 1 \quad \forall j\in\{1,\dots,m\} \quad \red{0\leq y_j\leq 1}\\
        \forall i\in\{1,\dots,n\}\quad \sum^m_{j=1}a_{ij}x_j &\leq 1 \quad \red{y_j=1-x_j}
    \end{aligned}
\]
\textit{Funzione obbiettivo}:
\[
    \max \sum^m_{j=1}r_jx_j \red{-\sum^m_{j=1}y_ip_j}
\]

\subsection{Problema della minimizzazione del massimo}
\subsubsection{Testo}
Una ditta di costruzioni edilizie ha deciso di subappaltare $n$ diverse opere ad $n$ diversi artigiani. Ad ogni artigiano $i = 1,\dots , n$ chiede di fornire il costo preventivo $c_{ij}$ che richiede per effettuare l’opera $j$, per ogni $j = 1, \dots, n$. Si vuole assegnare un’opera a ciascun artigiano in modo che tutte le opere siano effettuate e il costo massimo dei subappalti assegnati sia minimizzato. Formulare il problema

\subsubsection{Svolgimento}
\textit{Variabili}:
\[
    x_{ij} = \begin{cases}
        1 & \text{se l'opera $j$ è assegnata all'artigiano $i$}\\
        0 & \text{altrimenti}
    \end{cases}
\]

\textit{Vincoli}:
\[
    \forall j \sum^n_{i=1}x_{ij}=1 \quad \forall i \sum_{j=1}^n x_{ij}=1 
\]
Inoltre: 
\[
    \forall i, j\; c_{ij}x_{ij}\leq M
\]
\textit{Funzione obbiettivo}:
\[
    \min M
\]

\subsection{Problema della massimizzazione del minimo}
\subsubsection{Testo}
Si provi ora a massimizzare il costo minimo dei subappalti assegnati
\subsubsection{Svolgimento}
\textit{Variabili}:
\[
    x_{ij} = \begin{cases}
        1 & \text{se l'opera $j$ è assegnata all'artigiano $i$}\\
        0 & \text{altrimenti}
    \end{cases}
\]
\textit{Vincoli}:
\[
    \forall j \sum^n_{i=1}x_{ij}=1 \quad \forall i \sum_{j=1}^n x_{ij}=1
\]
Inoltre:
\[
    \forall i, j\; c_{ij}x_{ij}\geq m
\]
\textit{Funzione obbiettivo}:
\[
    \max m
\]

\subsection{Esercizio valore assoluto}
\subsection{testo}
Formulare il problema $\min\{|3 - 4x| : |x| \leq 2\}$
\subsection{Svolgimento}
\textit{Vincoli}
\[
    x \leq 2 \quad -x\leq 2
\]
\textit{Funzione obbiettivo}
\[
    \min\{3-4x\} \quad \min\{3-4x\}
\]

\section{1.26}

\textit{Variabili}

$ x_1,x_2 = \text{quantita di V1,V2} $

$ y_1,y_2,y_3 = \text{quantita di N1, N2, N3} $

$ z = \text{totale degli oli aquistati} $

\textit{Vincoli}

$ 0 \leq x_1 + x_2 \leq 200 $

$ 0 \leq y_1 + y_2 + y_3 \leq 250 $

$ z 3 \leq d_1x_1 + d_2x_2 + d_3y_1 + d_4y_2 + d_5y_3 \leq z 6  $

$ z = x_1 + x_2 + y_1 + y_2 + y_3 $ Dobbiamo fare in modo che $ z $ sia effetivamente il valore che gli vogliamo dare

\textit{Funzione Obbiettivo}

\[
 max\ 150 z - (110 x_1 + 120 x_2 + 130 y_1 + 110 y_2 + 115 y_3)
\]

\section{1.27}

\textit{Variabili}

$ x_1, x_2, x_3 = \text{numero di dolci A,B o C} $

$ x_1,x_2, x_3 \in \mathbb{N} $

\textit{Vincoli}

$ x_1 + x_2 + x_3 \leq 10000 $

$ x_1 \leq 0.5 x_1 + 0.5 x_2 + 0.5 x_3 $

$ x_3 \leq 0.25 x_2 $

\textit{Funzione Obbiettivo}

\[
0.2 x_1 + 0.1 x_2 + 0.4 x_3
\]

\section{1.28}

\textit{Variabili}

$ x_1,x_2,x_3 = \text{numero di beni prodotti con $ P_1, P_2, P_3 $} \in \mathbb{N} $

\textit{Vincoli}

$ 2x_1 + x_2 + 3x_3 \leq 50 $

$ 4x_1 + 2x_2 + 3x_3 \leq 50$

$ 3x_1 + 4 x_2 + 2x_3 \leq 50 $

$ 15 x_1 + 18 x_2 + 10 x_3 \geq 200 $

\textit{Funzione Obbiettivo}
\[
min \quad 4x_1 + 2 x_2 + 3 x_3
\]

\section{1.29}

\textit{Variabili}

$ x_1, x_2, x_3, x_4,x_5, x_6 = \text{numero ostetriche nuove a turno} \in \mathbb{N} $

\textit{Vincoli}

$ x_1 \geq 70 $

$ x_1 + x_2 \geq 80 $

$ x_2 + x_3 \geq 50 $

$ x_3 + x_4 \geq 60 $

$ x_4 + x_5 \geq 40 $

$ x_5 + x_6 \geq 30 $

\textit{Funzione Obbiettivo}
\[
min \quad x_1 + x_2 + x_3 +x_4 +x_5 +x_6
\]

\section{1.31}

\textit{Variabili}

$ \forall i = 1,...,7 \quad x_i = \text{variabili ogiche che indicano il fatto che un certo abitante $ a_i $ sia o meno membro del consiglio} $

\textit{Vincoli}

$ \sum_{i=1}^{7} x_i = 4 $

$ \forall i = 1,...,4 \sum_{x \in C_i} x = 1  $

$  $

\section{1.32}

\textit{Variabili}

$ x_i = \text{variabili logiche che mi dicono se scelgo i} $

\textit{Funzione Obbiettivo}

\[
max \quad \sum_{i=1}^{n} x_ib_i
\]

\textit{Vincoli}

$ \sum_{i=1}^{n} x_i = 11 $

$ 1 \leq \sum_{i \in C} x_i \leq 5 $

$ 1 \leq \sum_{i \in P} x_i \leq 1 $

$ 1 \leq \sum_{i \in D} x_i \leq 6 $

$ 1 \leq \sum_{i \in A} x_i \leq 4 $

$ \forall i = 1,...,m. \quad 0 \leq \sum_{i \in L_i} x_i  \leq 1 $

\section{1.35}

\textit{Variabili}

$ x_{ik} = \text{variabile logica che dice se teniamo l'impianto i per il mercato k} $

$ y_i = \text{variabile logica che dice se l'impianto i rimane aperto} $


\textit{Funzione obbiettivo}

$ min \qued \sum_{k \in K} \sum_{i \in I \cup J} x_{ik} c_{ik} b_k $

\textit{Vincoli}

$ \forall k \in K. \qued \sum_{i \in I \cup J} x_{ik} = 1 $

$ \sum_{i \in I} x_{ik} \geq \frac{|I|}{2} $ non va bene! possiamo contare piu' volte lo stesso impianto che pero' funziona su diversi mercati -> variabili ausiliarie 

$ \forall i \in I \cup J.\quad y_i \leq \sum_{k \in K} x_{ik} $ se tutti 0, allora y 0
$ y_i \geq x_{ik} $ disgiunzione logica generalizzata a k elemetni

$2 \sum_{i \in I} y_i \geq |I| $

$2 \sum_{i \in J} y_i \geq |J| $


