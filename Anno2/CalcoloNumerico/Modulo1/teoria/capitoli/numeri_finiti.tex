\chapter{I numeri finiti}
\section{Base-N and Binary}

I numeri hanno diversi modi per essere rappresentati, tra cui quello più comune che è la \textbf{rappresentazione decimale}, dove ogni numero è formato da una lista di caratteri compresi tra $[0,9]$ dove ad ogni cifra è associata una potenza del 10
\\
\esempio{
    \textbf{Esempio}: $147.3 = 1 \cdot 10^2 + 4 \cdot 10^1 + 7 \cdot 10 ^0 + 3 \cdot 10^{-1}$ 
}

In generale, comunque, per convertire un numero da una base \textit{n} a base 10:
\teorema{
    Dato un numero in base \(n\) \( (a_k a_{k-1} \dots a_1 a_0)_n \), la sua conversione in base 10 è data dalla formula:

    \[
    \sum_{i=0}^{k} a_i \cdot n^i
    \]

    dove \(a_i\) sono le cifre in base \(n\) e \(n\) è la base.

}

Per forza di cose i calcolatori operano con i numeri \textit{in base 2}, le cui cifre vengono chiamate\textbf{bit}, esempio $101101$ (base 2).
Per convertire un numero dalla base 10 alle base bisogna dividerlo in potenze di due 
\esempio{
    \[
        37 \, (\text{base } 10) = 32 + 4 + 1 = 1 \cdot 2^5 + 0 \cdot 2^4 + 0 \cdot 2^3 + 1 \cdot 2^2 + 0 \cdot 2^1 + 1 \cdot 2^0 = 100101 \, (\text{base } 2)
    \]
}

In generale occorre dividerlo per due finchè non si riduce ad uno ad esempio: 
\[
(35)_{10} = (100011)_2
\]

\begin{align*}
    35 : 2 &= 17 \quad \text{resto } 1 \\
    17 : 2 &= 8  \quad \text{resto } 1 \\
    8 : 2  &= 4  \quad \text{resto } 0 \\
    4 : 2  &= 2  \quad \text{resto } 0 \\
    2 : 2  &= 1  \quad \text{resto } 0 \\
    1 : 2  &= 0  \quad \text{resto } 1
\end{align*}
    
    % Creiamo la freccia accanto a tutto l'insieme di divisioni TODO: fixa la freccia
    \begin{tikzpicture}[overlay, remember picture]
        \draw[->, red, thick] (5.5,1) -- (5.5,-5.5);
    \end{tikzpicture}


\subsection{Addizione e moltiplicazione in binario}
\subsection{Conversione fra basi}

\chapter{Floating point}

\section{Cosa sono i floating point?}
Dato che in informatica si lavora con calcolatori con memoria finita, non è possibile implementare il concetto di infinito (a differenza della matematica). Quindi non si è in grado di rappresentare tutto l'insieme dei reali, pertanto occorre approssimarlo. Il metodo che oggi è comune per il calcolo scientifico è noto come sistema \textbf{floating point} che permette di rappresentare un ampio intervallo della retta reale con una distribuzione uniforme degli errori.\\
\\
L'idea di base e' quella di scrivere tutti i numeri come prodotto fra un indicatore $ s $ per il segno, un valore decimale $ f < 1 $ e la base elevata da un esponente intero $ \beta^e $:
\[
  \forall n \in \mathbb{R}. n = (s) f \beta^e
\]
\ex{}{
  E' lo stesso procedimento che si usa in base 10 in notazione scientifica:
  \[
    0,0000003467 = 0,3467 * 10^{-6}
  \]
  \[
    -987380000000 = -0,98738 * 10^{12}
  \]
}

\subsection{Definizione formale}
\dfn{Numeri macchina}{
    Si definisce \textbf{inseime dei numeri numeri macchina}  (floating point) con $t$ cifre significative, con base $\beta$, con $p \in [L,U]$ e con le cifre $d_i$ dove $0 \leq d_i \leq \beta -1$, $i = 1, 2, \dots$ e $d_1 \neq 0$
    \[
        \mathbb{F}(\beta, t, L, U) =  \{ 0 \} \cup \{ x \in \mathbb{R} = \text{sign}(x) \beta^{p} \sum_{i=1}^{t} d_i \beta^{-i} \}
    \]
    Usualmente $U$ è positivo e $L$ negativo
}

Questo insieme è, quindi, un'approssimazione all'intervallo preciso di numeri $[min, max] \subseteq \mathbb{R}$ che dipendono da $U$ e $L$. Se voglio rappresentare un numero maggiore/minore di questo intervallo avrò un errore chiamato \textbf{overflow/underflow}.

\section{Gap}
Con questa notazione non è impossibile scrivere ogni numero presente nella retta dei numeri Reali $\mathbb{R}$ (se ammettiamo che $ t $ possa essere infinito); ma, quando lavoriamo con calcolatori che hanno memoria finita, tra due numeri vi sara' sempre una certa distanza ($ t $ finito), quindi è pertanto vero che $\mathbb{F}\subseteq \mathbb{R}$. Questo concetto introduce la definizione di Gap:
\dfn{Gap}{
    Chiamiamo \textbf{gap} la distanza da un numero al suo sucessivo
}
È inoltre vero che il gap rimane costante quando l'esponente non cambia, mentre se nell'insieme $\mathbb{F}$ il sucessivo di un numero ha un espente diverso il suo gap sarà diverso di quello del suo precedente con l'esponente uguale. Facciamo un esempio per rendere il tutto più chiaro:
\ex{}{
    Sia $\beta = 2, t = 3, L = -1, U=2$ allora
    \[
        \mathbb{F}(2,3,-1,2) = \{0\}\cup\{0.100\times2^p, 0.101\times2^p, 0.110\times2^p,0.111\times2^p, p = -1, 0,1,2\}
    \]
    Questo insieme rappresenta 33 nuemri compreso lo 0:

    %! RIFARE BENE
    \begin{multicols}{4}
        \[
        \begin{array}{c}
        .100(-1) \\
        \frac{1}{4} \\
        .100(0) \\
        \frac{1}{2} \\
        .100(1) \\
        1 \\
        .100(2) \\
        2
        \end{array}
        \]
        \columnbreak
        \[
        \begin{array}{c}
        .101(-1) \\
        \frac{5}{16} \\
        .101(0) \\
        \frac{5}{8} \\
        .101(1) \\
        \frac{5}{4} \\
        .101(2) \\
        \frac{5}{2}
        \end{array}
        \]
        \columnbreak
        \[
        \begin{array}{c}
        .110(-1) \\
        \frac{6}{16} \\
        .110(0) \\
        \frac{6}{8} \\
        .110(1) \\
        \frac{6}{4} \\
        .110(2) \\
        \frac{6}{2}
        \end{array}
        \]
        \columnbreak
        \[
        \begin{array}{c}
        .111(-1) \\
        \frac{7}{16} \\
        .111(0) \\
        \frac{7}{8} \\
        .111(1) \\
        \frac{7}{4} \\
        .111(2) \\
        \frac{7}{2}
        \end{array}
        \]
    \end{multicols}
    La rappresentazione di $\mathbb{F}$ nella retta dei numeri reali è questa
    \begin{center}
      \includegraphics[width=10cm]{./img/retta_reale.png}
    \end{center}
}

Risulta quindi evidente l'aumento dell'ampiezza degli intervalli definiti dal valore $p$, in ognuno dai quali vengono localizzati le $t$ cifre della mantissa; ne risulta di conseguenza una diradazione delle suddvisioni verso gli estremi sinistro e destro della retta reale. \\
Il numero $t$, quindi, fissa la precisione di rappresentazione di ogni numero; infatti la retta viene suddivisa in intervalli $[\beta^p,\beta^{p+1}]$  di ampiezza crescente e in questi intervalli vengono rappresentati lo stesso numero $(\beta-1)\beta^{t-1}$ di valori, generando ,così, una suddivisione molto densa per valori vincini allo 0 e più rada per valori grandi in valore assoluto. Di conseguenza si ha che \textbf{maggiore è il numero $t$ di cifre della mantissa, minore sarà l'ampiezza dei intervalli e quindi migliore l'approssimazione introdotta}.

\section{floating point nei calcolatori}
La rappresentazione più comune del sistema nei calcolatori è definita dallo Standand IEEE 745 il cui insieme è $\mathbb{F}(2,52,1023,1024)$ ed è descritta dalla formula
\[
    n = (-1)^s \cdot 2^{e-1023} \cdot (1+f)   
\]

Dove: 
\begin{itemize}
    \item \textbf{s}: è un bit che determina il segno del numero. Se $s=0$ il numero è positivo, se $s=1$ il numero è negativo.
    \item \textbf{e}: è l'esponente del numero due, ed è un intero di 11 bit tale che $e \in [0,2047]$.
    \item \textbf{f}: il valore detto \textbf{mantissa} è una serie di 52 bit che rappresenta la parte frazionaria del numero, nel nostro caso $f \in [0,1]$.
    \item \textbf{1023}: è detto \textbf{bias} ed è usato per codificare esponenti negativi e positivi.
\end{itemize}
In memoria si traduce in un array contentente questi bit:
\begin{center}
    \includegraphics[width=7cm]{./img/floating_point_in_memoria.png}
\end{center}

\ex{}{
    Per trovare la codifica del numero in floating point 1 10000000010 1000000000000000000000000000000000000000000000000000 in base 10 occorre:
    \begin{enumerate}
        \item Trovare l'esponente decimale, che è $1 \times 2^{10} + 1 \times 10^1 - 1023 = 3$.
        \item Trovare la parte frazionaria, che è $1 \times \frac{1}{2^1} = 0,5$.
    \end{enumerate}

    Quindi $n = (-1)^1 \times 2^3 \times (1+0,5) = -12,0$.
}

\subsection{troncamento}

Sia l'insieme $\mathbb{F}(\beta, t, L, U) =\{ \{\emptyset\} \cup n = \pm \beta^p \cdot d_0 d_1 \dots d_t  \} $ con $-L \leq p \leq U$ l'insieme dei numeri floating point caratterizzati dai valori $\beta, t, L, U$

Inoltre sia $x \in \mathbb{R}$ dove $x = \pm \beta^p \cdot d_0 d_1 \dots d_t, d_{t+1}, \dots$ con $L\leq p \leq U$, nel caso in cui $x \notin \mathbb{F} $ rappresento $x$ con un elemento di $\mathbb{F} $ che indico con $fl(x) = \pm \beta^p \cdot d_0 d_1 \dots d_t$. Questa operzione si chiama \textbf{troncamento}

\subsection{Errore di arrotondamento}
Essendo un'approssimazione, il sistema floating point non puo' rappresentare perfettamente tutti i valori reali, in quanto i numeri infiniti (come il $\pi$) vengono delineati con una serie finita di numeri finiti. La differenza tra la sua approssimazione e il suo valore reale è detta \textbf{errore di arrotondamento}. Sia $x\in \mathbb{R}$ un numero reale e sia $fl(x)$ (dove $fl()$ è la funzione arrotondamento) il numero $x$ arrotondato, allora l'errore di arrotondamento puo' essere espresso come:
\begin{itemize}
\item L'errore \textbf{assoluto}, che e' piu' grande piu' e' grande l'ordine di $ x \in \mathbb{R}$
  \[
    |fl(x)-x|<\beta^{p-t}
  \]
\item E quello \textbf{relativo}, che dipende solo da $ t $
  \[
    \frac{|fl(x)-x|}{|x|} < \frac{1}{2}\beta^{1-t}
  \]
\end{itemize}
 La quantità $esp=\frac{1}{2}\beta^{1-t}$ è detta  \textbf{precisione macchina} nel sistema floating point, ed è il più piccolo numero macchina positivo tale che: 
 \[
    fl(1+esp)>1
 \]
\subsection{artimetica floating point}
Il calcolatore esegue operazioni solo con numeri rappresentabili da calcolatore stesso, quindi in questo caso dai numeri scritti in forma floating ponti (per i nuemri reali), il punto è che \red{le usuali operazioni aritmetiche non sono chiuse nell'insieme $\mathbb{F}$}, quindi presi 2 numeri dall'insieme $\mathbb{F}$ e computati tramite operazioni aritmetiche, il risultato di tale computazione potrebbe appartenere nell'insieme $\mathbb{R}$. Pertanto \red{i risultati delle varie computazioni dovranno essere sempre arrotondati ad un numero che stia in $\mathbb{F}$}. Definiamo due operazioni generiche, una per i reali e l'altra per i floating-point:
\begin{itemize}
\item $ \cdot: \mathbb{R} \times \mathbb{R} \to \mathbb{R} $
\item $ \circ: \mathbb{F} \times \mathbb{F} \to \mathbb{F} $
  \[
    x \circ y = fl(x \cdot y)
  \]
\end{itemize}
L'operazione floating point nei calcolatori viene strutturata in due fase:
\begin{enumerate}
    \item \textbf{Eseguo l'operazione esatta}: (es. $z=x+y$) dove viene fatta utilizzando più bit utilizzati di quelli utilizzati per memorizzare il risultato, non è accessibile all'utente ma viene memorizzato in un registro interno alla cpu
    \item \textbf{Trocamento del risultato}: (es. $x\oplus y = fl(z)$), dove una volta che il calcolo è stato fatto in precisione estesa il risultato viene arrotondato alla precisione del risultato
\end{enumerate}
\esempio{
    \includegraphics[width=10cm]{./img/arrotondamento_esempio.png}
}

Se un numero $ x \in \mathbb{R} $ ha un numero di cifre decimali maggiore di $ t $, allora $ x \not\in \mathcal{F} $ e deve essere approssimato (solitamente col troncamento).
Percio' in generale le operazioni floating-point generano un errore (di \textbf{arrotondamento}):
\[
  \left|\frac{(x \circ y) - (x \cdot y)}{x \cdot y}\right| < \text{eps}
\]  
Nel caso di somma di numeri di grandezza molto diversa, le cifre piu' significative del numero piu' piccolo possono essere perse. Per una sola operazione questo non causa un errore troppo grande, ma se l'operazione viene eseguita molte volte l'errore si \textbf{amplifica} e puo' diventare catastrofico.


