\section{Numeri finiti}
\subsection{Numeri binari ed in base N}
I numeri hanno diversi modi per essere rappresentati, tra cui quello più comune che è la \textbf{rappresentazione decimale}, dove ogni numero è formato da una lista di caratteri compresi tra $[0,9]$ dove ad ogni cifra è associata una potenza del 10
\\
\esempio{
    \textbf{Esempio}: $147.3 = 1 \cdot 10^2 + 4 \cdot 10^1 + 7 \cdot 10 ^0 + 3 \cdot 10^{-1}$ 
}

In generale, comunque, per convertire un numero da una base \textit{n} a base 10:
\teorema{
    Dato un numero in base \(n\) \( (a_k a_{k-1} \dots a_1 a_0)_n \), la sua conversione in base 10 è data dalla formula:

    \[
    \sum_{i=0}^{k} a_i \cdot n^i
    \]

    dove \(a_i\) sono le cifre in base \(n\) e \(n\) è la base.

}

Per forza di cose i calcolatori operano con i numeri \textit{in base 2}, le cui cifre vengono chiamate\textbf{bit}, esempio $101101$ (base 2).
Per convertire un numero dalla base 10 alle base bisogna dividerlo in potenze di due 
\esempio{
    \[
        37 \, (\text{base } 10) = 32 + 4 + 1 = 1 \cdot 2^5 + 0 \cdot 2^4 + 0 \cdot 2^3 + 1 \cdot 2^2 + 0 \cdot 2^1 + 1 \cdot 2^0 = 100101 \, (\text{base } 2)
    \]
}

In generale occorre dividerlo per due finchè non si riduce ad uno ad esempio: 
\[
(35)_{10} = (100011)_2
\]

\begin{align*}
    35 : 2 &= 17 \quad \text{resto } 1 \\
    17 : 2 &= 8  \quad \text{resto } 1 \\
    8 : 2  &= 4  \quad \text{resto } 0 \\
    4 : 2  &= 2  \quad \text{resto } 0 \\
    2 : 2  &= 1  \quad \text{resto } 0 \\
    1 : 2  &= 0  \quad \text{resto } 1
    \end{align*}
    
    % Creiamo la freccia accanto a tutto l'insieme di divisioni
    \begin{tikzpicture}[overlay, remember picture]
        \draw[->, red, thick] (5.5,1) -- (5.5,-5.5);
    \end{tikzpicture}
    
    

