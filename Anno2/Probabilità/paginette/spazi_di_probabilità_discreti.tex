\chapter{Spazi di probabilità discreti}
\section{Concetti introduttivi}
Innanzi tutto andiamo a definire che cosa intendiamo per \textit{esperimento aleatorio, esito, probabilità}

Con la dicitura esperimento \textit{aleatorio indicheremo} qualunque fenomeno (fisico, economico, sociale, \dots ) il cui esito non sia determinabile con certezza a priori. Il nostro obiettivo è di fornire una descrizione matematica di un esperimento aleatorio, definendo un modello probabilistico, un \textit{esito} invece è un ipotetico risultato di un'esperimento aleatorio sulla base di un cosiddetto \textit{spazio campionario} un insieme che contiene tutti gli esiti possibili dell’esperimento

\esempio{
    \begin{itemize}
        \item \textbf{Esperimento aleatorio:} Lancio di un dado.
        \item \textbf{Spazio campionario:} $\Omega = \{1, 2, 3, 4, 5, 6\}$.
        \item \textbf{Esito:} $4$.
    \end{itemize}
}

Adesso forniamo vere e priorie definizioni
\dfn{evento}{
    Si definisce \textbf{evento} un'affermazione riguardante l'ipotetico esito univoco dell'esperimento, di cui si può affermare con certezza se è vero o falso una volta noto l'esito
}
\ex{}{
    Esper. aleatorio: Lancio del dado\\
    $A = \text{"esce un numero pari"}$
}

